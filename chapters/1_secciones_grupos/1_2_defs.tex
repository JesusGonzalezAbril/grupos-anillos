\clearpage
\section{Definiciones y ejemplos}

\begin{definition}{Grupo}{grupo}
    Un grupo es una pareja \((G,\cdot)\), formada por un conjunto no vacío \(G\) junto con una operación binaria, que denotaremos por \(\cdot\), que satisface los siguientes axiomas:
    \begin{enumerate}
    \item {(Asociativa)} \((a\cdot b)\cdot c = a\cdot(b\cdot c)\), para todo \(a,b,c\in G\).
    \item {(Neutro)} Existe un elemento \(e\in G\), llamado {elemento neutro del grupo}, tal que \(e\cdot a = a = a\cdot e\), para todo \(a\in G\).
    \item {(Inverso)} Para todo \(a\in G\) existe otro elemento \(a^{-1}\in G\), llamado {elemento inverso} de \(a\), tal que \(a\cdot a^{-1} = e = a^{-1}\cdot a\).
    \end{enumerate}

    Si además se verifica el siguiente axioma se dice que el grupo es {abeliano} o {conmutativo}:

    \begin{enumerate}
    \setcounter{enumi}{3}
    \item {(Conmutativa)} \(a\cdot b = b\cdot a\), para todo \(a,b\in G\).
    \end{enumerate}
\end{definition}

Demostraremos ahora algunas propiedades de los grupos.

\begin{lemma}{Propiedades básicas de grupos}{prop_grupos}
    Sea \((G,\cdot)\) un grupo.
    \begin{enumerate}
    \item {(Unicidad del neutro)} El neutro de \(G\) es único y lo denotaremos \(e\). De hecho, si \(a,b\in G\) satisfacen que \(a\cdot b = a\) ó \(b\cdot a = a\) entonces \(b = e\).
    \item {(Unicidad del inverso)} El inverso de un elemento \(a\) de \(G\) es único y lo denotaremos \(a^{-1}\). De hecho, si \(e\) es el neutro de \(G\) y \(a,b\in G\) satisfacen \(a\cdot b = e\) ó \(b\cdot a = e\) entonces \(b = a^{-1}\).
    \item {(Propiedad Cancelativa)} Todo elemento de \(G\) es cancelativo.
    \item Para todo \(a,b\in G\), las ecuaciones \(a\cdot X = b\) y \(X\cdot a = b\) tienen una única solución en \(G\).
    \item \((a\cdot b)^{-1} = b^{-1}\cdot a^{-1}\).
    \end{enumerate}
\end{lemma}

\begin{proofbox}
    \begin{enumerate}
    \item
        Haremos solo el caso por la derecha, en efecto, si $a \cdot b = a$ entonces
        \[
        b = e \cdot b = a^{-1} \cdot a \cdot b = a^{-1} \cdot a = e.
        \]

    \item
        De nuevo hacemos solo el caso \(a\cdot b = e\)
        \[
        a^{-1} = a^{-1} \cdot e = a^{-1} \cdot a \cdot b = e \cdot b = b.
        \]

    \item
        Sea $x \in G$, entonces $x$ debe ser cancelable puesto que en caso contrario existirían $a,b \in G$ con $a \neq b$ tales que $x \cdot a = x \cdot b$, pero entonces
        \[
        a = e \cdot a = x^{-1} \cdot x \cdot a = x^{-1} \cdot x \cdot b = e \cdot b = b
        \]
        una contradicción.

    \item
        Sean $a,b \in G$ arbitrarios y $x,y$ dos soluciones cualesquiera, entonces
        \[
        x = e \cdot x = a^{-1} \cdot a \cdot x = a^{-1} \cdot b
        \]
        y de igual manera
        \[
        y = e \cdot y = a^{-1} \cdot a \cdot y = a^{-1} \cdot b
        \]
        luego $x=y$. Para la otra ecuación se razona igual. Notemos que también hemos demostrado la existencia de una solución ($x = a^{-1} \cdot b$).

    \item
        Basta realizar un sencillo cálculo y aplicar el apartado 2
        \[
        (a \cdot b) \cdot (b^{-1} \cdot a^{-1}) = a \cdot b \cdot b^{-1} \cdot a^{-1} = a \cdot e \cdot a^{-1} = a \cdot a^{-1} = e.
        \]

    \end{enumerate}
\end{proofbox}

\subsection{Ejemplos}

\begin{example}{Grupo trivial}{grupo_trivial}
    Sea \(X\) un conjunto y consideremos la aplicación identidad $1_X: X \to X$ tal que $1_X(x) = x$ para todo $x \in X$. Entonces el conjunto $T = \{1_X\}$ con la operación de composición es un grupo $(T, \circ)$ que llamaremos el grupo trivial (lo denotaremos $1$).
    
    En general, podríamos haber definido este grupo como un único elemento $\{x\}$ con la operación descrita por $x \cdot x = x$.
\end{example}

\begin{example}{Grupo simétrico}{grupo_simetrico}
    Sean \(X\) un conjunto y \(S_X\) el conjunto de todas las biyecciones de \(X\) en sí mismo. Entonces \((S_X, \circ)\) es un grupo, llamado {grupo simétrico} o {grupo de las permutaciones} de \(X\).
\end{example}

\begin{proofbox}
    Prescindiremos del uso de $\circ$ para simplificar la notación.
    \begin{enumerate}
        \item Asociativa: sean $f,g,h$ biyecciones, dado $x \in X$ cualquiera
        \[
            ((fg)h) x = (fg)(h(x)) = f(g(h(x))) = f(gh(x)) = (f(gh))x \implies (fg)h = f(gh)
        \]
        \item Neutro: basta considerar la aplicación identidad $id(x) = x$.
        \item Inverso: claramente el inverso de una biyección cualquiera $f$ es su inversa $f^{-1}$, que verifica
        \[
        (f f^{-1})(x) = f(f^{-1}(x)) = x
        \]
        luego $ff^{-1} = id$.
    \end{enumerate}
\end{proofbox}

\begin{remark}
    En general $S_X$ no es un grupo abeliano.
\end{remark}

\begin{example}{Producto de grupos}{prod_grupos}
    Si \((G, *)\) y \((H, *)\) son dos grupos, entonces el {producto directo} \(G \times H\) es un grupo en el que la operación viene dada componente a componente:
    \[
    (g_1, h_1) \cdot (g_2, h_2) = (g_1 * g_2, h_1 * h_2).
    \]
    Más generalmente, si \((G_i)_{i\in I}\) es una familia arbitraria de grupos, entonces el producto directo \(\prod_{i\in I} G_i\) tiene una estructura de grupo en el que el producto se realiza componente a componente. Para más información ver la Definición \ref{defn:producto_cartesiano}.
\end{example}

Probemos que el producto directo de dos grupos es un grupo:

\begin{proofbox}
    \begin{enumerate}
        \item Asociativa:
        \begin{align*}
            ((g_1, h_1) \cdot (g_2, h_2)) \cdot (g_3, h_3) &= (g_1 * g_2, h_1 * h_2) \cdot (g_3, h_3) = (g_1 * g_2 * g_3, h_1 * h_2 * h_3) = \\
            &= (g_1, h_1) \cdot (g_2 * g_3, h_2 * h_3) = (g_1, h_1) \cdot ((g_2, h_2) \cdot (g_3, h_3))
        \end{align*}
        donde hemos usado la asociatividad de los grupos $G, H$.
        \item Neutro: basta considerar el elemento $(e_G, e_H)$ donde $e_G$ es el neutro de $G$ y $e_H$ el de $H$.
        \item Inverso: claramente el inverso de un elemento cualquiera $(g_1, h_1)$ es $(g_1^{-1}, h_1^{-1})$, que verifica
        \[
        (g_1, h_1) \cdot (g_1^{-1}, h_1^{-1}) = (g_1 * g_1^{-1}, h_1 * h_1^{-1}) = (e_G, e_H).
        \]
    \end{enumerate}
\end{proofbox}

\begin{example}{Tabla de Cayley}{tabla_cayley}
    Dado un grupo finito podemos construir lo que llamaremos su tabla de Cayley (también llamada tabla de multiplicación o de suma, dependiendo del nombre que le demos a la operación del grupo). Esta tabla se obtiene disponiendo cada uno de los elementos del grupo tanto por columnas como por filas y calculando sus productos. Si el grupo tiene 2 elementos $a,b$ la tabla será de la forma:
    \begin{center}
        \begin{tabular}{c | c | c}
            $\cdot$ & $a$ & $b$ \\
            \hline
            $a$ & $a \cdot a$ & $a \cdot b$  \\ 
            \hline
            $b$ & $b \cdot a$ & $b \cdot b$ \\
            \hline
        \end{tabular}
    \end{center}

    Como ejemplo concreto, la tabla del grupo $\Z_3$ (enteros módulo 3) es la siguiente:
    \vspace{10pt}
    \begin{center}
        \begin{tabular}{c | c | c | c}
            $+$ & $0$ & $1$ & $2$ \\
            \hline
            $0$ & $0$ & $1$ & $2$ \\ 
            \hline
            $1$ & $1$ & $2$ & $0$ \\ 
            \hline
            $2$ & $2$ & $0$ & $1$ \\ 
            \hline
        \end{tabular}
    \end{center}
\end{example}

\subsection{El grupo diédrico}

Veamos ahora un grupo con especial significado geométrico. Consideremos un polígono regular de \(n\) lados y las transformaciones que lo dejan invariantes (rotaciones y reflexiones), a las que llamaremos simetrías. La composición de dos simetrías de un polígono regular es nuevamente una simetría de este objeto. Considerando la composición de simetrías como operación binaria, esto le da a las simetrías la estructura algebraica de un grupo finito.

La siguiente tabla de Cayley muestra el efecto de la composición en el grupo diédrico de orden 6, \(D_3\) -- las simetrías de un triángulo equilátero. Aquí, \(r_0\) denota la identidad, \(r_1\) y \(r_2\) denotan rotaciones en sentido antihorario de \(120^\circ\) y \(240^\circ\) respectivamente, mientras que \(s_0\), \(s_1\) y \(s_2\) denotan reflexiones a través de las tres líneas mostradas en la Figura \ref{fig:diedrico}.

\begin{figure}[h]
    \centering
    \includegraphics[width=5cm]{img/diedrico.png}
    \caption{Simetrías del triángulo.}
    \label{fig:diedrico}
\end{figure}

\[
\begin{array}{c|cccccc}
\circ & r_0 & r_1 & r_2 & s_0 & s_1 & s_2 \\
\hline
r_0 & r_0 & r_1 & r_2 & s_0 & s_1 & s_2 \\
r_1 & r_1 & r_2 & r_0 & s_1 & s_2 & s_0 \\
r_2 & r_2 & r_0 & r_1 & s_2 & s_0 & s_1 \\
s_0 & s_0 & s_2 & s_1 & r_0 & r_2 & r_1 \\
s_1 & s_1 & s_0 & s_2 & r_1 & r_0 & r_2 \\
s_2 & s_2 & s_1 & s_0 & r_2 & r_1 & r_0 \\
\end{array}
\]

Por ejemplo, \(s_2 s_1 = r_1\), porque la reflexión \(s_1\) seguida de la reflexión \(s_2\) resulta en una rotación de \(120^\circ\). El orden de los elementos que denotan la composición es de derecha a izquierda, reflejando la convención de que el elemento actúa sobre la expresión a su derecha. La operación de composición no es conmutativa.

El siguiente ejemplo abstrae y generaliza el concepto de grupo diédrico prescindiendo de la interpretación geométrica.

\begin{example}{Grupo diédrico}{diedrico}
    Para cada número natural positivo \(n\) definimos un grupo formado por \(2n\) elementos
    \[
    D_n = \{1, a, a^2, \ldots, a^{n-1}, b, ab, a^2b, \ldots, a^{n-1}b\}
    \]
    en el que la multiplicación viene dada por la siguiente regla:
    \[
    (a^{i_1} b^{j_1})(a^{i_2} b^{j_2}) = a^{[i_1 + (-1)^{j_1} i_2]_n} b^{[j_1 + j_2]_2}
    \]
    donde \([x]_n\) denota el resto de dividir \(x\) entre \(n\). Este grupo se llama {grupo diédrico de orden \(2n\)}.

    El grupo diédrico infinito \(D_\infty\) está formado por elementos de la forma \(a^n b^m\), con \(n \in \mathbb{Z}\) y \(m = 0, 1\) con el producto \((a^{i_1} b^{j_1})(a^{i_2} b^{j_2}) = a^{i_1 + (-1)^{j_1} i_2} b^{[j_1 + j_2]_2}\).
\end{example}

Si ahora consideramos únicamente las rotaciones que dejan invariante un polígono de \(n\) lados obtenemos otro grupo, en este caso con \(n\) elementos, cada uno de ellos correspondiente a rotar por un múltiplo de $\frac{360^\circ}{n}$. El siguiente ejemplo abstrae este grupo de rotaciones.

\begin{example}{Grupo cíclico}{ciclico}
    Para cada número natural positivo \(n\) definimos un grupo \(C_n\) formado por \(n\) elementos
    \[
    C_n = \{1, a, a^2, \ldots, a^{n-1}\},
    \]
    donde \(a\) es un símbolo, y en el que la multiplicación viene dada por la siguiente regla:
    \[
    a^i a^j = a^{[i+j]_n}
    \]
    con notación como en el ejemplo anterior. Este grupo se llama {cíclico de orden \(n\)}.

    También definimos el {grupo cíclico infinito} como el conjunto \(C_\infty = \{a^n : n \in \mathbb{Z}\}\), donde \(a\) es un símbolo y consideramos \(a^n = a^m\) si y solo si \(n = m\), y en el que el producto viene dado por \(a^n \cdot a^m = a^{n+m}\).
\end{example}

\begin{remark}
    Es fácil notar la similitud entre $C_n$ y $\Z_n$, así como entre $C_\infty$ y $\Z$. Más tarde formalizaremos esta intuición probando que estos grupos son equivalentes (isomorfos).
\end{remark}