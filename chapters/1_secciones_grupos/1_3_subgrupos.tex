\clearpage
\section{Subgrupos}

\begin{definition}{Subgrupo}{subgrupo}
Sea $G$ un grupo. Un subconjunto $S$ de $G$ se dice que es un subgrupo si la operación que define la estructura de grupo en $G$ induce también una estructura de grupo en $S$.
\end{definition}

\begin{lemma}{Caracterización de subgrupos}{caracterizacion_subgrupos}
Sean $G$ un grupo y $S$ un subconjunto de $G$. Las siguientes condiciones son equivalentes:
\begin{enumerate}
    \item $S$ es un subgrupo de $G$.
    \item $e \in S$ y para todo $a, b \in S$, se verifican $ab, a^{-1} \in S$.
    \item $S \neq \emptyset$ y para todo $a, b \in S$, se verifican $ab, a^{-1} \in S$.
    \item $e \in S$ y para todo $a, b \in S$, se verifica $ab^{-1} \in S$.
    \item $S \neq \emptyset$ y para todo $a, b \in S$, se verifica $ab^{-1} \in S$.
\end{enumerate}
\end{lemma}

\begin{proofbox}
$(1) \Rightarrow (2)$: Que $ab \in S$ es inmediato porque $S$ es un grupo, para ver que $a^{-1} \in S$ basta notar que los inversos son únicos: como $a$ debe de tener un inverso en $S$, este debe coincidir con su inverso en $G$. Sea $a \in S$, entonces $a^{-1} \in S$, luego $e = aa^{-1} \in S$.

$(2) \Rightarrow (3)$: Inmediato.

$(3) \Rightarrow (4)$: Dado $a \in S$ (existe por hipótesis) tenemos que $a^{-1} \in S$, luego $e = aa^{-1} \in S$. Además, $b^{-1} \in S$, luego $ab^{-1} \in S$.

$(4) \Rightarrow (5)$: inmediato.

$(5) \Rightarrow (1)$: Dado $a \in S$ (existe por hipótesis) tenemos que $aa^{-1} \in S$, luego $e = aa^{-1} \in S$. Este elemento es el neutro de $S$ puesto que el neutro es único. Sea $a \in S$, entonces $ea^{-1} \in S$, luego $a^{-1} \in S$, y este ha de ser el inverso de $a$ puesto que el inverso es único en $G$.
\end{proofbox}

\subsection{Ejemplos de subgrupos}
\label{ejemplos_subgrupos}

Para probar que un conjunto es un subgrupo podemos emplear cualquiera de las caracterizaciones del Lema \ref{lem:caracterizacion_subgrupos}, usualmente emplearemos (5) por ser la más sencilla.

\begin{example}{}{}
Si $G$ es un grupo, entonces $\{1\}$ y $G$ son subgrupos de $G$. El primero se llama subgrupo trivial, denotado $1$ y el segundo subgrupo impropio de $G$. Los subgrupos de $G$ diferentes de $G$ se dice que son subgrupos propios.
\end{example}

\begin{example}{}{}
Si $(A, +)$ es el grupo aditivo de un anillo, entonces todo subanillo y todo ideal de $A$ son subgrupos de este grupo.
\end{example}
    
\begin{example}{Subgrupos de \(\Z\)}{}
Si $S$ es un subgrupo de $(\mathbb{Z}, +)$, entonces dado $x \in S$ arbitrario, $x + x \in S$. Por inducción se deduce que $nx \in S$ para todo $n \in \mathbb{Z}$ y todo $x \in S$. Eso implica que $S$ es un ideal de $\mathbb{Z}$ y por tanto los subgrupos de $(\mathbb{Z}, +)$ son los de la forma $n\mathbb{Z}$ para $n$ un entero no negativo.
\end{example}

\begin{example}{}{}
Sea $\mathrm{GL}_n(K)$ el grupo de las matrices invertibles de tamaño $n$ con entradas en el cuerpo $K$. Entonces el conjunto $\mathrm{SL}_n(K)$ formado por las matrices de determinante $1$ es un subgrupo de $\mathrm{GL}_n(K)$.

\begin{proofbox}
Claramente $\mathrm{SL}_n(K)$ es no vacío. Sean $A, B \in \mathrm{SL}_n(K)$, entonces $\det(A) = \det(B) = 1$, como además $B$ es invertible sabemos que
\[
1 = \det(I) = \det(BB^{-1}) = \det(B)\det(B^{-1}) = \det(B^{-1}),
\]
luego 
\[
\det(AB^{-1}) = \det(A)\det(B^{-1}) = 1 \implies AB^{-1} \in \mathrm{SL}_n(K)
\]
como queríamos ver.
\end{proofbox}

\end{example}

\begin{example}{Automorfismos}{}
Supongamos que $A$ es un anillo y sea $S_A$ el grupo de las permutaciones de $A$. Entonces el conjunto $\mathrm{Aut}(A)$ formado por los automorfismos de $A$ es un subgrupo de $S_A$.
\begin{proofbox}
$\mathrm{Aut}(A) \neq \emptyset$, dados $f,g \in \mathrm{Aut}(A)$, $g^{-1}$ también es un automorfismo y, como la composición de homomorfismos de anillos es homomorfismo, $fg^{-1} \in \mathrm{Aut}(A)$ al ser un homomorfismo de $A$ en $A$.
\end{proofbox}

Ejemplos similares se pueden obtener con casi todas las estructuras matemáticas. Por ejemplo, si $G$ es un grupo, entonces decimos que $f : G \to G$ es un automorfismo si $f$ es biyectivo y $f(gh) = f(g)f(h)$ para todo $g, h \in G$. Entonces el conjunto $\mathrm{Aut}(G)$ formado por todos los automorfismos de $G$ es un subgrupo del grupo simétrico $S_G$ de $G$.
\end{example}

\begin{example}{}{}
Si $X$ es un espacio topológico entonces el conjunto de todos los homeomorfismos de $X$ de $X$ en si mismo es un subgrupo de $S_X$. Recuérdese que un homeomorfismo entre dos espacios topológicos es una aplicación biyectiva tal que tanto ella como su inversa son continuas. 

Si $X$ es un espacio métrico con distancia $d$, entonces el conjunto de las isometrías es un subgrupo de $S_X$. Recuérdese que una isometría entre dos espacios métricos es una biyección $f$ de uno al otro verifica $d(f(x),f(y))=d(x,y)$ para todos los elementos $x,y$ del dominio de $f$.
\end{example}

\begin{example}{Subgrupo cíclico}{}
Si $G$ es un grupo y $g \in G$, entonces 
\[
\langle g\rangle = \{g^n : n \in \mathbb{Z}\}
\]
es un subgrupo de $G$, llamado grupo cíclico generado por $g$.
\begin{proofbox}
Notemos que $e = g^0 \in \langle g\rangle$. Si $a, b \in \langle g \rangle$ entonces $a = g^{n_0}, b = g^{n_1}$ y es obvio que
\[
bg^{-n_1} = g^0 = e \implies b^{-1} = g^{-n_1}
\]
luego
\[
ab^{-1} = g^{n_0}g^{-n_1} = g^{n_0 - n_1} \in \langle g\rangle. 
\]
\end{proofbox}

Un grupo $G$ se dice que es cíclico si tiene un elemento $g$ tal que $G = \langle g\rangle$. En tal caso se dice que $g$ es un generador de $G$. Por ejemplo, $(\mathbb{Z},+)$ es cíclico generado por $1$ y $(\mathbb{Z}_n,+)$ es otro grupo cíclico generado por la clase de $1$. Otros ejemplos de grupos cíclicos son los grupos $C_n$ y $C_\infty$.
\end{example}

\begin{example}{}{}
Si $X$ es un subconjunto arbitrario de $G$, entonces el conjunto formado por todos los elementos de $G$ de la forma $x_1^{n_1}x_2^{n_2}\ldots x_m^{n_m}$, con $x_1,\ldots x_m \in X$ y $n_1,\ldots,n_m \in \mathbb{Z}$, es un subgrupo de $G$, que resulta ser el menor subgrupo de $G$ que contiene a $X$ y por tanto se llama subgrupo generado por $X$ y se denota $\langle X\rangle$.
\begin{proofbox}
Sea $x \in X$, entonces $e = x^0 \in \langle X\rangle$. Sean $a,b \in \langle X\rangle$ de la forma
\[
a = x_1^{n_1}x_2^{n_2}\ldots x_m^{n_m}, b = y_1^{l_1}y_2^{l_2}\ldots y_k^{l_k}
\]
es inmediato que
\[
b(y_k^{-l_k} \ldots y_1^{-l_1}) = e \implies b^{-1} = y_k^{-l_k} \ldots y_1^{-l_1}
\]
y por tanto
\[
ab^{-1} = x_1^{n_1}x_2^{n_2} \ldots x_m^{n_m} y_k^{-l_k} \ldots y_1^{-l_1} = z_1^{j_1} \ldots z_{m+k}^{j_k}
\]
donde $z_i = x_i, j_i = n_i$ si $1 \leq i \leq m$, $z_i = y_{k + m + 1 - i}, j_i = l_{k + m + 1 - i}= $ si $m < i \leq m + k$. Luego $ab^{-1} \in \langle X \rangle$.
\end{proofbox}

\end{example}

\begin{example}{}{}
El subgrupo generado por $X$ se puede construir de otra forma. Es un sencillo ejercicio comprobar que la intersección de subgrupos de $G$, es un subgrupo.

\begin{proofbox}
Sean $G_i \subseteq G$ subgrupos de $G$ y sea $H = \cap_{i \in I} G_i$. Entonces $e \in H$ ya que está en cada $G_i$. Si $a,b \in H$ entonces para todo $i \in I$
\[
a, b \in G_i \implies ab^{-1} \in G_i \implies ab^{-1} \in H.
\]
\end{proofbox}

Por tanto la intersección de todos los subgrupos de $G$ que contienen a $X$ es un subgrupo de $G$ y es el menor subgrupo de $G$ que contiene a $X$, con lo que es el subgrupo generado por $X$.
\end{example}

\begin{example}{Suma directa de grupos}{}
Si $(G_i)_{i \in I}$ es una familia arbitraria de grupos, entonces el subconjunto $\oplus_{i \in I} G_i$ formado por los elementos $(g_i) \in \prod_{i \in I} G_i$ tales que $g_i = 1$ para casi todo $i$, es un subgrupo de $\prod_{i \in I} G_i$.

\begin{proofbox}
El elemento neutro de $\prod_{i \in I} G_i$ está, obviamente, en $\oplus_{i \in I} G_i$. Si $a, b \in \oplus_{i \in I} G_i$ entonces el inverso de $b = (b_i)$ es el elemento $b^{-1} = (c_i) = (b_i^{-1})$, notemos que como $1^{-1} = 1$ casi todo $c_i = 1$, luego $b^{-1} \in \oplus_{i \in I} G_i$ y de hecho 
\[
ab^{-1} = (d_i)
\]
donde todos los $d_i = 1$ salvo un número finito (como mucho, hay tantos distintos de 1 como la suma de los distintos de 1 de $a$ y $b$). En resumen, $ab^{-1} \in \oplus_{i \in I} G_i$, luego es un subgrupo.
\end{proofbox}

\end{example}

\begin{example}{Centro y centralizador}{}
Si $G$ es un grupo arbitrario, entonces 
\[
Z(G) = \{g \in G : gx = xg, \text{ para todo } x \in G\}
\]
es un subgrupo abeliano de $G$, llamado centro de $G$. Más generalmente, si $x \in G$, entonces 
\[
C_G(x) = \{g \in G : gx = xg\}
\]
es un subgrupo de $G$, llamado centralizador de $x$ en $G$. Observese que $Z(G)$ es la intersección de todos los centralizadores de los elementos de $G$ en $G$.
\end{example}

\begin{proofbox}
Dado $x \in G$ veamos que $C_G(x)$ es un subgrupo, a partir de esto es fácil ver que $Z(G)$ es un subgrupo abeliano. La identidad está en $C_G(x)$ ya que
\[
ex = xe = x,
\]
dados $a,b \in C_G(x)$ es fácil ver que
\[
b^{-1}x = b^{-1}xbb^{-1} = b^{-1}bxb^{-1} = xb^{-1} \implies b^{-1} \in C_G(x)
\]
y, además,
\[
abx = axb = xab \implies ab \in C_G(x).
\]
\end{proofbox}

\subsection{Clases laterales}

Sea $G$ un grupo y $H$ un subgrupo de $G$. Se define la siguiente relación binaria en $G$:
\[
a \equiv_i b \mod H \quad \Leftrightarrow \quad a^{-1}b \in H. \qquad (a,b \in G).
\]
Se puede comprobar fácilmente que esta relación es de equivalencia y por tanto define una partición de $G$ en clases de equivalencia. La clase de equivalencia que contiene a $a$ es
\[
aH = \{ah : h \in H\}
\]
y se llama clase lateral de $a$ módulo $H$ por la izquierda.

Análogamente se puede definir otra relación de equivalencia:
\[
a \equiv_d b \mod H \quad \Leftrightarrow \quad ab^{-1} \in H. \quad (a,b \in G)
\]
para la que las clase de equivalencia que contiene a $a$ es
\[
Ha = \{ha : h \in H\}
\]
y se llama clase lateral de $a$ módulo $H$ por la derecha.

El conjunto de las clases laterales por la izquierda de $G$ módulo $H$ se denota por $G/H$ y el de clases laterales por la derecha $H \backslash G$.

Como consecuencia del Lema \ref{lem:prop_grupos} las aplicaciones
\[
\begin{array}{ccc}
H \to aH & \quad & H \to Ha \\
h \mapsto ah & & h \mapsto ha
\end{array}
\]
son biyectivas, con lo que todas las clases laterales tienen el mismo cardinal. Además la aplicación
\[
\begin{array}{ccc}
G/H & \to & H \backslash G \\
aH & \mapsto & Ha^{-1}
\end{array}
\]
es otra biyección con lo que también $G/H$ y $H \backslash G$ tienen el mismo cardinal.

\begin{proofbox}
Veamos que $L_a(h) = ah$ es una biyección. Si $ah = bh \implies a=b$ por la propiedad cancelativa, luego es inyectiva. Si $ax \in aH$ entonces $L_a(x) = ax$, luego es sobreyectiva.

En cuanto a $\phi(aH) = Ha^{-1}$ notemos que está bien definida y es inyectiva
\[
\phi(aH) = \phi(bH) \iff Ha^{-1} = Hb^{-1} \iff a^{-1}b \in H \iff aH = bH.
\]
También es sobreyectiva, puesto que dado $Ha^{-1} \in H \backslash G$, $\phi(aH) = Ha^{-1}$.

\end{proofbox}

Denotamos con $|X|$ el cardinal de un conjunto cualquiera. En el caso en que $G$ sea un grupo el cardinal de $G$ se suele llamar orden de $G$. Acabamos de ver que para cada subgrupo $H$ de $G$ se verifica:
\[
|aH| = |Ha| = |H| \quad \text{y} \quad |G/H| = |H \backslash G|
\]

El cardinal de $G/H$ (y $H \backslash G$) se llama índice de $H$ en $G$ y se denota $[G : H]$. Una consecuencia inmediata de estas fórmulas es el siguiente Teorema.

\begin{theorem}{Teorema de Lagrange}{lagrange}
Si $G$ es un grupo finito y $H$ es un subgrupo de $G$ entonces $|G| = |H||G : H|$.
\end{theorem}

\begin{corollary}{}{corolario_lagrange}
Si $G$ es un grupo finito de orden primo entonces los únicos subgrupos de $G$ son $1$ y $G$. En particular $G$ es cíclico y cualquier elemento de $G$ distinto de $1$ es un generador de $G$.
\end{corollary}