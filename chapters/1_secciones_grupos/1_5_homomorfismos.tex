\section{Homomorfismos de grupo y Teoremas de Isomorfía}

\begin{definition}{Homomorfismo de grupos}{homomorfismo_grupos}
Un homomorfismo del grupo $G$ en el grupo $H$ es una aplicación $f:G\to H$ que conserva la operación; es decir, que verifica
\[
f(ab)=f(a)f(b)
\]
para cualesquiera $a,b\in G$. Si $G=H$ decimos que $f$ es un endomorfismo de $G$.

Si $f:G\to H$ es un homomorfismo biyectivo, diremos que es un isomorfismo y que los grupos $G$ y $H$ son isomorfos. Un automorfismo de $G$ es isomorfismo de $G$ en $G$.
\end{definition}

\begin{definition}{Núcleo}{nucleo_homomorfismo}
El núcleo de un homomorfismo $f:G\to H$ es
\[
\ker f=f^{-1}(1_{H})=\{x\in G:f(x)=1_{H}\}.
\]
\end{definition}

\begin{lemma}{Propiedades de homomorfismos}{propiedades_homomorfismos}
Si $f:G\to H$ es un homomorfismo de grupos entonces se verifican las siguientes propiedades para $a,a_{1},\ldots,a_{n}\in G$:

\begin{enumerate}
    \item ($f$ conserva el neutro) $f(1_{G})=1_{H}$.
    \item ($f$ conserva inversos) $f(a^{-1})=f(a)^{-1}$.
    \item ($f$ conserva productos finitos) $f(a_{1}\cdots a_{n})=f(a_{1}) \cdots f(a_{n})$.
    \item ($f$ conserva potencias) Si $n\in\mathbb{Z}$ entonces $f(a^{n})=f(a)^{n}$.
    \item Si $f$ es un isomorfismo entonces la aplicación inversa $f^{-1}:H\to G$ también lo es.
    \item Si $g:H\to K$ es otro homomorfismo de grupos entonces $g\circ f:G\to K$ es un homomorfismo de grupos.
    \item Si $H_{1}$ es un subgrupo de $H$ entonces $f^{-1}(H_{1})=\{x\in G:f(x)\in H_{1}\}$ es un subgrupo de $G$. Si además $H_{1}$ es normal en $H$ entonces $f^{-1}(H_{1})$ es normal en $G$; en particular, $\ker f$ es un subgrupo normal de $G$.
    \item $f$ es inyectivo si y solo si $\ker f=\{1\}$.
    \item Si $G_{1}$ es un subgrupo de $G$ entonces $f(G_{1})$ es un subgrupo de $H$; en particular, $\Im f$ es un subgrupo de $H$. Si además $G_{1}$ es normal en $G$ y $f$ es suprayectiva entonces $f(G_{1})$ es normal en $H$.
\end{enumerate}
\end{lemma}

\begin{proofbox}
\begin{enumerate}
    \item $f(1_G) = f(1_G 1_G) = f(1_G)f(1_G) \implies f(1_G) = 1_H$.
    \item $f(a^{-1})f(a) = f(aa^{-1}) = f(1_G) = 1_H \implies f(a^{-1})=f(a)^{-1}$.
    \item Por inducción, el caso base es cierto por hipótesis, si lo suponemos cierto para $n$ entonces
    \[
    f(a_{1}\cdots a_{n+1})=f(a_{1}\cdots a_{n})f(a_{n+1}) = f(a_{1}) \cdots f(a_{n}) f(a_{n+1}).
    \]
    \item De nuevo es fácil razonar por inducción, queda como ejercicio. Para el caso base $f(a^2)=f(aa)=f(a)f(a)=f(a)^2$.
    \item Si $f$ es un isomorfismo a inversa es biyectiva, por lo que basta ver que verifica
    \[
    f^{-1}(ab) = f^{-1}(a)f^{-1}(b).
    \]
    Para ello notemos que
    \[
    f(f^{-1}(ab))= ab = f(f^{-1}(a))f(f^{-1}(b)) = f(f^{-1}(a)f^{-1}(b)).
    \]
    y como $f$ es inyectiva debe ser 
    \[
    f^{-1}(ab) = f^{-1}(a)f^{-1}(b).
    \]
    \item Basta notar que
    \[
    g \circ f(ab) = g(f(ab))=g(f(a)f(b))=g(f(a))g(f(b)) = [g \circ f(a)] [g \circ f(b)].
    \]
    \item Claramente, $1_G \in f^{-1}(H_{1}) \neq \emptyset$, y si $a,b \in f^{-1}(H_{1})$, entonces existen $x,y \in H_1$ tales que
    \[
    x = f(a), y = f(b) \implies f(ab^{-1}) = xy^{-1} \in H_1 \implies ab^{-1} \in f^{-1}(H_1).
    \]
    Si además $H_{1}$ es normal en $H$ entonces poniendo $N = f^{-1}(H_{1})$ y dado $g \in G$
    \begin{align*}
    x &\in gNg^{-1} \implies x = gng^{-1}, f(n) \in H_1 \\
    &\implies f(x) = f(g)f(n)f(g)^{-1} \in f(g)H_1f(g)^{-1} = H_1 \implies x \in N,
    \end{align*}
    luego $gNg^{-1} \subseteq N$, por lo que por la Proposición \ref{prop:caracterizacion_normales} $N$ es normal.
    \item Si $f$ es inyectivo entonces
    \[
    f(x) = 1 = f(1) \implies x = 1.
    \]
    Por otro lado, si $\ker f = \{1\}$, entonces
    \[
    f(x) = f(y) \implies f(xy^{-1}) = 1 \implies xy^{-1} = 1 \implies x = y.
    \]
    \item Claramente $1 = f(1) \in f(G_1) \neq \emptyset$. Además, si $a,b \in f(G_1)$,entonces existen $x, y \in G_1$ tales que 
    \[
    a = f(x), b = f(y) \implies ab^{-1} = f(xy^{-1}) \in f(G_1).
    \]
    
    Si $G_1 \unlhd G$ y $f$ es sobreyectivo, llamando $N = f(G_1)$ tenemos para todo $h \in H$
    \[
    x \in hNh^{-1} \implies x = hnh^{-1} = f(a)f(b)f(a)^{-1} =f(aba^{-1})
    \]
    donde $h = f(a)$ por ser $f$ sobreyectivo, $n=f(b)$ con $b \in G_1$ por ser $n \in N$.
    
    Como $aba^{-1} \in aG_1a^{-1} = G_1$ entonces
    \[
    x = f(aba^{-1}) \in N
    \]
    como queríamos ver.
\end{enumerate}
\end{proofbox}


\subsection{Ejemplos de homomorfismos}

\begin{example}{}{}
Si $H$ es un subgrupo de $G$, la inclusión $i$ de $H$ en $G$ es un homomorfismo inyectivo.
\end{example}
\begin{proofbox}
Si $a,b \in H$, entonces
\[
i(ab) = ab = i(a)i(b),
\]
luego es un homomorfismo. Además, $i(a) = i(b) \implies a = b$ en $G$, pero entonces $a = b$ en $H$, lo que prueba la inyectividad.
\end{proofbox}

\begin{example}{}{}
Si $N$ es un subgrupo normal de $G$, la aplicación $\pi:G\to G/N$ dada por $\pi(x)=xN$ es un homomorfismo suprayectivo, que recibe el nombre de proyección canónica de $G$ sobre $G/N$. Su núcleo es $N$.
\end{example}
\begin{proofbox}
Si $a,b \in G$, entonces
\[
\pi(ab) = abN = aNbN = \pi(a)\pi(b),
\]
luego es un homomorfismo. Además, dado $xN \in G/N$, $\pi(x) = xN$.
\end{proofbox}

\begin{example}{}{}
Dados dos grupos $G$ y $H$, la aplicación $f:G\to H$ dada por $f(a)=1_{H}$ para cada $a\in G$ es un homomorfismo llamado homomorfismo trivial de $G$ en $H$. Su núcleo es todo $G$.
\end{example}
\begin{proofbox}
Si $a,b \in G$, entonces
\[
f(ab) = 1_H  = 1_H 1_H = f(a)f(b),
\]
luego es un homomorfismo.
\end{proofbox}

\begin{example}{}{}
La aplicación $f:\mathbb{Z}\to\mathbb{Z}$ dada por $f(n)=2n$ es un endomorfismo del grupo aditivo de $\mathbb{Z}$ que es inyectivo y no suprayectivo. (Nótese que $f$ no es un endomorfismo de anillos.)
\end{example}
\begin{proofbox}
En este caso usaremos notación aditiva. Si $a,b \in \Z$, entonces
\[
f(a+b) = 2(a+b) = 2a + 2b = f(a) + f(b)
\]
luego es un homomorfismo. Para ver que es inyectivo
\[
f(a) = f(b) \implies 2a = 2b \implies 2(a-b) = 0 \implies a = b
\]
usando que $\Z$ es un dominio. Que no es sobreyectivo es fácil de ver, ya que $3 \in \Z$ pero $f(a) = 3 \implies 2a = 3$ lo cual es imposible al ser $3$ impar.
\end{proofbox}

\begin{example}{}{}
Si $G$ es cualquier grupo y $x\in G$ entonces la aplicación $\mathbb{Z}\to G$ dada por $n\mapsto x^{n}$ es un homomorfismo de grupos; como en $\mathbb{Z}$ usamos notación aditiva y en $G$ multiplicativa, la afirmación anterior es equivalente al hecho, que ya conocemos, de que $x^{n+m}=x^{n}x^{m}$.
\end{example}

\begin{example}{}{}
Otro ejemplo en el que se mezclan las notaciones aditiva y multiplicativa es el siguiente: Fijado un número real positivo $\alpha$, la aplicación 
\[
(\mathbb{R},+) \rightarrow (\mathbb{R}^{+}=\{x\in\mathbb{R}:x>0\},\cdot), \quad r \mapsto \alpha^{r}
\]
es un isomorfismo de grupos cuya inversa es la aplicación $\mathbb{R}^{+}\to\mathbb{R}$ dada por $s\mapsto\log_{\alpha}s$.
\end{example}

Claramente, si $f:G\to H$ es un homomorfismo inyectivo de grupos entonces $f:G\to\Im f$ es un isomorfismo de grupos que nos permite ver a $G$ como un subgrupo de $H$. Este resultado, y algunas consecuencias, vienen recogidas en el siguiente teorema.

\begin{theorem}{Teoremas de Isomorfía para grupos}{teoremas_isomorfia}
\begin{enumerate}
    \item Si $f:G\to H$ es un homomorfismo de grupos entonces existe un único isomorfismo de grupos $\bar{f}:G/\ker f\to\Im f$ que hace conmutativo el diagrama
    \[
    \begin{tikzcd}
    G \arrow[r, "f"] \arrow[d, "p"] & H \\
    G/\ker f \arrow[r, "\bar{f}"] & \Im f \arrow[u, "i"]
    \end{tikzcd}
    \]
    es decir, $i\circ\bar{f}\circ p=f$, donde $i$ es la inclusión y $p$ es la proyección canónica. En particular
    \[
    \frac{G}{\ker f}\cong\Im f.
    \]
    
    \item Sean $N$ y $H$ subgrupos normales de un grupo $G$ con $N\subseteq H$. Entonces $H/N$ es un subgrupo normal de $G/N$ y se tiene
    \[
    \frac{G/N}{H/N}\cong G/H.
    \]
    
    \item Sean $G$ un grupo, $H$ un subgrupo de $G$ y $N$ un subgrupo normal de $G$. Entonces $NH$ es un subgrupo de $G$ que contiene a $H$, $N\cap H$ es un subgrupo normal de $H$ y se tiene
    \[
    \frac{H}{N\cap H}\cong\frac{NH}{N}.
    \]
\end{enumerate}
\end{theorem}

\begin{remark}
En general no es verdad que si $H$ y $K$ son subgrupos de $G$ entonces $HK$ es un subgrupo de $G$. Por ejemplo, consideremos el grupo $S_{3}$ de las permutaciones de tres elementos y sean $\sigma$ y $\tau$ las permutaciones dadas por $\sigma(1)=2$, $\sigma(2)=1,\sigma(3)=3$, $\tau(1)=3$, $\tau(2)=2$ y $\tau(3)=1$. Entonces $\langle\sigma\rangle=\{1,\sigma\}$ y $\langle\tau\rangle=\{1,\tau\}$. Luego $\langle\sigma\rangle\langle\tau\rangle=\{1,\sigma,\tau,\sigma\tau\}$ y por tanto $|\langle\sigma\rangle\langle\tau\rangle|=4$ que no divide a $|S_{3}|=6$. Del \hyperref[thm:lagrange]{Teorema de Lagrange} deducimos que $\langle\sigma\rangle\langle\tau\rangle$ no es subgrupo de $S_{3}$.
\end{remark}

Usando el \hyperref[thm:teorema_correspondencia_grupos]{Teorema de la Correspondencia} se obtiene el siguiente corolario.

\begin{corollary}{}{corolario_correspondencia_homomorfismos}
Si $f:G\to H$ es un homomorfismo de grupos entonces $K\mapsto f(K)$ define una biyección entre el conjunto de los subgrupos de $G$ que contienen a $\ker f$ y el de los subgrupos de $\Im f$.
\end{corollary}

\begin{example}{Aplicaciones de los Teoremas de Isomorfía}{aplicaciones_isomorfia}
\begin{enumerate}
    \item Consideremos los grupos multiplicativos $\mathbb{C}^{*}$ y $\mathbb{R}^{*}$, y la aplicación norma $\delta:\mathbb{C}^{*}\to\mathbb{R}^{*}$ dada por $\delta(a+bi)=a^{2}+b^{2}$. Entonces $\delta$ es un homomorfismo que tiene por núcleo a la circunferencia de radio $1$ en $\mathbb{C}$, y por imagen a $\mathbb{R}^{+}$. Por tanto, el grupo cociente de $\mathbb{C}^{*}$ por la circunferencia de radio $1$ es isomorfo a $\mathbb{R}^{+}$.
    
    \item La aplicación $\det:\mathrm{GL}_{n}(\mathbb{R})\to\mathbb{R}^{*}$ que lleva una matriz a su determinante es un homomorfismo supravectivo de grupos con núcleo $\mathrm{SL}_{n}(\mathbb{R})$. Esto nos dice que el cociente de $\mathrm{GL}_{n}(\mathbb{R})$ por $\mathrm{SL}_{n}(\mathbb{R})$ es isomorfo a $\mathbb{R}^{*}$.

    Esto es totalmente esperable, pues dada una matriz invertible genérica $M \in GL_n(\R)$ podemos obtener una matriz "equivalente" en $SL_n(\R)$ sin más que hacer
    \[
    M' = \lambda M,\quad \lambda = (\det(M))^{-1/n}
    \]
    puesto que $\det(M') = \lambda^n \det(M) = \det(M)^{-1} \det(M) = 1$.
\end{enumerate}
\end{example}