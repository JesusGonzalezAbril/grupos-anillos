\clearpage
\section{Conjugación y acciones de grupos en conjuntos}

\begin{definition}{Conjugación}{conjugacion}
Sea $G$ un grupo. Si $a, g \in G$, entonces se define el conjugado de $g$ por $a$ como
\[
g^a = a^{-1}ga.
\]

Más generalmente, si $X$ es un subconjunto de $G$, entonces el conjugado de $X$ por $a$ es
\[
X^a = \{a^{-1}xa : x \in X\}.
\]

Se dice que dos elementos o subconjuntos $x$ y $y$ de $G$ son conjugados en $G$ si $x^a = y$ para algún $a \in G$.
\end{definition}

\begin{remark}
Notemos que un subgrupo de $G$ es normal si y solo si todos sus conjugados en $G$ son iguales.
\end{remark}

Si $a \in G$, entonces la aplicación
\[
\iota_a : G \to G,\quad \iota_a(x) = x^a = a^{-1}xa
\]
es un automorfismo de $G$, llamado automorfismo interno definido por $a$, con inverso $\iota_{a^{-1}}$. Eso implica que dos elementos o subconjuntos conjugados de un grupos tienen propiedades similares. Por ejemplo, dos elementos conjugados de $G$ tiene el mismo orden y el conjugado de un subgrupo de $G$ es otro subgrupo de $G$ del mismo orden.

Es fácil ver que
\[
g^{ab} = (ab)^{-1}gab = b^{-1}(a^{-1}ga)b = (g^a)^b
\]
para todo $g, a, b \in G$, y utilizando esto se demuestra de forma fácil que la relación ser conjugados (tanto de elementos, como de subconjuntos de $G$) es una relación de equivalencia. Las clases de equivalencia de esta relación de equivalencia en $G$ se llaman clases de conjugación de $G$. La clase de conjugación de $G$ que contiene a $a$ se denota por $a^G$. Es decir
\[
a^G = \{a^g : g \in G\}.
\]

\begin{definition}{Acción de grupo}{accion_grupo}
Sean $G$ un grupo y $X$ un conjunto. Una acción por la izquierda de $G$ en $X$ es una aplicación
\[
\begin{array}{r c c c}
    \cdot : & G \times X & \to & X\\
            & (g, x)     & \mapsto     & g \cdot x
\end{array}
\]
que satisface las siguientes propiedades:
\begin{enumerate}
    \item $(gh) \cdot x = g \cdot (h \cdot x)$, para todo $x \in X$ y todo $g, h \in G$.
    \item $1 \cdot x = x$, para todo $x \in X$.
\end{enumerate}

Análogamente se define una acción por la derecha.
\end{definition}

\begin{definition}{Órbita y estabilizador}{orbita_estabilizador}
Sea $\cdot : G \times X \rightarrow X$ una acción por la izquierda de un grupo $G$ en un conjunto $X$. Si $x \in X$ entonces 
\[
G \cdot x = \{g \cdot x : g \in G\}
\]
se llama órbita de $x$ y 
\[
\operatorname{Estab}_{G}(x) = \{g \in G : g \cdot x = x\}
\]
se llama estabilizador de $x$ en $G$. Obsérvese que las órbitas forman una partición de $G$.
\end{definition}

\begin{remark}
En el caso de acciones por la derecha la órbita de $x$ se denota $x \cdot G$.
\end{remark}

Vamos a ver una definición alternativa para las acciones. Sea $\cdot : G \times X \to X$ una acción por la izquierda del grupo $G$ en el conjunto $X$. Consideremos la aplicación
\[
\phi : G \to S_X, \quad \phi(g)(x) = g \cdot x,
\]
notemos que, fijado $g$, $g \cdot x$ es una biyección:
\[
g \cdot x : X \to X
\]
ya que tiene aplicación inversa $g^{-1} \cdot x : X \to X$. Además, $\phi$ es un homomorfismo de grupos, puesto que
\[
\phi(gh)(x) = (gh) \cdot x = g \cdot (h \cdot x) = \phi(g)(\phi(h)(x)) = (\phi(g) \circ \phi(h))(x) 
\]
(recordemos que la operación en $S_X$ es la composición de aplicaciones).

Recíprocamente, si $\phi : G \to S_X$ es un homomorfismo de grupos, entonces la aplicación
\[
\cdot : G \times X \to X, \quad g \cdot x = \phi(g)(x)
\]
es una acción por la izquierda de $G$ en $X$ ya que
\[
(gh) \cdot x = \phi(gh)(x) = (\phi(g) \circ \phi(h))(x) = \phi(g)(\phi(h)(x)) = g \cdot (h \cdot x)
\]
\[
1 \cdot x = \phi(1)(x) = Id_X(x)
\]
donde hemos usado que $\phi$ es un homomorfismo y, por tanto, $\phi(1) = Id_{X}$.

Por tanto, es lo mismo hablar de una acción por la izquierda de un grupo $G$ en un conjunto $X$ que de un homomorfismo de grupos $G \rightarrow S_{X}$. Tomémonos un segundo para reflexionar sobre este resultado, hemos determinado que, dado un grupo cualquiera $G$ y un conjunto cualquiera $X$, podemos ver los elementos de $G$ como biyecciones en $X$ (aplicaciones que "mantienen" el conjunto subyacente invariante), es decir, todo elemento de un grupo se puede asociar con una cierta transformación sobre algún conjunto.

Análogamente podemos identificar las acciones por la derecha de $G$ en $X$ con los antihomomorfismos de grupos $\phi : G \rightarrow S_{X}$, es decir las aplicaciones $\phi : G \rightarrow S_{X}$ que satisfacen
\[
\phi(gh) = \phi(h) \circ \phi(g),
\]
para todo $g,h \in G$.

En realidad podemos ver las acciones por la izquierda y por la derecha como los mismos objetos matemáticos pues si $\cdot : G \times X \rightarrow X$ es una aplicación y definimos $\ast : X \times G \rightarrow G$ poniendo
\[
x \ast g = g^{-1} \cdot x
\]
entonces $\cdot$ es una acción por la izquierda de $G$ en $X$ si y solo si $\ast$ es una acción por la derecha de $G$ en $X$.

Esto mismo lo podríamos haber visto observando que si $\phi : G \rightarrow S_{X}$ es una aplicación y definimos $\psi : G \rightarrow S_{X}$ poniendo $\psi(g) = \phi(g^{-1})$, entonces $\phi$ es un homomorfismo si y solo si $\psi$ es un antihomomorfismo.

\subsection{Ejemplos de acciones de grupos}

Sea $G$ un grupo arbitrario.

\begin{example}{}{}
Consideremos la acción por la izquierda de $G$ en sí mismo dada por $g \cdot x = gx$. Esta acción se llama acción por traslación a la izquierda. En ocasiones escribimos
\[
L_g(x) = g\cdot x = gx.
\]

Análogamente se define una acción por traslación a la derecha $R_g(x) = x \ast g = xg$. Obsérvese que $\operatorname{Estab}_{G}(x) = \left\{ e \right\}$ y, para todo $x \in G$, $G \cdot x = G$.

Más generalmente, si $H$ es un subgrupo de $G$, entonces $G$ actúa por la izquierda en $G/H$ mediante la regla: $g \cdot xH = (gx)H$. Análogamente se define una acción por la derecha de $G$ en $H \backslash G$. De nuevo, todos los elementos están en la misma órbita y    
\begin{align*}
\operatorname{Estab}_{G}(xH) &= \{g \in G : gxH = xH\} = \{g \in G : x^{-1}gx \in H\} \\
&= \{g \in G : g \in xHx^{-1}\} = xHx^{-1} = H^{x^{-1}}.
\end{align*}
\end{example}

\begin{example}{}{}
La acción por conjugación de $G$ en sí mismo es la acción por la derecha dada por $x \cdot g = x^{g} = g^{-1}xg$. Para ver que es una acción notemos que
\[
x^{e} = x,\quad x^{ab} = (x^{a})^{b}.
\]

La órbita $x \cdot G$ es $x^{G}$, la clase de conjugación de $x$ en $G$ y el estabilizador es $\operatorname{Estab}_{G}(x) = C_{G}(x)$, el centralizador de $x$ en $G$.
\end{example}

\begin{example}{}{}
$G$ actúa por la derecha en el conjunto $S$ de sus subgrupos mediante la regla $H \cdot g = H^{g}$. Esta acción se llama acción por conjugación de $G$ en sus subgrupos. El estabilizador de $H$ es
\[
\operatorname{Estab}_{G}(H) = \{g \in G : H^{g} = H\} = N_{G}(H),
\]
el normalizador de $H$ en $G$. Obsérvese que $N_{G}(H)$ es el mayor subgrupo de $G$ que contiene a $H$ como subgrupo normal.
\end{example}

\begin{example}{}{}
Para cada entero positivo $n$, consideramos $S_{n}$ (el grupo de permutaciones de orden $n$) actuando por la izquierda en $\{1, 2, \ldots, n\}$ mediante: $\sigma \cdot x = \sigma(x)$.

Claramente, todo elemento está en la misma órbita ya que fijado $x$ arbitrario
\[
S_n \cdot x = \{\sigma(x) : \sigma \in S_n\} 
\]
y para cualquier otro $y$ podemos considerar la permutación dada por
\[
\tau(x) = y,\ \tau(y) = x,\ \tau(z) = z\quad (z \neq x,y),
\]
luego $y = \tau(x) \in S_n \cdot x$.

Además,
\[
\operatorname{Estab}_{S_{n}}(i) = \{\sigma \in S_{n} : \sigma(i) = i\} \cong S_{n-1}
\]
ya que las permutaciones que dejan invariantes a $i$ pueden verse como permutaciones de $n-1$ elementos.
\end{example}

\begin{example}{}{}
Sea $n$ un número natural y $X$ un conjunto. Entonces la siguiente fórmula define una actuación por la izquierda del grupo simétrico $S_{n}$ en $X^{n} = \prod_{i = 1}^{n} X$:
\[
\sigma \cdot (x_{1}, \ldots, x_{n}) = (x_{\sigma(1)}, \ldots, x_{\sigma(n)}).
\]

Si $X = \R$, $n = 3$ y $\sigma \in S_3$ es la biyección
\[
\sigma(1) = 3,\ \sigma(2) = 1,\ \sigma(3) = 2
\]
entonces
\[
\sigma \cdot (x, y, z) = (y, z, x)
\]
\end{example}

\begin{example}{}{}
Sea $G$ un grupo actuando por la izquierda en un conjunto $X$ y sean $H \leq G$ un subgrupo de $G$ e $Y \subseteq X$ un subconjunto de $X$. Si para todo $h \in H$ y todo $y \in Y$ se verifica que $h \cdot y \in Y$ entonces la restricción a $H \times Y$ de la acción de $G$ en $X$ es una acción por la izquierda de $H$ en $Y$.
\end{example}

\begin{proofbox}
Supongamos que $\cdot: G \times X \to X$ es la acción de $G$ en $X$. Sea $\ast$ la restricción de $\cdot$ a $H \times Y$. Como $h \cdot y \in Y$, $\ast$ está bien definida. Que verifica las condiciones para ser una acción es inmediato.
\end{proofbox}

\subsection{Propiedades de acciones}

\begin{proposition}{Propiedades de acciones}{propiedades_acciones}
Sea $G$ un grupo actuando en un conjunto $X$ y sean $x \in X$ y $g \in G$. Entonces

\begin{enumerate}
    \item $\operatorname{Estab}_{G}(x)$ es un subgrupo de $G$.
    \item $[G : \operatorname{Estab}_{G}(x)] = |G \cdot x|$. En particular, si $G$ es finito, entonces el número de elementos de cada órbita es un divisor del orden de $G$.
    \item Si se trata de una acción por la izquierda entonces $\operatorname{Estab}_{G}(g \cdot x) = \operatorname{Estab}_{G}(x)g^{-1}$. Sin embargo, si se trata de una acción por la derecha entonces $\operatorname{Estab}(x \cdot g) = \operatorname{Estab}_{G}(x)g$.
    \item (Ecuación de Órbitas) Si $R$ es un conjunto de representantes de las órbitas de la acción de $G$ en $X$ (es decir, $R$ contiene exactamente un elemento de cada órbita) entonces
    \[
    |X| = \sum_{r \in R} |G \cdot r| = \sum_{r \in R} [G : \operatorname{Estab}_{G}(r)].
    \]
\end{enumerate}
\end{proposition}

\begin{proofbox}
\begin{enumerate}
    \item Para cualquier acción $e \cdot x = x$, luego $e \in \operatorname{Estab}_{G}(x)$. Si tomamos $a, b \in \operatorname{Estab}_{G}(x)$ entonces
    \[
    a \cdot x = x,\ b \cdot x = x \implies (ab) \cdot x = a \cdot (b \cdot x) = a \cdot x = x,
    \]
    luego $ab \in \operatorname{Estab}_{G}(x)$. Para ver que $a^{-1} \in \operatorname{Estab}_{G}(x)$ notemos que
    \[
    a^{-1} \cdot x = a^{-1} \cdot (a \cdot x) = (a^{-1}a) \cdot x = e \cdot x = x.
    \]

    \item Basta ver que existe una biyección que asigna a cada clase lateral $g\operatorname{Estab}_{G}(x)$, $g \in G$, un elemento de la órbita $G \cdot x$. Para ello, sea $N = \operatorname{Estab}_{G}(x)$
    \[
    \phi : G/N \to G \cdot x, \quad \phi(gN) = (g \cdot x),
    \]
    que está bien definida y es inyectiva ya que
    \[
    \phi(gN) = \phi(hN) \implies g \cdot x = h \cdot x \implies g^{-1}h \cdot x = g^{-1} \cdot (h \cdot x) = g^{-1} \cdot (g \cdot x) = e \cdot x = x,
    \]
    por tanto $g^{-1}h \in N$, luego $gN = hN$.
    Por otro lado, dado $\alpha \in G \cdot x$ entonces $\alpha = g \cdot x$ con $g \in G$, luego
    \[
    \phi(gN) = g \cdot x = \alpha
    \]
    y por tanto $\phi$ es sobreyectiva.

    \item Probaremos solo el caso por la izquierda. Sea $h \in \operatorname{Estab}_{G}(g \cdot x)$, entonces
    \[
    h \cdot (g \cdot x) = x \implies hg \cdot x = x
    \]
    por lo que $hg \in \operatorname{Estab}_{G}(x) \implies h = (hg)g^{-1} \in \operatorname{Estab}_{G}(x)g^{-1}$.

    Por otro lado, si $h \in \operatorname{Estab}_{G}(x)g^{-1}$, entonces
    $h = ag^{-1}$ con $a \in \operatorname{Estab}_{G}(x)$, por lo que
    \[
    h \cdot (g \cdot x) = ag^{-1} \cdot (g \cdot x) = a \cdot x = x
    \]
    con lo que concluimos que $h \in \operatorname{Estab}_{G}(g \cdot x)$.

    \item Que
    \[
    \sum_{r \in R} |G \cdot r| = \sum_{r \in R} [G : \operatorname{Estab}_{G}(r)].
    \]
    es inmediato por (2).

    Para ver
    \[
    |X| = \sum_{r \in R} |G \cdot r|.
    \]
    notemos que, como las órbitas forman una partición de $X$,
    \[
    X = \bigcup_{r \in R} \left\vert G \cdot r \right\vert
    \]
    siendo la unión disjunta, de lo que deducimos que $|X| = \sum_{r \in R} |G \cdot r|$.
\end{enumerate}
\end{proofbox}

Aplicando la Proposición \ref{prop:propiedades_acciones} a la acción de $G$ en sí mismo y en sus subgrupos por conjugación obtenemos el siguiente corolario.

\begin{corollary}{Propiedades de conjugación}{propiedades_conjugacion}
Sea $G$ un grupo y sean $a, g \in G$ y $H$ un subgrupo de $H$.

\begin{enumerate}
    \item $|a^G| = [G : C_G(a)]$. En particular, $a^G$ tiene un único elemento si y solo si $a$ es un elemento del centro $Z(G)$ de $G$.
    \item El cardinal del conjunto de conjugados de $H$ en $G$ es $[G : N_G(H)]$.
    \item $C_G(x^g) = C_G(x)^g$ y $N_G(H^g) = N_G(H)^g$.
    \item (Ecuación de Clases). Si $G$ es finito y $X$ es un subconjunto de $G$ que contiene exactamente un elemento de cada clase de conjugación con al menos dos elementos, entonces
    
    \[
    |G| = |Z(G)| + \sum_{x \in X} [G : C_G(x)].
    \]
\end{enumerate}
\end{corollary}

Si $p$ es un primo, entonces un $p$-grupo finito es un grupo finito de orden una potencia de $p$.

\begin{proposition}{Centro de $p$-grupos}{centro_p_grupos}
Si $G$ es un $p$-grupo finito no trivial para $p$ un primo entonces $Z(G) \neq 1$.
\end{proposition}

\begin{proofbox}
Utilizando la notación del Corolario \ref{cor:propiedades_conjugacion} tenemos $|G| = |Z(G)| + \sum_{x \in X} [G : C_G(x)]$. Entonces $|G|$ y $[G : C_G(x)]$ son potencias de $p$ para todo $x \in X$, con lo que $|Z(G)|$ es múltiplo de $p$ y por tanto $Z(G) \neq 1$.
\end{proofbox}

\begin{theorem}{Estructura de $p$-grupos}{estructura_p_grupos}
Si $G$ es un $p$-grupo finito entonces $G$ tiene una cadena de subgrupos normales
\[
1 = G_0 \subset G_1 \subset G_2 \subset \cdots \subset G_n = G
\]
tales que $[G_i : G_{i-1}] = p$ para todo $i = 1, \dots, n$.
\end{theorem}

\begin{theorem}{Teorema de Cauchy}{teorema_cauchy}
Si $G$ es un grupo finito cuyo orden es múltiplo de un primo $p$, entonces $G$ tiene un elemento de orden $p$.
\end{theorem}