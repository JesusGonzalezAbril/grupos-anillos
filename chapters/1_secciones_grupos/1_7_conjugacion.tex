\subsection{Conjugación y acciones de grupos en conjuntos}

\begin{definition}{Conjugación}{conjugacion}
Sea $G$ un grupo. Si $a, g \in G$, entonces se define el \emph{conjugado} de $g$ por $a$ como

\[
g^a = a^{-1}ga.
\]

Más generalmente, si $X$ es un subconjunto de $G$, entonces el conjugado de $X$ por $a$ es

\[
X^a = \{x^a : x \in X\}.
\]

Se dice que dos elementos o subconjuntos $x$ y $y$ de $G$ son \emph{conjugados} en $G$ si $x^a = y$ para algún $a \in G$. Por ejemplo, un subgrupo de $G$ es normal si y solo si todos sus conjugados en $G$ son iguales.
\end{definition}

Si $a \in G$, entonces la aplicación

\[
\begin{array}{ccc}
t_a : G & \rightarrow & G \\
x & \mapsto & t_a(x) = x^a
\end{array}
\]

es un automorfismo de $G$, llamado \emph{automorfismo interno} definido por $a$, con inverso $t_{a^{-1}}$. Eso implica que dos elementos o subconjuntos conjugados de un grupos tienen propiedades similares. Por ejemplo, dos elementos conjugados de $G$ tiene el mismo orden y el conjugado de un subgrupo de $G$ es otro subgrupo de $G$ del mismo orden.

Es fácil ver que

\[
g^{ab} = (g^a)^b
\]

para todo $g, a, b \in G$,

y utilizando esto se demuestra de forma fácil que la relación ser conjugados (tanto de elementos, como de subconjuntos de $G$) es una relación de equivalencia. Las clases de equivalencia de esta relación de equivalencia en $G$ se llaman \emph{clases de conjugación} de $G$. La clase de conjugación de $G$ que contiene a $a$ se denota por $a^G$. Es decir

\[
a^G = \{a^g : g \in G\}.
\]

\begin{definition}{Acción de grupo}{accion_grupo}
Sean $G$ un grupo y $X$ un conjunto. Una \emph{acción por la izquierda} de $G$ en $X$ es una aplicación

\[
\begin{array}{ccc}
\cdot : G \times X & \rightarrow & X \\
(g, x) & \mapsto & g \cdot x
\end{array}
\]

que satisface las siguientes propiedades:

\begin{enumerate}
    \item $(gh) \cdot x = g \cdot (h \cdot x)$, para todo $x \in X$ y todo $g, h \in G$.
    \item $1 \cdot x = x$, para todo $x \in X$.
\end{enumerate}

Análogamente se define una acción por la derecha.
\end{definition}

Vamos a ver una definición alternativa. Sea $\cdot : G \times X \rightarrow X$ una acción por la izquierda del grupo $G$ en el conjunto $X$. Entonces la aplicación $\phi : G \rightarrow S_X$ dada por $\phi(g)(x) = g \cdot x$ es un homomorfismo de grupos. Recíprocamente, si $\phi : G \rightarrow S_X$ es un homomorfismo de grupos, entonces la aplicación $\cdot : G \times X \rightarrow X$, dada por $g \cdot x = \phi(g)(x)$ es una acción por la izquierda de $G$ en $X$. Por tanto, es lo mismo hablar de una acción por la izquierda de un grupo $G$ en un conjunto $X$ que de un homomorfismo de grupos $G \rightarrow S_{X}$. 

Análogamente podemos identificar las acciones por la derecha de $G$ en $X$ con los antihomomorfismos de grupos $\phi : G \rightarrow S_{X}$, es decir las aplicaciones $\phi : G \rightarrow S_{X}$ que satisfacen $\phi(gh) = \phi(h)\phi(g)$, para todo $g,h \in G$. En realidad podemos ver las acciones por la izquierda y por la derecha como los mismos objetos matemáticos pues si $: G \times X \rightarrow X$ es una aplicación y definimos $* : X \times G \rightarrow G$ poniendo $x * g = g^{-1} \cdot x$ entonces - es una acción por la izquierda de $G$ en $X$ si y solo si $*$ es una acción por la derecha de $G$ en $X$. Esto mismo lo podríamos haber visto observando que si $\phi : G \rightarrow S_{X}$ es una aplicación y definimos $\psi : G \rightarrow S_{X}$ poniendo $\psi(g) = \phi(g^{-1})$, entonces $\phi$ es un homomorfismo si y solo si $\psi$ es un antihomomorfismo.

Sea $: G \times X \rightarrow X$ una acción por la izquierda de un grupo $G$ en un conjunto $X$. Si $x \in X$ entonces $G \cdot x = \{g \cdot x : g \in G\}$ se llama \emph{órbita} de $x$ y $\operatorname{Estab}_{G}(x) = \{g \in G : g \cdot x = x\}$ se llama \emph{estabilizador} de $x$ en $G$. Obsérvese que las órbitas forman una partición de $G$.

En el caso de acciones por la derecha la órbita de $x$ se denota $x \cdot G$.

\begin{example}{Ejemplos de acciones de grupos}{ejemplos_acciones}
Sea $G$ un grupo arbitrario.

\begin{enumerate}
    \item Consideremos la acción por la izquierda de $G$ en sí mismo dada por $g \cdot x = gx$. Esta acción se llama \emph{acción por traslación a la izquierda}. Análogamente se define una acción por traslación a la derecha. Obsérvese que $\operatorname{Estab}_{G}(x) = 1$ y $G \cdot x = G$, para todo $x \in G$.
    
    Más generalmente, si $H$ es un subgrupo de $G$, entonces $G$ actúa por la izquierda en $G/H$ mediante la regla: $g \cdot xH = (gx)H$. Análogamente se define una acción por la derecha de $G$ en $H \backslash G$. De nuevo, todos los elementos están en la misma órbita y
    
    \[
    \operatorname{Estab}_{G}(xH) = \{g \in G : x^{-1}gx \in H\} = xHx^{-1} = Hx^{-1}
    \]
    
    \item La acción por conjugación de $G$ en sí mismo es la acción por la derecha dada por $x \cdot g = x^{g} = g^{-1}xg$. La órbita $x \cdot G$ es $x^{G}$, la clase de conjugación de $x$ en $G$ y el estabilizador es $\operatorname{Estab}_{G}(x) = C_{G}(x)$, el centralizador de $x$ en $G$.
    
    \item $G$ actúa por la derecha en el conjunto $S$ de sus subgrupos mediante la regla $H \cdot g = H^{g}$. Esta acción se llama \emph{acción por conjugación de $G$ en sus subgrupos}. El estabilizador de $H$ es $\operatorname{Estab}_{G}(H) = \{g \in G : H^{g} = H\} = N_{G}(H)$, el normalizador de $H$ en $G$. Obsérvese que $N_{G}(H)$ es el mayor subgrupo de $G$ que contiene a $H$ como subgrupo normal.
    
    \item Para cada entero positivo $n$, consideramos $S_{n}$ actuando por la izquierda en $\{1, 2, \ldots, n\}$ mediante: $\sigma \cdot x = \sigma(x)$. Claramente, todo elemento está en la misma órbita y $\operatorname{Estab}_{S_{n}}(i) = \{\sigma \in S_{n} : \sigma(i) = i\} \simeq S_{n-1}$.
    
    \item Sea $n$ un número natural y $X$ un conjunto. Entonces la siguiente fórmula define una actuación por la izquierda del grupo simétrico $S_{n}$ en $X^{n}$:
    
    \[
    \sigma \cdot (x_{1}, \ldots, x_{n}) = (x_{\sigma(1)}, \ldots, x_{\sigma(n)}).
    \]
    
    \item Sea $G$ un grupo actuando por la izquierda en un conjunto $X$ y sean $H$ un subgrupo de $G$ e $Y$ un subconjunto de $X$. Si para todo $h \in H$ y todo $y \in Y$ se verifica que $h \cdot y \in Y$ entonces la restricción a $H \times Y$ de la acción de $G$ en $X$ es una acción por la izquierda de $H$ en $Y$.
\end{enumerate}
\end{example}

\begin{proposition}{Propiedades de acciones}{propiedades_acciones}
Sea $G$ un grupo actuando en un conjunto $X$ y sean $x \in X$ y $g \in G$. Entonces

\begin{enumerate}
    \item $\operatorname{Estab}_{G}(x)$ es un subgrupo de $G$.
    \item $[G : \operatorname{Estab}_{G}(x)] = |G \cdot x|$. En particular, si $G$ es finito, entonces el número de elementos de cada órbita es un divisor del orden de $G$.
    \item Si se trata de una acción por la izquierda entonces $\operatorname{Estab}_{G}(g \cdot x) = \operatorname{Estab}_{G}(x)g^{-1}$. Sin embargo, si se trata de una acción por la derecha entonces $\operatorname{Estab}(x \cdot g) = \operatorname{Estab}_{G}(x)g$.
    \item (Ecuación de Órbitas) Si $R$ es un conjunto de representantes de las órbitas de la acción de $G$ en $X$ (es decir, $R$ contiene exactamente un elemento de cada órbita) entonces
    
    \[
    |X| = \sum_{r \in R} |G \cdot r| = \sum_{r \in R} [G : \operatorname{Estab}_{G}(r)].
    \]
\end{enumerate}
\end{proposition}

Aplicando la Proposición 4.27 a la acción de $G$ en sí mismo y en sus subgrupos por conjugación obtenemos el siguiente corolario.

\begin{corollary}{Propiedades de conjugación}{propiedades_conjugacion}
Sea $G$ un grupo y sean $a, g \in G$ y $H$ un subgrupo de $H$.

\begin{enumerate}
    \item $|a^G| = [G : C_G(a)]$. En particular, $a^G$ tiene un único elemento si y solo si $a$ es un elemento del centro $Z(G)$ de $G$.
    \item El cardinal del conjunto de conjugados de $H$ en $G$ es $[G : N_G(H)]$.
    \item $C_G(x^g) = C_G(x)^g$ y $N_G(H^g) = N_G(H)^g$.
    \item (Ecuación de Clases). Si $G$ es finito y $X$ es un subconjunto de $G$ que contiene exactamente un elemento de cada clase de conjugación con al menos dos elementos, entonces
    
    \[
    |G| = |Z(G)| + \sum_{x \in X} [G : C_G(x)].
    \]
\end{enumerate}
\end{corollary}

Si $p$ es un primo, entonces un $p$-grupo finito es un grupo finito de orden una potencia de $p$.

\begin{proposition}{Centro de $p$-grupos}{centro_p_grupos}
Si $G$ es un $p$-grupo finito no trivial para $p$ un primo entonces $Z(G) \neq 1$.
\end{proposition}

\begin{proofbox}
Utilizando la notación del Corolario 4.28 tenemos $|G| = |Z(G)| + \sum_{x \in X} [G : C_G(x)]$. Entonces $|G|$ y $[G : C_G(x)]$ son potencias de $p$ para todo $x \in X$, con lo que $|Z(G)|$ es múltiplo de $p$ y por tanto $Z(G) \neq 1$.
\end{proofbox}

\begin{theorem}{Estructura de $p$-grupos}{estructura_p_grupos}
Si $G$ es un $p$-grupo finito entonces $G$ tiene una cadena de subgrupos normales $1 = G_0 \subset G_1 \subset G_2 \subset \cdots \subset G_n = G$ tales que $[G_i : G_{i-1}] = p$ para todo $i = 1, \ldots, n$.
\end{theorem}

\begin{theorem}{Teorema de Cauchy}{teorema_cauchy}
Si $G$ es un grupo finito cuyo orden es múltiplo de un primo $p$, entonces $G$ tiene un elemento de orden $p$.
\end{theorem}