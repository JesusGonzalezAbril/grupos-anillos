\chapter{Anillos}

\section{Anillos}

\begin{definition}{Anillo}{anillo}
    Un anillo es una terna \((A, +, \cdot)\) formada por un conjunto no vacío \(A\) y dos operaciones \(+\) (suma) y \(\cdot\) (producto) en \(A\) que verifican:
    \begin{enumerate}
        \item \((A, +)\) es un grupo abeliano.
        \item \((A, \cdot)\) es un monoide.
        \item Distributiva del producto respecto de la suma: \(a \cdot (b + c) = (a \cdot b) + (a \cdot c)\) para todo \(a, b, c \in A\).
    \end{enumerate}
    
    Si además \(\cdot\) es conmutativo en \(A\), decimos que \((A, +, \cdot)\) es un anillo conmutativo.
\end{definition}

\begin{remark}
    \begin{itemize}
        \item El neutro de \(A\) con respecto a \(+\) se llama {cero} y se denota \(0\).
        \item El neutro de \(A\) con respecto a \(\cdot\) se llama {uno} y se denota \(1\).
        \item El simétrico de un elemento \(a\) con respecto a \(+\) se llama {opuesto} y se denota \(-a\).
        \item Si \(a\) es invertible con respecto a \(\cdot\), su simétrico se llama {inverso} y se denota \(a^{-1}\).
        \item En general para $+$ y $\cdot$ usamos la notación usual para sumas y productos
        \[
        a \cdot (b + c) = a(b+c)=ab+ac.
        \] 
    \end{itemize}
\end{remark}

Como $(A, +)$ es un grupo, todo elemento de A es invertible respecto de la suma y por tanto cancelable. Diremos que un elemento de $A$ es regular en $A$ si es cancelable con respecto al producto. En caso contrario decimos que el elemento es singular en $A$ o divisor de cero. El termino divisor de cero se justifica por lo siguiente. Supongamos que $x \in A$ no es cancelable respecto al producto, en tal caso existen dos elementos distintos $a \neq b$ tales que $ax = bx$. Pero entonces es inmediato que
\[
(a-b)x = 0
\]
sin embargo, ni $(a-b)$ ni $x$ son cero, por lo que podemos interpretar que ambos son <<divisores del cero>>.

\subsection{Ejemplos de anillo}
\begin{example}{}{}
    Los conjuntos $\Z$, $\Q$, $\R$ y $\C$ son anillos conmutativos con la suma y el producto usuales. Notemos que todo elemento no nulo de $\Q$, $\R$ o $\C$ es invertible. Sin embargo, en $\Z$ solo hay dos elementos invertibles (1 y -1) aunque todos los elementos son regulares menos el 0.
\end{example}

\begin{proofbox}
    Demostrar que se trata de anillos conmutativos es muy sencillo, basta comprobar que se verifican todas las propiedades pertinentes.
    
    Probaremos que en $\C$ todos los elementos salvo el 0 son invertibles, el resto de afirmaciones quedan como ejercicio. Sea $z = a + bi$ un número complejo cualquiera no nulo, en tal caso el número $w = \frac{a-bi}{a^2+b^2}$ verifica
    \[
    zw = \frac{(a+bi)(a-bi)}{a^2+b^2} = \frac{a^2 -abi +abi -b(-1)}{a^2+b^2} = \frac{a^2+b^2}{a^2+b^2} = 1
    \]
    luego $w = z^{-1}$ y por tanto $z$ tiene inverso.
\end{proofbox}

\begin{example}{Producto de anillos}{prod_anillos}
    Sean $A$ y $B$ dos anillos. Entonces el producto cartesiano $A \times B$ tiene una estructura de anillo con las operaciones definidas componente a componente:
    \[
    (a_1, b_1) + (a_2, b_2) = (a_1 + a_2, b_1 + b_2)
    \]
    \[
    (a_1, b_1) \cdot (a_2, b_2) = (a_1 \cdot a_2, b_1 \cdot b_2)
    \]
    Obsérvese que $A \times B$ es conmutativo si y solo si lo son $A$ y $B$, y que esta construcción se puede generalizar a productos cartesianos de cualquier familia (finita o no) de anillos.
\end{example}


\begin{example}{}{}
    Dados un anillo $A$ y un conjunto $X$, el conjunto $A^X$ de las aplicaciones de $X$ en $A$ es un anillo con las siguientes operaciones:
    \[
    (f + g)(x) = f(x) + g(x)
    \]
    \[
    (f \cdot g)(x) = f(x) \cdot g(x)
    \]
    Si definimos la familia de conjuntos $\{A_i = A : i \in X\}$ entonces es inmediato que $\cup_{i\in X} A_i = A$. Recordemos ahora que el producto $\prod_{i \in X} A_i$ es el conjunto de funciones $f: X \to \cup_{i\in X} A_i$, es decir, el conjunto de funciones $f : X \to A$, luego $A^X$ es un anillo correspondiente a un producto <<infinito>> del anillo $A$ consigo mismo. Para más información ver la Definición \ref{defn:producto_cartesiano}.
\end{example}

\begin{example}{Anillo de polinomios}{polinomios}
    Dado un anillo \( A \), un polinomio en una indeterminada es una expresión:
    \[
    P = a_0 + a_1 X + a_2 X^2 + \cdots + a_n X^n,
    \]
    donde \( n \geq 0 \) y \( a_i \in A \) para todo \( i \). El conjunto de polinomios con coeficientes en \( A \) se denota \( A[X] \). La suma y producto en \( A[X] \) se definen de la forma usual.
\end{example}

\begin{example}{Sucesiones}{sucesiones}
    Dado un anillo \( A \), denotamos por \( A[[X]] \) el conjunto de sucesiones \((a_0, a_1, a_2, \ldots)\) de elementos de \( A \). Con las operaciones:
    \[
    (a_0, a_1, \ldots) + (b_0, b_1, \ldots) = (a_0 + b_0, a_1 + b_1, \ldots),
    \]
    \[
    (a_0, a_1, \ldots)(b_0, b_1, \ldots) = (a_0 b_0, a_0 b_1 + a_1 b_0, \ldots),
    \]
    \( A[[X]] \) es un anillo llamado anillo de series de potencias con coeficientes en \( A \).
\end{example}

\subsection{Propiedades de los anillos}

\begin{lemma}{}{prop_anillos}
    Sea $A$ un anillo y sean $a,b,c \in A$. Se verifican las siguientes propiedades
    \begin{enumerate}
        \item Todo elemento de $A$ es cancelable respecto de la suma.
        \item Todo elemento invertible de $A$ es regular en $A$.
        \item Si $b + a = a$ entonces $b=0$. Si $ba=a$ para todo $a$, entonces $b=1$. En particular, el cero y uno son únicos.
        \item El opuesto de $a$ es único y si $a$ es invertible, entonces $a$ tiene un único inverso.
        \item $0a=0=a0$.
        \item $a(-b)=(-a)b=-(ab)$.
        \item $a(b-c)=ab-ac$.
        \item $a$ y $b$ son invertibles si y solo si $ab$ y $ba$ son invertibles. En tal caso $(ab)^{-1}=b^{-1}a^{-1}$.
        \item Si $0=1$ entonces $A=\{0\}$.
    \end{enumerate}
\end{lemma}

\begin{proofbox}
    \begin{enumerate}
        \item Como $A$ es un grupo respecto de la suma todo elemento tiene inverso, y por la Proposición \ref{prop:prop_operaciones} todo elemento invertible (respecto a la suma) es cancelable (respecto a la suma).
        \item De nuevo por la Proposición \ref{prop:prop_operaciones} todo elemento invertible (respecto al producto) es cancelable (respecto al producto).
        \item Si $b + a = a$ entonces como $a$ es cancelable por el apartado 1, tenemos $b=0$. Si $ba=a$ para todo $a$, entonces como el neutro es único $b=1$.
        \item De nuevo se sigue de la Proposición \ref{prop:prop_operaciones}.
        \item Basta aplicar un pequeño truco
        \[
        0a=(0+0)a = 0a + 0a \implies 0 = 0a
        \]
        para el caso $a0$ se procede igual.
        \item Basta notar que
        \[
        ab + a(-b) = a(b-b) = 0, ab + (-a)b = (a-a)b = 0 \implies -(ab)=a(-b) = (-a)b 
        \]
        ya que los opuestos son únicos.
        \item $a(b-c)=a(b+(-c))=ab+a(-c)=ab+(-ac)=ab-ac$.
        \item En primer lugar si $a,b$ son invertibles entonces existen $a^{-1},b^{-1}$ y es fácil ver que
        \[
        ab(b^{-1}a^{-1})=e=(b^{-1}a^{-1})ab, ba(a^{-1}b^{-1})=e=(a^{-1}b^{-1})ba
        \]
        luego $ab,ba$ son invertibles. Para el recíproco, si $ab,ba$ son invertibles entonces
        \[
        a(b(ab)^{-1})=ab(ab)^{-1}=e, ((ba)^{-1}b)a=(ba)^{-1}ba=e
        \]
        por tanto, por la Proposición \ref{prop:prop_operaciones} ambos simétricos $b(ab)^{-1},(ba)^{-1}b$ son iguales (ambos son $a^{-1}$) y $a$ es invertible. Para ver que $b$ es invertible se procede igual.
        \item Si $0=1$ entonces dado $x \in A$ tenemos
        \[
        x = x1 = x0 = 0 \implies A=\{0\}.
        \]
    \end{enumerate}
\end{proofbox}

Dados un anillo $A$, un elemento $a \in A$ y un entero positivo $n$, la notación $na$ (respectivamente $a^n$) representa el resultado de sumar (respectivamente multiplicar) a consigo mismo $n$ veces, y si $n = 0$ convenimos que $0a = 0$ y $a^0 = 1$. Más rigurosamente, a partir de estas últimas igualdades se definen $na$ y $a^n$ de forma recurrente poniendo $(n + 1)a = a + na$ y $a^{n+1} = aa^n$ para $n \geq 0$. Por último, si $n \geq 1$ se define $(-n)a = -(na)$, y si además $a$ es invertible se define $a^{-n} = (a^{-1})^n$.

\begin{lemma}{}{multiplicacion_anillos}
    Sea \( A \) un anillo, \( a, b \in A \), y \( m, n \in \mathbb{Z} \). Se verifican:
    \begin{enumerate}
        \item \( n (a + b) = n a + n b \).
        \item \( (n + m) a = n a + m a \).
        \item Si \( n, m \geq 0 \), entonces \( a^{n + m} = a^n a^m \). Si \( a \) es invertible, la igualdad vale para \( n, m \) arbitrarios.
        \item Si \( A \) es conmutativo y \( n \geq 0 \), entonces \( (a b)^n = a^n b^n \). Si \( a \) y \( b \) son invertibles, la igualdad vale para todo \( n \).
    \end{enumerate}
\end{lemma}

\begin{proofbox}
    \begin{enumerate}
        \item Por inducción: el caso base $n=0$ es inmediato, si lo suponemos para $n$ entonces
        \[
        (n+1) (a + b) = (a+b)+ n a + n b = (n+1) a + (n+1) b.
        \]
        \item Basta aplicar recursivamente que $(n+1)a = a + na$.
        %%%% Faltan casos < 0 en los dos primeros apartados
        \item Basta aplicar recursivamente que $a^{n+1} = a a^n$. Si $a$ es invertible entonces podemos usar que $a^{-n}=(a^{-1})^n$ distinguiendo casos. Por ejemplo si $n>0,m<0, n>m$ entonces
        \[
        a^{n}a^{m}=a^n(a^{-1})^{-m}=a^{n+m}a^{-m}(a^{-1})^{-m}=a^{n+m}.
        \]
        \item Por inducción: el caso base $n=0$ es inmediato, si lo suponemos para $n$ entonces
        \[
        (ab)^{n+1}=ab(ab)^n=aba^nb^n=aa^nbb^n=a^{n+1}b^{n+1}.
        \]
        Cuando \( a \) y \( b \) son invertibles, si $n<0$
        \[
        (ab)^n=((ab)^{-1})^{-n}=(b^{-1}a^{-1})^{-n}=(b^{-1})^{-n}(a^{-1})^{-n}=b^n a^n.
        \]
    \end{enumerate}
\end{proofbox}

\clearpage
\section{Subanillos}

\begin{remark}
    A partir de ahora supondremos que todos los anillos que aparecen son conmutativos.
\end{remark}

Sea $*$ una operación en un conjunto $A$ y sea $B$ un subconjunto de $A$. Decimos que $B$ es cerrado con respecto a $*$ si para todo $a, b \in B$ se verifica que $a * b \in B$. En tal caso podemos
considerar $*$ como una operación en $B$ que se dice inducida por la operación en $A$.

\begin{definition}{Subanillo}{subanillo}
    Un subanillo de un anillo es un subconjunto suyo que con la misma suma y producto es un anillo con el mismo uno.
\end{definition}

\begin{proposition}{Caracterización de subanillos}{carac_subanillos}
    Las siguientes condiciones son equivalentes para \( B \subseteq A \):
    \begin{enumerate}
    \item \( B \) es un subanillo de \( A \).
    \item \( B \) contiene al 1 y es un anillo, luego es cerrado para sumas, productos y opuestos.
    \item \( B \) contiene al 1 y es cerrado para restas y productos.
    \end{enumerate}
\end{proposition}

\begin{proofbox}
    $(1) \implies (2)$: Si \( B \) es un subanillo de \(A\) entonces contiene al 1 y es cerrado para sumas y productos. Por otro lado, como $B$ es un anillo, tiene un cero, que de momento denotamos $0_B$ y cada elemento $b \in B$ tiene un opuesto en $B$. En realidad $0_B + 0_B = 0_B = 0 + 0_B$, con lo que aplicando la propiedad cancelativa de la suma deducimos que $0_B = 0$, o sea, el cero de $A$ está en $B$ y por tanto es el cero de B (el único que puede tener). Por la unicidad del opuesto, el opuesto de $b$ ha de ser el de $A$, con lo que $B$ es cerrado para opuestos.

    \noindent$(2) \implies (3)$: Inmediato.

    \noindent$(3) \implies (1)$: Si \( B \) contiene al 1 y es cerrado para restas, entonces \( 0 = 1 - 1 \in B \), y para \( b \in B \), \(-b = 0 - b \in B \). Además, \( a + b = a - (-b) \in B \), luego \( B \) es cerrado para sumas, por tanto, es un subanillo de $A$.
\end{proofbox}

\begin{example}{Subanillo impropio}{}
    Todo anillo $A$ es un subanillo de si mismo, al que llamamos impropio por oposición al resto de subanillos, que se dicen propios.
\end{example}

\begin{example}{}{}
    En la cadena de contenciones $\Z \subset \Q \subset \R \subset \C$ cada uno es un subanillo de los posteriores.
\end{example}

\begin{example}{}{}
    Si $A$ es un anillo, el subconjunto $\{0\}$ es cerrado para sumas, productos y opuestos. Si $A = \{0\}$ entonces $\{0\}$ sería subanillo de $A$, pero este es el único caso en el que esto pasa pues en todos los demás casos $1 \neq 0$.

    En efecto si $1=0$ entonces para cualquier $a \in A, a = 1a = 0a = 0 \implies A = \{0\}$.
\end{example}

\begin{example}{}{}
    Si $A$ y $B$ son anillos entonces $A \times \{0\}$ es un anillo, pero no es un subanillo de $A \times B$ porque no contiene a $(1_A, 1_B)$.

    De igual manera, $A \times \{1_B\}$ con las operaciones
    \[
    (a,1_B) + (b, 1_B) = (a+b, 1_B), \quad (a,1_B )\cdot (b, 1_B) = (ab, 1_B)
    \]
    es un anillo, pero no es subanillo de $A \times B$ porque las operaciones no son las inducidas por la operaciones de $A \times B$.
\end{example}

\begin{proofbox}
    Veamos que $A \times \{1_B\}$ es un anillo con las operaciones indicadas. Claramente es un grupo abeliano para la suma con $-(a,1_B) = (-a, 1_B)$. También es un monoide para el producto con neutro $(1_A, 1_B)$ ya que $(1_A, 1_B) (a, 1_B) = (a, 1_B)$. Finalmente la conmutatividad es fácil de comprobar gracias a que $A$ es un anillo
    \begin{align*}
        (a, 1_B)[(b, 1_B) + (c, 1_B)] = (a, 1_B)(b+c, 1_B) = (ab + ac, 1_B) \\
        = (ab, 1_B) + (ac, 1_B) = (a, 1_B)(b, 1_B) + (a, 1_B)(c, 1_B).
    \end{align*}
\end{proofbox}

\begin{example}{Subanillo primo}{subanillo_primo}
    Si \(A\) es un anillo entonces el conjunto:
    \[
    \mathbb{Z}1 = \{n1 : n \in \mathbb{Z}\}
    \]
    es un subanillo de \(A\) contenido en cualquier otro subanillo de \(A\). 
    
    Este se conoce como el subanillo primo de \(A\). 
\end{example}

\begin{remark}
    \(\mathbb{Z}\) y los \(\mathbb{Z}_n\) son sus propios subanillos primos, por tanto, no tienen subanillos propios.
\end{remark}

\begin{example}{}{}
    Dado un número entero \(m\), los conjuntos:
    \begin{align*}
        \mathbb{Z}[\sqrt{m}] &= \{a + b\sqrt{m} : a, b \in \mathbb{Z}\} \\
        \mathbb{Q}[\sqrt{m}] &= \{a + b\sqrt{m} : a, b \in \mathbb{Q}\}
    \end{align*}
    son subanillos de \(\mathbb{C}\).
    
    Observaciones:
    \begin{itemize}
        \item Si \(m > 0\), ambos son subanillos de \(\mathbb{R}\)
        \item Si \(m\) es un cuadrado perfecto, estos conjuntos coinciden con \(\mathbb{Z}\) y \(\mathbb{Q}\) respectivamente
        \item Cuando \(m\) no es cuadrado perfecto, la igualdad \(a + b\sqrt{m} = 0\) implica \(a = 0\) y \(b = 0\)
    \end{itemize}
    
    Caso particular importante:
    \begin{itemize}
        \item \(\mathbb{Z}[i] = \{a + bi : a, b \in \mathbb{Z}\}\) con \(i = \sqrt{-1}\) es el {anillo de los enteros de Gauss}
    \end{itemize}
\end{example}

\begin{example}{}{}
    Todo anillo \(A\) puede verse como un subanillo del anillo de polinomios \(A[X]\) identificando los elementos de \(A\) con los {polinomios constantes} (del tipo \(P = a_0\)).
\end{example}

\begin{example}{Diagonal}{diagonal}
    Sea \(A\) un anillo y \(X\) un conjunto. Entonces la {diagonal}:
    \[
    B = \{f \in A^X : f(x) = f(y) \text{ para todo } x, y \in X\}
    \]
    (es decir, el conjunto de las {aplicaciones constantes} de \(X\) en \(A\)) es un subanillo de \(A^X\).
\end{example}

\begin{proofbox}
    Claramente $B$ contiene a la aplicación $1 : A \to X$ dada por $1(x) = 1_A,\ \forall x \in X$. Es fácil notar que esta aplicación es el elemento neutro del producto en $A^X$. Sean $f,g \in B$, entonces $h = f - g$ está en $B$ ya que
    \[
    \forall x,y \in X \quad h(x)= f(x) - g(x) = f(y) - g(y) = h(y)
    \]
    luego $B$ es cerrado para restas, de igual manera sea $H = fg$, entonces
    \[
    \forall x,y \in X \quad H(x)= f(x)g(x) = f(y)g(y) = H(y)
    \]
    lo que prueba que $B$ es cerrado para productos. Por tanto, por \ref{prop:carac_subanillos} $B$ es un subanillo de $A^X$.
\end{proofbox}

\begin{example}{}{}
    Sea \(A = M_n(B)\), donde \(B\) es un anillo. Son subanillos:
    \begin{itemize}
        \item El conjunto de las matrices diagonales
        \item El conjunto de las matrices escalares: \(\{a I_n : a \in B\}\)
        \item El conjunto de las matrices triangulares superiores
    \end{itemize}
\end{example}

\begin{example}{}{}
    Sea \(A = B \times B\), con \(B\) un anillo. Son subanillos:
    \begin{itemize}
        \item \(A_1 = \{(b, b) : b \in B\}\) (la diagonal)
        \item \(A_2 = B_1 \times B_2\), donde \(B_1\) y \(B_2\) son subanillos de \(B\)
    \end{itemize}
\end{example}

\begin{example}{}{}
    Sea \(A\) un anillo cualquiera y \(B\) un subanillo de \(A\). Para \(\alpha \in A\), el conjunto:
    \[
    A_1 = \{a_0 + a_1\alpha + a_2\alpha^2 + \cdots + a_n\alpha^n : n \geq 0, a_0, a_1, \ldots, a_n \in B\}
    \]
    es un subanillo de \(A\) llamado {subanillo generado por \(B\) y \(\alpha\)}.
\end{example}

\clearpage
\section{Homomorfismos de anillos}

\begin{definition}{Homomorfismo de anillos}{hom_anillo}
    Sean \(A\) y \(B\) dos anillos. Un homomorfismo de anillos entre \(A\) y \(B\) es una aplicación \(f: A \to B\) que satisface:
    \begin{enumerate}
        \item \(f(x + y) = f(x) + f(y)\)
        \item \(f(x \cdot y) = f(x) \cdot f(y)\)
        \item \(f(1) = 1\)
    \end{enumerate}
    
    Un isomorfismo es un homomorfismo biyectivo. Dos anillos \(A\) y \(B\) son isomorfos (\(A \cong B\)) si existe un isomorfismo entre ellos.
\end{definition}

\begin{remark}
    En la definición anterior hemos usado el mismo símbolo para las operaciones en ambos anillos, pero es importante notar que:
    \begin{itemize}
        \item En \(f(x + y)\), la suma se realiza en \(A\)
        \item En \(f(x) + f(y)\), la suma se realiza en \(B\)
        \item Lo mismo aplica para el producto
    \end{itemize}
\end{remark}

\begin{definition}{Tipos de homomorfismos}{tipos_hom}
    \begin{itemize}
        \item Un {endomorfismo} es un homomorfismo de un anillo en sí mismo.
        \item Un {isomorfismo} es un homomorfismo biyectivo.
        \item Un {automorfismo} es un isomorfismo de un anillo en sí mismo.
    \end{itemize}
\end{definition}

\begin{example}{}{}
    Si $B = \{0\}$ entonces la aplicación $f(a) = 0_B, \forall a \in A$ es un homomorfismo. Si $B \neq \{0\}$ entonces $f$ no es un homomorfismo ya que $f(1) = 0_B \neq 1_B$.
\end{example}

\begin{proposition}{Propiedades básicas}{prop_hom_basicas}
    Sea \(f: A \to B\) un homomorfismo de anillos. Entonces para todo \(a, b, a_1, \ldots, a_n \in A\) se verifica:
    \begin{enumerate}
        \item \(f(0_A) = 0_B\)
        \item \(f(-a) = -f(a)\)
        \item \(f(a - b) = f(a) - f(b)\)
        \item \(f(a_1 + \cdots + a_n) = f(a_1) + \cdots + f(a_n)\)
        \item \(f(na) = nf(a)\) para todo \(n \in \mathbb{Z}\)
        \item Si \(a\) es invertible en \(A\), entonces \(f(a)\) es invertible en \(B\) y \(f(a^{-1}) = f(a)^{-1}\)
        \item \(f(a_1 \cdots a_n) = f(a_1) \cdots f(a_n)\)
    \end{enumerate}
\end{proposition}

\begin{proofbox}
    En la mayoría de apartados usaremos los anteriores.
    \begin{enumerate}
        \item \(f(0_A) = f(0_A + 0_A) = f(0_A) + f(0_A)\), luego por cancelación en \(B\), \(f(0_A) = 0_B\).
        
        \item \(f(a) + f(-a) = f(a + (-a)) = f(0_A) = 0_B\), luego \(f(-a) = -f(a)\).

        \item \(f(a - b) = f(a + (-b)) = f(a) + f(-b) = f(a) - f(b)\).

        \item Por inducción, el caso $n = 2$ es inmediato. Si lo suponemos cierto para $n$ entonces
        \[
        f(a_1 + \cdots + a_{n+1}) = f(a_1 + \cdots a_n) + f(a_{n+1}) = f(a_1) + \cdots + f(a_n) + f(a_{n+1}).
        \]
        
        \item Por inducción, el caso $n = 1$ es inmediato. Si lo suponemos cierto para $n$ entonces
        \[
        f((n+1) a) = f(na + a) = f(na) + f(a) = nf(a) + f(a) = (n+1) f(a).
        \]

        \item Si \(a\) es invertible \(aa^{-1} = 1_A\), luego \(f(a)f(a^{-1}) = f(aa^{-1}) = f(1_A) = 1_B\), por tanto \(f(a^{-1}) = f(a)^{-1}\).

        \item Por inducción, el caso $n = 1$ es inmediato. Si lo suponemos cierto para $n$ entonces
        \[
        f(a_1 \cdots a_{n+1}) = f(a_1 \cdots a_n)f(a_{n+1}) = f(a_1) \cdots f(a_n) f(a_{n+1}).
        \]
    \end{enumerate}
\end{proofbox}

\begin{definition}{Núcleo e imagen}{nucleo_imagen}
    Sea \(f: A \to B\) un homomorfismo de anillos. Definimos:
    \begin{itemize}
        \item El {núcleo} de \(f\): \(\ker f = \{a \in A : f(a) = 0_B\}\)
        \item La {imagen} de \(f\): \(\operatorname{Im} f = \{f(a) \in B : a \in A\}\)
    \end{itemize}
\end{definition}

Resulta interesante ahora estudiar algunas propiedades del núcleo y la imagen, en concreto, su relación con los ideales (ver \ref{defn:ideal}).

\begin{proposition}{Propiedades del núcleo e imagen}{prop_nucleo_imagen}
    Sea \(f: A \to B\) un homomorfismo de anillos. Entonces:
    \begin{enumerate}
        \item \(\ker f\) es un ideal de \(A\)
        \item \(\operatorname{Im} f\) es un subanillo de \(B\)
        \item \(f\) es inyectivo si y solo si \(\ker f = \{0_A\}\)
        \item \(f\) es sobreyectivo si y solo si \(\operatorname{Im} f = B\)
    \end{enumerate}
\end{proposition}

\begin{proofbox}
    \begin{enumerate}
        \item Para ver que \(\ker f\) es un ideal:
        \begin{itemize}
            \item \(0_A \in \ker f\) pues \(f(0_A) = 0_B\), luego $\ker f$ es no vacío.
            \item Si \(x, y \in \ker f\), entonces \(f(x + y) = f(x) + f(y) = 0_B + 0_B = 0_B\), luego \(x + y \in \ker f\).
            \item Si \(x \in \ker f\) y \(a \in A\), entonces \(f(ax) = f(a)f(x) = f(a) 0_B = 0_B\), luego \(ax \in \ker f\)
        \end{itemize}
        
        \item Basta notar lo siguiente:
        \begin{itemize}
            \item \(1_B = f(1_A) \in \operatorname{Im} f\).
            \item Si \(a, b \in \operatorname{Im} f\), entonces \(f(x) = a, f(y) = b\) para ciertos $x,y \in A$, luego \(a - b = f(x) - f(y) = f(x - y) \in \operatorname{Im} f\).
            \item Si \(a, b \in \operatorname{Im} f\), entonces \(f(x) = a, f(y) = b\) para ciertos $x,y \in A$, luego \(ab = f(x)f(y) = f(xy) \in \operatorname{Im} f\).
        \end{itemize}

        \item Si \(f\) es inyectivo y \(x \in \ker f\), entonces \(f(x) = 0_B = f(0_A)\), luego \(x = 0_A\). Recíprocamente, si \(\ker f = \{0_A\}\) y \(f(a) = f(b)\), entonces \(f(a - b) = 0_B\), luego \(a - b \in \ker f = \{0_A\}\), por tanto \(a = b\).

        \item Es inmediato.
    \end{enumerate}
\end{proofbox}

\subsection{Ejemplos de homomorfismos}

\begin{example}{Homomorfismo inclusión}{}
    Si \(B\) es un subanillo de \(A\), la aplicación inclusión \(i: B \hookrightarrow A\) dada por \(i(b) = b\) es un homomorfismo inyectivo ya que $\ker i = \{0\}$.
\end{example}


\begin{example}{Homomorfismo proyección}{}
    Si \(I\) es un ideal de \(A\), la proyección canónica \(\eta: A \to A/I\) dada por \(\pi(a) = a + I\) es un homomorfismo suprayectivo con \(\ker \eta = I\).
\end{example}

\begin{remark}
    Usamos indiscriminadamente $\eta$ o $p$ para denominar a esta proyección canónica.
\end{remark}

\begin{example}{Homomorfismo de sustitución}{}
    Sea \(A\) un anillo y \(b \in A\). La aplicación \(\varphi_b: A[X] \to A\) dada por:
    \[
    \varphi_b(a_0 + a_1X + \cdots + a_nX^n) = a_0 + a_1b + \cdots + a_nb^n
    \]
    es un homomorfismo suprayectivo llamado homomorfismo de sustitución en \(b\). Para ver que es suprayectivo notemos que dado $a \in A$ el polinomio $a = aX^0 \in A[X]$, luego $\eta(aX^0) = a$.
\end{example}

\begin{example}{Homomorfismo único \(\mathbb{Z} \to A\)}{}
    Para cualquier anillo \(A\), existe un único homomorfismo \(f: \mathbb{Z} \to A\) dado por \(f(n) = n 1_A\).
\end{example}

\begin{example}{Conjugación en \(\mathbb{C}\)}{}
    La conjugación compleja \(f: \mathbb{C} \to \mathbb{C}\) dada por \(f(a + bi) = a - bi\) es un automorfismo de \(\mathbb{C}\). Claramente es inyectivo ($f(z) = 0 \iff z = 0 \implies \ker f = \{0\}$) y también sobreyectivo.
\end{example}

\subsection{Propiedades de los homomorfismos}


\begin{proposition}{Composición de homomorfismos}{comp_hom}
    Si \(f: A \to B\) y \(g: B \to C\) son homomorfismos de anillos, entonces la composición \(g \circ f: A \to C\) es un homomorfismo de anillos.
\end{proposition}

\begin{proofbox}
    Claramente $g \circ f (1) = g(f(1)) = g(1) = 1$. Para la suma
    \[
    g \circ f (x + y) = g(f(x + y)) = g(f(x) + f(y)) = g(f(x)) + g(f(y)) = g \circ f (x) + g \circ f (y)
    \]
    y para el producto
    \[
    g \circ f (xy) = g(f(xy)) = g(f(x)f(y)) = g(f(x))g(f(y)) = g \circ f (x) g \circ f (y).
    \]
\end{proofbox}

\begin{proposition}{Propiedades de isomorfismos}{prop_iso}
    \begin{enumerate}
        \item La composición de isomorfismos es un isomorfismo.
        \item Si \(f: A \to B\) es un isomorfismo, entonces \(f^{-1}: B \to A\) es un isomorfismo.
        \item La relación <<ser isomorfo>> es una relación de equivalencia en la clase de todos los anillos.
    \end{enumerate}
\end{proposition}

\begin{proofbox}
    \begin{enumerate}
        \item La composición de homomorfismos es homomorfismo y la composición de aplicaciones biyectivas es biyectiva.

        \item Claramente $f^{-1}(1) = 1$. Para la suma
        \[
        f(f^{-1}(x + y)) = x + y = f(f^{-1}(x)) + f(f^{-1}(y)) = f(f^{-1}(x) + f^{-1}(y))
        \]
        luego por la inyectividad de $f$ debe ser $f^{-1}(x + y) = f^{-1}(x) + f^{-1}(y)$.
        Para el producto hacemos el mismo truco
        \[
        f(f^{-1}(xy)) = xy = f(f^{-1}(x))f(f^{-1}(y)) = f(f^{-1}(x)f^{-1}(y))
        \]
        luego por la inyectividad de $f$ tenemos $f^{-1}(xy) = f^{-1}(x) f^{-1}(y)$.
        \item Resumidamente
        \begin{itemize}
            \item Reflexividad: basta considerar la identidad $id$ que es isomorfismo.

            \item Simetría: si $A \cong B$ entonces existe $f: A \to B$ isomorfismo, luego $f^{-1} : B \to A$ es isomorfismo y, por tanto, $B \cong A$.

            \item Transitividad: se sigue de que la composición de isomorfismos es isomorfismo.
        \end{itemize}
    \end{enumerate}
\end{proofbox}

\begin{proposition}{Preservación de subestructuras}{preservacion_subestructuras}
    Sea \(f: A \to B\) un homomorfismo de anillos.
    \begin{enumerate}
        \item Si \(A_1\) es un subanillo de \(A\), entonces \(f(A_1)\) es un subanillo de \(B\)
        \item Si \(B_1\) es un subanillo de \(B\), entonces \(f^{-1}(B_1)\) es un subanillo de \(A\)
        \item Si \(I\) es un ideal de \(B\), entonces \(f^{-1}(I)\) es un ideal de \(A\)
        \item Si \(f\) es sobreyectivo e \(I\) es un ideal de \(A\), entonces \(f(I)\) es un ideal de \(B\).
    \end{enumerate}
\end{proposition}

\begin{proofbox}
    \begin{enumerate}
        \item Como $A_1$ es subanillo contiene al uno, luego $f(A_1)$ también, que es cerrado para restas y productos es inmediato.
        \item Como $B_1$ es subanillo contiene al uno, luego $f^{-1}(B_1)$ también. Sean $x, y \in f^{-1}(B_1)$, es cerrado para restas ya que
        \[
            f(x - y) = f(x) - f(y) \in B_1 \implies x - y \in f^{-1}(B_1)
        \]
        al ser $B_1$ cerrado para restas. Para el producto se razona igual.
        \item Si \(I\) es un ideal contiene al $0$, luego \(f^{-1}(I)\) es no vacío. Si $x,y \in f^{-1}(I)$ entonces $f(x),f(y) \in I$, por tanto, $f(x + y) = f(x) + f(y) \in I$ y finalmente $x + y \in f^{-1}(I)$. De igual manera, sea $a \in A$, entonces
        \[
        f(ax) = f(a)f(x) \in I \implies ax \in f^{-1}(I)
        \]
        ya que $f(a) \in B$ e $I$ es un ideal.
        \item Claramente $f(I)$ es no vacío ya que $0\in I \implies f(0)\in f(I)$. Si $x,y \in f(I)$, entonces, $x = f(a), y = f(b)$ y 
        \[
        x + y = f(a) + f(b) = f(a + b) \in f(I)
        \]
        ya que $a + b \in I$ al ser ideal. Para el producto necesitaremos la sobreyectividad, dado $z \in B$ entonces existe $c\in A$ tal que $z = f(c)$, luego
        \[
        zx = f(c)f(a) = f(ca) \in f(I)
        \]
        porque $ca \in I$ al ser ideal.

    \end{enumerate}
\end{proofbox}

\begin{remark}
    La imagen de un ideal por un homomorfismo no necesariamente es un ideal si el homomorfismo no es suprayectivo.
\end{remark}

\begin{example}{Contraejemplo}{}
    Sea \(i: \mathbb{Z} \hookrightarrow \mathbb{Q}\) la inclusión. El conjunto \(2\mathbb{Z}\) es un ideal de \(\mathbb{Z}\), pero \(i(2\mathbb{Z}) = 2\mathbb{Z}\) no es un ideal de \(\mathbb{Q}\), pues por ejemplo \(\frac{1}{2} \in \mathbb{Q}\) y \(2 \in 2\mathbb{Z}\), pero \(\frac{1}{2} \cdot 2 = 1 \notin 2\mathbb{Z}\) en \(\mathbb{Q}\).
\end{example}

\begin{theorem}{Homomorfismos en productos}{hom_productos}
    Sean \(A, B, C\) anillos. Existe una biyección natural $\Phi$ tal que
    \[
    \operatorname{Hom}(A, B \times C) \cong \operatorname{Hom}(A, B) \times \operatorname{Hom}(A, C)
    \]
    dada por \(f \mapsto (\pi_B \circ f, \pi_C \circ f)\), donde \(\pi_B\) y \(\pi_C\) son las proyecciones canónicas.
\end{theorem}

\begin{example}{Aplicación}{}
    Para determinar todos los homomorfismos \(f: \mathbb{Z} \to \mathbb{Z}_2 \times \mathbb{Z}_3\), basta determinar los homomorfismos \(\mathbb{Z} \to \mathbb{Z}_2\) y \(\mathbb{Z} \to \mathbb{Z}_3\) por separado.
\end{example}

\begin{proofbox}
    Si $f : \Z \to \Z_2$ es un homomorfismo entonces
    \[
    f(1) = [1]_2, f(n) = nf(1) = n[1]_2 = [n]_2
    \]
    luego solo existe un homomorfismo de ese tipo.

    Similarmente, si $f : \Z \to \Z_3$ es un homomorfismo entonces
    \[
    f(1) = [1]_3, f(n) = nf(1) = n[1]_3 = [n]_3
    \]
    por tanto este homomorfismo también esta totalmente determinado.

    Finalmente deducimos que el único homomorfismo de $\Z$ a $\Z_2 \times \Z_3$ es 
    \[
    g(n) = ([n]_2, [n]_3).
    \]
\end{proofbox}

\clearpage
\section{Ideales y anillos cociente}

Supongamos que queremos construir un anillo cociente $A/I$ que herede la estructura algebraica de $A$. Para ello, necesitamos definir una relación de equivalencia compatible con las operaciones del anillo. Si consideramos $(A,+)$ como grupo abeliano bajo la suma, para formar el grupo cociente $(A/I,+)$ necesitamos que $I$ sea un subgrupo normal de $(A,+)$. Como $A$ es abeliano, todo subgrupo es normal, por lo que basta con que $I$ sea un subgrupo de $(A,+)$, es decir:
\begin{itemize}
    \item $0 \in I$
    \item Si $a,b \in I$, entonces $a+b \in I$
    \item Si $a \in I$, entonces $-a \in I$
\end{itemize}

Esto nos permite definir el grupo cociente $(A/I,+)$ con la operación:
\[
(a+I) + (b+I) = (a+b) + I
\]
que está bien definida gracias a que $I$ es subgrupo.

Sin embargo, para que $A/I$ herede la estructura de anillo, necesitamos definir también una multiplicación. La definición natural sería:
\[
(a+I)(b+I) = ab + I
\]

Sin embargo, debemos verificar que esta operación esta bien definida. Supongamos que tomamos diferentes representantes de las clases:
\[
a + I = a' + I, \quad b + I = b' + I
\]
esto significa que
\[
a - a' \in I \quad \text{y} \quad b - b' \in I.
\]
Para que el producto esté bien definido, debemos tener:
\[
ab + I = a'b' + I
\]
es decir
\[
ab - a'b' \in I.
\]
Finalmente observemos que:
\[
ab - a'b' = ab - ab' + ab' - a'b' = a(b-b') + (a-a')b'
\]

Como $b-b' \in I$ y $a-a' \in I$, para garantizar que $ab - a'b' \in I$, necesitamos que:
\begin{enumerate}
    \item Si $x \in I$ y $a \in A$, entonces $ax \in I$
    \item Si $x \in I$ y $a \in A$, entonces $xa \in I$
\end{enumerate}
En un anillo conmutativo, estas dos condiciones son equivalentes. Esto nos lleva a la siguiente definición.

\begin{definition}{Ideal}{ideal}
    Un subconjunto \(I\) de un anillo \(A\) es un {ideal} si:
    \begin{enumerate}
        \item \(I \neq \emptyset\)
        \item Para todo \(x, y \in I\), se verifica que \(x + y \in I\)
        \item Para todo \(x \in I\) y \(a \in A\), se verifica que \(ax \in I\)
    \end{enumerate}
\end{definition}

Si $I$ es un ideal de $A$ escribiremos $I \leq A$.

\begin{remark}
    \begin{itemize}
        \item La condición \(I \neq \emptyset\) puede sustituirse por \(0 \in I\), ya que si \(a \in I\) entonces \(0 = a + (-1)a \in I\).
        \item Si \(I\) es un ideal de \(A\), entonces para todo \(a_1, \ldots, a_n \in A\) y \(x_1, \ldots, x_n \in I\) se tiene que \(\sum_{i=1}^n a_i x_i \in I\).
        \item Todo ideal es un grupo respecto de la suma.
    \end{itemize}
\end{remark}

\begin{example}{Ideales triviales}{}
    \begin{itemize}
        \item El {ideal cero}: \(\{0\}\)
        \item El {ideal impropio}: \(A\)
    \end{itemize}
\end{example}

Todo aquel ideal que no sea impropio, es decir, que verifique $I \leq A, I \neq A$ se llama ideal propio. En ocasiones nos interesará trabajar solo con ideales que sean propios, por lo que resulta muy útil caracterizar aquellos que no lo son.

\begin{lemma}{Caracterización de ideales impropios}{caract_ideales_impropios}
    Sea \(A\) un anillo. Para un ideal \(I \leq A\), las siguientes condiciones son equivalentes:
    \begin{enumerate}
        \item \(I = A\), es decir, $I$ es un ideal impropio.
        \item \(1 \in I\)
        \item \(I\) contiene una unidad de \(A\) (i.e., \(I \cap A^* \neq \emptyset\))
    \end{enumerate}
\end{lemma}

\begin{proofbox}
    \begin{itemize}
        \item (1) \(\implies\) (2): si \(I = A\), entonces \(1 \in I\)
        \item (2) \(\implies\) (3): \(1\) es una unidad
        \item (3) \(\implies\) (1): si \(u \in I \cap A^*\), entonces \(1 = uu^{-1} \in I\), luego \(I = A\)
    \end{itemize}
\end{proofbox}

\subsection{Ejemplos de ideales}

\begin{example}{Ideales principales}{}
    Sea \(A\) un anillo y \(b \in A\). El conjunto:
    \[
    (b) = bA = \{ba : a \in A\}
    \]
    es un ideal de \(A\) llamado {ideal principal generado por \(b\)}.
    
    Observaciones:
    \begin{itemize}
        \item \((1) = A\)
        \item \((0) = \{0\}\)
        \item \((b)\) es el menor ideal de \(A\) que contiene a \(b\)
    \end{itemize}
\end{example}

\begin{example}{Ideal generado por un conjunto}{}
    Sea \(T \subseteq A\). El {ideal generado por \(T\)} es:
    \[
    (T) = \left\{\sum_{i=1}^n a_i t_i : n \geq 0, a_i \in A, t_i \in T\right\}
    \]
    Este es el menor ideal de \(A\) que contiene a \(T\).
\end{example}

\begin{remark}
    Frecuentemente, cuando el conjunto sea finito escribiremos
    \[
    (\{x_1, x_2,\dots, x_n\}) = (x_1, x_2, \dots, x_n)
    \]
\end{remark}

\begin{example}{Ideales en anillos producto}{}
    Si \(A\) y \(B\) son anillos, entonces \(A \times \{0\} = \{(a, 0) : a \in A\}\) es un ideal de \(A \times B\).
\end{example}

\begin{proofbox}
    Claramente es no vacío. Si $x,y \in A \times \{0\}$ entonces 
    \[
    x = (a,0),\ y = (b,0) \implies x + y = (a+b, 0) \in A \times \{0\}.
    \]
    Si $(a',b') \in A \times B$ entonces 
    \[
    (a',b')x = (a',b')(a,0) = (aa',0) \in A \times \{0\}.
    \]
\end{proofbox}

\begin{example}{Ideales en anillos de polinomios}{}
    Sea \(A[X]\) el anillo de polinomios.
    \begin{itemize}
        \item \(I = \{a_1X + \cdots + a_nX^n : a_i \in A\}\) (polinomios sin coeficiente independiente) es un ideal
        \item Si \(I\) es ideal de \(A\), entonces \(J = \{a_0 + a_1X + \cdots + a_nX^n : a_0 \in I\}\) es un ideal de \(A[X]\)
        \item \(I[X] = \{a_0 + a_1X + \cdots + a_nX^n : a_i \in I\}\) es un ideal de \(A[X]\)
    \end{itemize}
\end{example}

\begin{proofbox}
    \begin{itemize}
        \item Que \(I\) es no vacío es inmediato. Si $P,Q \in I$ entonces son de la forma
        \[
        P = a_1X + \cdots + a_nX^n,\ Q = b_1X + \cdots + b_mX^m
        \]
        donde podemos suponer sin pérdida de generalidad que $m \geq n$, luego definiendo $c_k = a_k + b_k$ (tomando $a_k = 0$ si $k > n$) tenemos
        \[
        P + Q = c_1 X + \cdots + c_mX^m \in I.
        \]
        De igual manera, el producto de polinomios sin término independiente es un polinomio sin término independiente. Sea $d_0, d_1 = 0, d_k = \sum_{i + j = k} a_{i} b_{j}$
        \[
        PQ = a_1b_1 X^2 + \cdots + a_nb_m X^{n + m} = d_2 X^2 + \cdots + d_{n + m} X^{n + m} \in I.
        \]

        \item Como $I$ es no vacío existe $y \in I$, luego el polinomio $y = yX^0 \in J$ y $J$ es no vacío. Dados $P,Q \in J$ su suma es el polinomio con coeficientes obtenidos sumando los de $P$ y $Q$, como ambos coeficientes independiente están en $I$, que es un ideal, su suma también está en $I$, luego $P + Q \in J$. Para el producto, notemos que el coeficiente independiente de $PQ$ es el producto de dos elementos de $I$, luego está en $I$ y por tanto $PQ \in J$. 

        \item Como $I$ es no vacío existe $y \in I$, luego el polinomio $y = yX^0 \in I[X]$ e $I[X]$ es no vacío. Dados $P,Q \in I[X]$ su suma es el polinomio con coeficientes obtenidos sumando los de $P$ y $Q$, como estos coeficientes están en $I$, que es un ideal, su suma también está en $I$, luego $P + Q \in I[X]$. Para el producto, notemos que los coeficientes de $PQ$ son combinaciones de elementos obtenidos como producto de dos elementos de $I$, luego los coeficientes de $PQ$ están en $I$ y por tanto $PQ \in I[X]$.
    \end{itemize}
\end{proofbox}

\begin{proposition}{Intersección de ideales}{interseccion_ideales}
    La intersección de cualquier familia de ideales de \(A\) es un ideal de \(A\).
\end{proposition}

\begin{proofbox}
    Si $I_\alpha$ es una familia de ideales indexada por $X$ y $J = \cap_{\alpha \in X} I_\alpha$ entonces
    \[
    0 \in I_\alpha \ \forall \alpha \in X \implies 0 \in J \implies J \neq \emptyset
    \]
    Además, 
    \[
    x,y \in J \implies x,y \in I_\alpha \ \forall \alpha \in X \implies x+y \in I_\alpha \ \forall \alpha \in X \implies x+y \in J
    \]
    y para cualquier $a \in A$
    \[
    x \in J \implies x \in I_\alpha \ \forall \alpha \in X \implies ax \in I_\alpha \ \forall \alpha \in X \implies ax \in J.
    \]
\end{proofbox}

\begin{proposition}{Ideales de \(\mathbb{Z}\)}{ideales_Z}
    Todos los ideales de \(\mathbb{Z}\) son principales. Es decir, para todo ideal \(I \subset \mathbb{Z}\), existe \(n \in \mathbb{Z}\) tal que \(I = (n)\).
\end{proposition}

\begin{proofbox}
    Sea \(I\) un ideal de \(\mathbb{Z}\). Si \(I=0\) entonces \(I=(0)\) con lo que \(I\) es principal. Supongamos que \(I\neq 0\) y sea \(n\in I\setminus 0\). Entonces \(-n\in I\), con lo que \(I\) tiene un elemento positivo, o sea \(I\cap\mathbb{N}\neq\emptyset\). Como \(\mathbb{N}\) está bien ordenado, \(I \cap \N\) tiene un mínimo que denotamos como \(a\). Como \(a\in I\) se tiene que \((a)\subseteq I\).
    
    Para ver que se da la igualdad tomamos \(b\in I\) y sean \(q\) y \(r\) el cociente y el resto de la división entera de \(b\) entre \(a\). Entonces \(b=qa+r\) y \(0\leq r<a\). Pero \(r=b-qa\in I\), por que \(I\) es un ideal de \(\mathbb{Z}\) que contiene a \(a\) y \(b\) y \(q\in\mathbb{Z}\). Como \(r\) es estrictamente menor que \(a\) y \(a\) es mínimo en \(I\cap\mathbb{N}\), necesariamente \(r\not\in\mathbb{N}\), es decir \(r\) no es positivo. Luego \(r=0\), con lo que \(b=qa\in(a)\).
\end{proofbox}

\subsection{Anillos cociente}

\begin{definition}{Congruencia módulo un ideal}{congruencia_ideal}
    Sea \(I\) un ideal de un anillo \(A\). Decimos que \(a, b \in A\) son {congruentes módulo \(I\)}, y escribimos \(a \equiv b \pmod{I}\), si \(b - a \in I\).
\end{definition}

\begin{lemma}{Propiedades de la congruencia}{prop_congruencia}
    Sea \(I\) ideal de \(A\). Para todo \(a, b, c, d \in A\):
    \begin{enumerate}
        \item \(a \equiv a \pmod{I}\) (reflexiva).
        \item Si \(a \equiv b \pmod{I}\), entonces \(b \equiv a \pmod{I}\) (simétrica).
        \item Si \(a \equiv b \pmod{I}\) y \(b \equiv c \pmod{I}\), entonces \(a \equiv c \pmod{I}\) (transitiva).
        \item \(a \equiv b \pmod{(0)}\) si y solo si \(a = b\).
    \end{enumerate}
\end{lemma}

\begin{proofbox}
    \begin{enumerate}
        \item Como $0 \in A$, dado $x \in I$ debe ser $0x = 0 \in I$, luego \(a - a = 0 \in I \implies a \equiv a \pmod{I}\).
        \item Si \(a \equiv b \pmod{I}\), entonces
        \[
        b - a \in I \implies (-1)(b-a) = a - b \in I \implies b \equiv A \pmod{I}.
        \]
        \item Si \(a \equiv b \pmod{I}\) y \(b \equiv c \pmod{I}\), entonces
        \[
        b - a \in I, c - b \in I \implies c - a \in I \implies a \equiv c \pmod{I}.
        \]
        \item \(a \equiv b \pmod{(0)} \iff b - a = 0 \iff a = b\).
    \end{enumerate}
\end{proofbox}

Del Lema \ref{lem:prop_congruencia} se deduce que la relación <<ser congruente módulo $I$>> es una relación de equivalencia en $A$ y, por tanto, las clases de equivalencia por esta relación definen una partición de $A$. 

La clase de equivalencia que contiene a un elemento $a \in A$ es
\[
a + I = \{a + x : x \in I\}
\]
(en particular $0 + I = I$), de modo que
\[
a + I = b + I \Leftrightarrow a \equiv b \pmod{I}
\]
(en particular $a + I = 0 + I \Leftrightarrow a \in I$). 

El conjunto de las clases de equivalencia se denota
\[
A/I = \frac{A}{I} = \{a + I : a \in A\}.
\]
En ocasiones se escriben las clases de equivalencia como
\[
\bar{a} = a + I, \bar{b} = b + I, \bar{0} = 0 + I = I.
\]

\begin{definition}{Anillo cociente}{anillo_cociente}
    Sea \(I\) un ideal de \(A\). El conjunto de clases de equivalencia:
    \[
    A/I = \{a + I : a \in A\}
    \]
    con las operaciones:
    \begin{align*}
        (a + I) + (b + I) &= (a + b) + I \\
        (a + I) \cdot (b + I) &= (ab) + I
    \end{align*}
    es un anillo llamado {anillo cociente de \(A\) módulo \(I\)}.
\end{definition}

\begin{proposition}{Buena definición del cociente}{bien_def_cociente}
    Las operaciones en \(A/I\) están bien definidas y dotan a \(A/I\) de estructura de anillo con:
    \begin{itemize}
        \item Elemento cero: \(0 + I\)
        \item Elemento uno: \(1 + I\)
    \end{itemize}
\end{proposition}

\begin{proofbox}
    Sean \(a + I = a' + I\) y \(b + I = b' + I\). Entonces \(a - a', b - b' \in I\). Luego
    \begin{itemize}
        \item La suma está bien definida, para ello veamos que $(a+b) + I = (a' + b') + I$, o equivalentemente, $(a + b) - (a' + b') \in I$, en efecto
        \[
        (a + b) - (a' + b') = (a - a') + (b - b') \in I
        \]
        ya que $a - a', b - b' \in I$.
        \item El producto está bien definido, para ello veamos que $(ab) + I = (a'b') + I$, es decir, $(ab) - (a'b') \in I$
        \[
        ab - a'b' = ab - ab' + ab' - a'b' = a(b - b') + (a - a')b' \in I
        \]
        de nuevo porque $a - a', b - b' \in I$.
    \end{itemize}
    Por tanto las operaciones están bien definidas. La comprobación del cero y el uno son inmediatas.
\end{proofbox}

\begin{definition}{Proyección canónica}{proyeccion_canonica}
    La aplicación \(\eta: A \to A/I\) dada por \(\eta(a) = a + I\) es un homomorfismo sobreyectivo llamado {proyección canónica}.
\end{definition}

\begin{proofbox}
    Que la proyección es sobreyectiva es inmediato, dado $a + I \in A/I$ es inmediato que $\eta(a) = a + I$. Comprobar que es un homomorfismo es trivial por la manera en que hemos definido las operaciones en $A/I$.
\end{proofbox}

\begin{example}{Anillos \(\mathbb{Z}_n\)}{}
    Para \(n > 0\), \(\mathbb{Z}_n\)  es el anillo cociente \(\mathbb{Z}/(n)\). Tiene exactamente \(n\) elementos: \(0 + (n), 1 + (n), \ldots, n-1 + (n)\).
\end{example}

\begin{example}{Cocientes triviales}{}
    \begin{itemize}
        \item \(A/\{0\} \cong A\)
        \item \(A/A \cong \{0\}\)
    \end{itemize}
\end{example}

\begin{example}{Cociente por ideales en polinomios}{cociente_sustitucion}
    Sea \(I = \{a_1X + \cdots + a_nX^n\} \leq A[X]\). Entonces:
    \[
    A[X]/I \cong A
    \]
    mediante el isomorfismo que envía \(P(X) + I\) al término constante de \(P\).
\end{example}

\begin{proofbox}
    Sea $P(X) \in A[X], P(X) = a_0 + a_1 X + \cdots + a_n X^n$ entonces $P(X) + I \in A[X]/I$ es de la forma $P(X) + I = a_0 + I$ ya que
    \[
    P(X) - a_0 = a_1 X + \cdots + a_n X^n \in I.
    \]
    luego el isomorfismo es $\phi(a_0 + I) = a_0$. Claramente es un homomorfismo sobreyectivo, para ver que es inyectivo supongamos
    \[
    \phi(a_0 + I) = \phi(b_0 + I) \implies a_0 = b_0 \implies a_0 - b_0 = 0 \in I \implies a_0 + I = b_0 + I.
    \]
\end{proofbox}

\begin{example}{Cociente en productos}{cociente_productos}
    Sean \(A, B\) anillos, \(I = A \times \{0\}\). Entonces:
    \[
    (A \times B)/I \cong B
    \]
\end{example}

\subsection{Teorema de correspondencia}

Recordemos algunas definiciones y caracterizaciones útiles.

\begin{definition}{Núcleo de un homomorfismo}{nucleo_homomorfismo}
    Sea \(f: A \to B\) un homomorfismo de anillos. El {núcleo} de \(f\) es:
    \[
    \ker f = \{a \in A : f(a) = 0\}
    \]
\end{definition}

\begin{proposition}{Inyectividad y núcleo}{inyectividad_nucleo}
    Un homomorfismo \(f: A \to B\) es inyectivo si y solo si \(\ker f = \{0\}\).
\end{proposition}

\begin{proofbox}
    \begin{itemize}
        \item Si \(f\) es inyectivo y \(a \in \ker f\), entonces \(f(a) = 0 = f(0)\), luego \(a = 0\)
        \item Si \(\ker f = \{0\}\) y \(f(a) = f(b)\), entonces \(f(a - b) = 0\), luego \(a - b \in \ker f = \{0\}\), por tanto \(a = b\)
    \end{itemize}
\end{proofbox}

El siguiente resultado describe los ideales de un anillo cociente. Consideremos la proyección canónica $\eta : A \to A / I$. La imagen de un subconjunto $J \subset A$ es 
\[
\eta(J) = \{a + I : a \in J\}
\]
si $J$ contiene a $I$ denotaremos a este conjunto $\eta(J) = J/I$.

\begin{theorem}{Teorema de correspondencia}{teorema_correspondencia}
    Sea \(I\) un ideal de un anillo \(A\). Sea $\mathcal{A}$ el conjunto de ideales de \(A\) que contienen a \(I\)
    \[
    \mathcal{A} = \{J \leq A : I \subseteq J\}.
    \]
    Sea $\mathcal{K}$ el conjunto de todos los ideales de \(A/I\)
    \[
    \mathcal{K} = \{ K \leq A/I \}.
    \]
    Entonces las asignaciones
    \begin{align*}
    &\Phi : \mathcal{A} \to \mathcal{K},\ \Phi(J) = J/I \\
    &\Psi : \mathcal{K} \to \mathcal{A},\ \Psi(K) = \eta^{-1}(K)
    \end{align*}
    definen aplicaciones biyectivas, una inversa de la otra, que conservan la inclusión en $\mathcal{A}$ y $\mathcal{K}$.
\end{theorem}

\begin{proofbox}
    En primer lugar, veamos que son aplicaciones, para ello necesitamos que se cumpla
    \begin{itemize}
        \item Si $J$ es un ideal que contiene a $I$ entonces $\eta(J) = J/I$ es un ideal.
        \begin{itemize}
            \item Como $J$ es un ideal $0 \in J$, luego $0 + I = \eta(0) \in \eta(J) = J/I \neq \emptyset$.
            \item Sean $x + I, y + I \in J/I$ (es decir, $x, y \in J$ tales que $\eta(x) = x + I, \eta(y) = y + I$). Es claro que
            \[
            \eta(x + y) = (x + y) + I = (x + I) + (y + I) \in J/I
            \]
            como necesitamos.
            \item Sea $x + I \in J/I$, $a + I \in A/I$ entonces
            \[
            (a + I)(x + I) = ax + I = \eta(ax) \in J/I
            \]
            ya que $ax \in J$ al ser $J$ ideal.
        \end{itemize}

        \item Si $K$ es un ideal de $A/I$ entonces $\eta^{-1}(K)$ es un ideal que contiene a $I$.
        \begin{itemize}
            \item Como $K$ es un ideal $0 + I \in K$, y al ser $0 + I = \eta(0) \implies 0 \in \eta^{-1}(K) \neq \emptyset$.
            
            \item Sean $x,y \in \eta^{-1}(K)$, entonces $x + I, y + I \in K \implies (x+y) + I \in K$. Finalmente
            \[
            \eta(x + y) = (x + y) + I = \in K \implies (x + y) \in \eta^{-1}(K)
            \]
            como necesitamos.
            \item Sean $x \in \eta{-1}(K)$, $a \in A$ entonces
            \[
            \eta(ax) = ax + I = (a + I)(x + I) \in K
            \]
            ya que $a + I \in A/I$ y $K$ es un ideal, pero entonces $ax \in \eta^{-1}(K)$ como queríamos ver.

            \item Sea $x \in I$, entonces
            \[
            \eta(x) = 0 + I \in K \implies x \in \eta^{-1}(K).
            \]
        \end{itemize}
    \end{itemize}

    Veamos ahora que una es inversa de la otra, lo cual implica directamente que son biyectivas.
    \begin{itemize}
        \item Dado $J \in \mathcal{A}$
        \[
        \Psi(\Phi(J)) = \eta^{-1}(\eta(J)) \supseteq J
        \]
        por las propiedades básicas de las aplicaciones. Para la otra inclusión, si $x \in \eta^{-1}(\eta(J))$ entonces $\eta(x) \in \eta(J)$, luego existe $y\in J$ tal que 
        \[
        x + I = \eta(y) = y + I \implies x - y \in I \subseteq J \implies x = (x - y) + y \in J
        \] 
        usando que $J$ es ideal.

        \item Sea ahora $K \in \mathcal{K}$, entonces 
        \[
        \Phi(\Psi(K)) = \eta(\eta^{-1}(K)) \subseteq K
        \]
        por las propiedades de las aplicaciones. Por otro lado, si $x + I \in K$ entonces, al ser $\eta$ sobreyectiva existe $y \in \eta^{-1}(K)$ tal que $\eta(y) = x + I \in K$. Por tanto $x + I \in \eta(\eta^{-1}(K))$ 
    \end{itemize}

    Finalmente veamos que respetan las inclusiones.
    \begin{itemize}
        \item Si $J,J' \in \mathcal{A}, J \subseteq J'$ entonces dado
        \[
        x+I \in \Phi(J) = \eta(J) \implies x+I = \eta(j),\, j \in J \subseteq J' \implies x+I \in \eta(J') = \Phi(J'),
        \]
        es decir $\Phi(J) \subseteq \Phi(J')$.
        \item De igual manera, si $K, K' \in \mathcal{K}, K \subseteq K'$ entonces
        \[
        x \in \Psi(K) = \eta^{-1}(K) \implies \eta(x) \in K \subseteq K' \implies x \in \eta^{-1}(K') = \Psi(J'),
        \]
        es decir, $\Psi(K) \subseteq \Psi(K')$.
    \end{itemize}
\end{proofbox}

\begin{remark}
    En \ref{prop:preservacion_subestructuras} ya habíamos probado casi todo lo que necesitabamos para ver que $\Phi, \Psi$ son aplicaciones.
\end{remark}

\begin{remark}
    Recordemos que
    \[
    \eta(\eta^{-1}(K)) = K
    \]
    es una de las caracterizaciones vista en Conjuntos y Números para que una aplicación $\eta$ sea sobreyectiva.
\end{remark}

\begin{example}{Aplicación del teorema de correspondencia}{}
    En \(\mathbb{Z}_n = \mathbb{Z}/(n)\), los ideales son de la forma \(d\mathbb{Z}_n = (d)/(n)\) donde \(d \mid n\). Además, \(d\mathbb{Z}_n \subseteq d'\mathbb{Z}_n\) si y solo si \(d' \mid d\).
\end{example}

\clearpage

\section{Operaciones con ideales}

\begin{definition}{Suma de ideales}{suma_ideales}
    Si \(I\) y \(J\) son ideales de \(A\), su {suma} es:
    \[
    I + J = \{x + y : x \in I, y \in J\} = (I \cup J)
    \]
\end{definition}

\begin{definition}{Producto de ideales}{producto_ideales}
    Si \(I\) y \(J\) son ideales de \(A\), su {producto} es:
    \[
    IJ = \left\{\sum_{i=1}^n x_i y_i : x_i \in I, y_i \in J, n \geq 0\right\} = \Big(\{xy : x \in I, y \in J\}\Big)
    \]
\end{definition}

\begin{remark}
    Más generalmente, para ideales \(I_1, \ldots, I_n\):
    \begin{itemize}
        \item \(I_1 + \cdots + I_n = \{x_1 + \cdots + x_n : x_i \in I_i\}\)
        \item \(I_1 \cdots I_n\) está generado por productos \(x_1 \cdots x_n\) con \(x_i \in I_i\)
    \end{itemize}
\end{remark}

\begin{proposition}{Propiedades de las operaciones}{prop_operaciones_ideales}
    Para ideales \(I, J, K\) de \(A\):
    \begin{enumerate}
        \item \(IJ \subseteq I \cap J\)
        \item \(I(J \cap K) \subseteq IJ \cap IK\)
        \item \(I(JK) = (IJ)K\)
        \item \(I(J + K) = IJ + IK\)
        \item \(IA = I\)
    \end{enumerate}
\end{proposition}

\begin{proofbox}
    \begin{enumerate}
        \item Sea \(x \in IJ\), entonces \(x = \sum_{i=1}^n x_i y_i\) con cada $x_i \in I, y_i \in J$. Por tanto $x_iy_i \in I$ al ser $I$ un ideal, de hecho $\sum_{i=1}^n x_i y_i \in I$ al ser suma de elementos de $I$. Para $J$ ocurre igual ya que los $y_i \in J$, luego \(\sum_{i=1}^n x_i y_i \in J\) y finalmente \(x \in I \cap J\).

        \item Sea \( x \in I(J \cap K)\), entonces $x = \sum_{i=1}^n x_i y_i$ con cada $x_i \in I, y_i \in J \cap K$. En concreto $x \in IJ$ ya que cada $y_i \in J$, de igual manera $x \in IK$, por tanto \(x \in IJ \cap IK\).
        
        \item Sea \( x \in I(JK)\), entonces $x = \sum_{i=1}^n x_i y_i$ con cada $x_i \in I, y_i \in JK$, en concreto cada $y_i = \sum_{k=1}^m a_k b_k$ con $a_k \in J, b_k \in K$. Entonces
        \[
        x = \sum_{i=1}^n \left( x_i \left[ \sum_{k=1}^m a_k b_k \right]\right) = \sum_{k=1}^m \left( a_k b_k \left[ \sum_{i=1}^n x_i\right] \right) = \sum_{k=1}^m \left( b_k \left[ \sum_{i=1}^n x_i a_k \right] \right)
        \]
        sea $c_k = \sum_{i=1}^n x_i a_k$, entonces cada $c_k \in IJ$ ya que $x_i \in J, a_k \in J$, por tanto 
        \[
        x = \sum_{k=1}^m \left( b_k c_k \right) \in (IJ)K
        \]
        ya que cada $c_k \in IJ, b_k \in K$. Esto prueba que $I(JK) \subseteq (IJ)K$.

        El otro contenido es inmediato ya que $IJ = JI$, luego $(IJ)K = K(IJ) \subseteq (KI)J = J(KI) \subseteq (JK)I = I(JK)$ usando lo anterior.

        \item Sea \(x \in I(J + K)\), entonces
        \[
        x = \sum_{i=1}^n x_i y_i = \sum_{i=1}^n x_i (a_i + b_i)
        \]
        con $a_i \in J, b_i \in K$ ya que cada $y_i \in J + K$. Finalmente
        \[
        x = \sum_{i=1}^n x_i a_i + \sum_{i=1}^n x_i b_i \in IJ + IK
        \]
        ya que $\sum_{i=1}^n x_i a_i \in IJ, \sum_{i=1}^n x_i b_i \in IK$.

        Para la otra inclusión, si \(x \in IJ + IK\) entonces \(x = a + b\) con \(a \in IJ, b \in IK\), luego 
        \[
        a = \sum_{i=1}^n x_i a_i,\ b = \sum_{k=1}^m y_k b_k; \quad x_i, y_k \in I, a_i \in J, b_k \in K 
        \]
        pero notemos que $a_i = a_i + 0 \in J + K$ ya que $0 \in K$, de igual manera $b_k = 0 + b_k \in J + K$, luego definiendo
        \begin{align*}
        c_l =
            \begin{cases} 
                x_l, & 1 \leq l \leq n \\
                y_{n-l}, & n < l \leq n+m
            \end{cases} \\
        d_l = 
            \begin{cases} 
                a_l, & 1 \leq l \leq n \\
                b_{n-l}, & n < l \leq n+m
            \end{cases}
        \end{align*}
        tenemos que $c_l \in I, d_l \in J + K$ y 
        \[
        x = \sum_{l=1}^{n+m} c_l d_l
        \]
        luego \(x \in I(J + K)\).

        \item Por $1.$ tenemos que $IA \subseteq I \cap A = I$. Por otro lado, si $x \in I$ entonces definiendo $x_1 = x, a_1 = 1_A$ tenemos
        \[
        x = x 1_A = \sum_{i=1}^1 x_i a_i \implies x \in IA.
        \]
    \end{enumerate}
\end{proofbox}

\begin{example}{Operaciones en \(\mathbb{Z}\)}{}
    Sean \((n)\) y \((m)\) ideales de \(\mathbb{Z}\). Entonces:
    \begin{align*}
        (n)(m) &= (nm) \\
        (n) \cap (m) &= (\mathrm{mcm}(n, m)) \\
        (n) + (m) &= (\mathrm{mcd}(n, m))
    \end{align*}
\end{example}

\begin{example}{Ideal no principal}{ideal_no_principal}
    En \(\mathbb{Z}[X]\), el ideal \((2) + (X)\) (polinomios con término constante par) no es principal.
\end{example}

\begin{proofbox}
    Supongamos que \((2) + (X) = (a)\) para algún \(a \in \mathbb{Z}[X]\). Entonces:
    \begin{itemize}
        \item \(2 = aP\) para algún \(P \in \Z[X]\), pero entonces $P$ ha de ser un polinomio solo con término independiente y \(a\) también.
        \item Como $a \in (2) + (X)$, $a$ debe ser par. 
        \item Si suponemos \(X \in (a)\) debe ser $X = aP$ para algún \(P \in \Z[X]\), pero $aP$ tiene coeficiente par en la $X$ porque $a$ es par, luego $X \notin (a) = (2) + (X)$ lo cual es contradictorio.
    \end{itemize}
\end{proofbox}

\begin{example}{}{}
    Sea $A$ un anillo y sean $a,b \in A$, entonces $(a,b) = (a) + (b)$
\end{example}

\begin{proofbox}
    En efecto si $x \in (a,b)$ entonces
    \[
    x = a x_1 + b x_2 \implies x \in (a) + (b).
    \]
    De igual manera, si $x \in (a) + (b)$ entonces
    \[
    x = a_1 + b_1 
    \]
    con $a_1 \in (a) \implies a_1 = a x_1, b_1 \in (b) \implies b_1 = b x_2$, por tanto
    \[
    x = ax_1 + bx_2 \in (a,b).
    \]
\end{proofbox}

\clearpage
\section{Teoremas de isomorfía y Teorema chino de los restos}

\begin{theorem}{Primer teorema de isomorfía}{isomorfia_uno}
Sea $f : A \to B$ un homomorfismo de anillos. Entonces existe un único isomorfismo de anillos $\bar{f} : A/\ker f \to \Im f$ que hace conmutativo el diagrama
\[
\begin{tikzcd}
A \arrow[r, "f"] \arrow[d, "p"'] & B \\
A/\ker f \arrow[r, "\bar{f}"'] & \Im f \arrow[u, "i"]
\end{tikzcd}
\]
es decir, $i \circ \bar{f} \circ p = f$, donde $i$ es la inclusión y $p$ es la proyección. En particular
\[
\frac{A}{\ker f} \cong \Im f.
\]
\end{theorem}

\begin{proofbox}
Sean $K = \ker f$ e $I = \Im f$. La aplicación $\bar{f} : A/K \to I$ dada por $\bar{f}(x + K) = f(x)$ está bien definida (no depende de representantes) pues si $x + K = y + K$ entonces $x - y \in K$ y por lo tanto $f(x) - f(y) = f(x - y) = 0$; es decir, $f(x) = f(y)$. Además es elemental ver que es un homomorfismo de anillos y que es suprayectiva.

Para ver que es inyectiva usamos la Proposición \ref{prop:prop_nucleo_imagen}: si $x + K$ está en el núcleo de $\bar{f}$ entonces $0 = \bar{f}(x + K) = f(x)$, de modo que $x \in K$ y así $x + K = 0 + K$. Es decir $\ker \bar{f} = 0$ y por lo tanto $\bar{f}$ es inyectiva. En conclusión, $\bar{f}$ es un isomorfismo, y hace conmutativo el diagrama porque, para cada $x \in A$, se tiene
\[
i(\bar{f}(p(x))) = i(\bar{f}(x + K)) = i(f(x)_{\Im}) = f(x).
\]

En cuanto a la unicidad, supongamos que otro homomorfismo $\bar{f}' : A/K \to I$ verifica $i \circ \bar{f}' \circ p = f$; entonces para cada $x \in K$ se tiene 
\[
i(\bar{f}'(p(x))) = i(\bar{f}'(x + K)) = f(x) = i(\bar{f}(p(x))) = i(\bar{f}(x + K)).
\]
Por la inyectividad de $i$ debe ser entonces
\[
\bar{f}'(p(x)) = \bar{f}(p(x))
\]
como $p$ es sobreyectiva, dado $y + K \in A/\ker f$ existe $x_0 \in A$ tal que $y + K = p(x_0)$, por tanto para cualquier $y + K$
\[
\bar{f}'(y + K) = \bar{f}'(p(x_0)) = \bar{f}(p(x_0)) = \bar{f}(y + K)
\]
y finalmente $\bar{f}' = \bar{f}$.
\end{proofbox}

\begin{theorem}{Segundo teorema de isomorfía}{isomorfia_dos}
Sea $A$ un anillo y sean $I$ y $J$ dos ideales tales que $I \subseteq J$. Entonces $J/I$ es un ideal de $A/I$ y existe un isomorfismo de anillos
\[
\frac{A/I}{J/I} \cong \frac{A}{J}.
\]
\end{theorem}

\begin{proofbox}
Por el Teorema de la Correspondencia \ref{thm:teorema_correspondencia}, $J/I$ es un ideal de $A/I$. Sea $f : A/I \to A/J$ la aplicación definida por $f(a + I) = a + J$. Es elemental ver que $f$ está bien definida, sean $a + I = b + I \in A/I$, entonces $a-b \in I \subset J$, luego
\[
a - b \in J \implies a + J = b + J \implies f(a + I) = f(b + I).
\]
Que es un homomorfismo suprayectivo de anillos se deja como ejercicio. Veamos que $\ker f = J/I$
\[
f(a+I) = a + J = 0 + J \iff a \in J \implies a + I \in \eta(J) = J/I
\]
si por el contrario $a + I \in J/I$ entonces
\[
a + I = \eta(j) = j + I \implies f(a + I) = f(j + I) = j + J = 0 + J \implies a + I \in \ker f.
\]
Entonces el isomorfismo buscado se obtiene aplicando el Primer teorema de isomorfía a $f$.
\end{proofbox}

\begin{theorem}{Tercer teorema de isomorfía}{isomorfia_tres}
Sea $A$ un anillo con un subanillo $B$ y un ideal $I$. Entonces:
\begin{enumerate}
    \item $B \cap I$ es un ideal de $B$.
    \item $B + I$ es un subanillo de $A$ que contiene a $I$ como ideal.
    \item Se tiene un isomorfismo de anillos $\dfrac{B}{B \cap I} \cong \dfrac{B + I}{I}$.
\end{enumerate}
\end{theorem}

\begin{proofbox}
Los dos primeros apartados se dejan como ejercicio. En cuanto al último, sea $f : B \to A/I$ la composición de la inclusión $j : B \to A$ con la proyección $\eta : A \to A/I$
\[
f(b) = \eta(j(b)) = b + I.
\]
Es claro que $\ker f = B \cap I$, puesto que dado $b \in B$
\[
f(b) = b + I = 0 + I \iff b \in I \iff b \in B \cap I
\]
y que $\Im f = (B + I)/I$, ya que dado $b + I \in \Im f$ existe un cierto $b_0 \in B$ tal que
\[
b + I = f(b_0) = b_0 + I \iff b - b_0 \in I \iff b - b_0 = i \in I \implies b = b_0 + i \in B + I.
\]
Por tanto, el resultado se sigue del Primer Teorema de Isomorfía.
\end{proofbox}

\begin{example}{Aplicaciones del Primer Teorema de Isomorfía}{aplicaciones_isomorfia_uno}
\begin{enumerate}
    \item Si $A$ y $B$ son anillos, el homomorfismo $A \times B \to A$ de proyección en la primera componente es suprayectivo y tiene núcleo $I = 0 \times B$, por lo que $\frac{A \times B}{0 \times B} \simeq A$. En realidad ya habíamos visto esto en \ref{ex:cociente_productos}.
    
    \item Sea $n$ un entero positivo. Hemos visto que todo ideal de $\mathbb{Z}_n = \mathbb{Z}/ (n)$ es de la forma $(\overline{d}) = (d) / (n)$, para cierto divisor positivo $d$ de $n$. El Segundo Teorema de Isomorfía nos permite identificar el cociente $\mathbb{Z}_n/ (\overline{d})$, pues
    \[
    \frac{\mathbb{Z}_n}{(\overline{d})} = \frac{\mathbb{Z}/ (n)}{(d) / (n)} \simeq \frac{\mathbb{Z}}{ (d) } = \mathbb{Z}_d.
    \]
    
    \item Si $A$ es un anillo, el homomorfismo $f : A[X] \to A$ de sustitución en $0$ (dado por $a_0 + a_1 X + \cdots \mapsto a_0$) es suprayectivo y tiene por núcleo el ideal $(X)$ generado por $X$ (consistente en los polinomios con coeficiente independiente nulo), de modo que $A[X]/(X) \simeq A$, como ya habíamos observado en el Ejemplo \ref{ex:cociente_sustitucion}.
    
    \item Sean $A$ un anillo e $I$ un ideal de $A$. Para cada $a \in A$, sea $\overline{a} = a + I$. La aplicación $f : A[X] \to (A/I)[X]$ dada por $f(a_0 + a_1 X + \cdots + a_n X^n) = \overline{a}_0 + \overline{a}_1 X + \cdots + \overline{a}_n X^n$ es un homomorfismo suprayectivo de anillos cuyo núcleo es $I[X] = \{a_0 + a_1 X + \cdots + a_n X^n : a_i \in I\}$ (estas afirmaciones quedan como ejercicio para el lector). Del Primer Teorema de Isomorfía se deduce entonces que
    \[
    \frac{A[X]}{I[X]} \simeq (A/I)[X].
    \]
\end{enumerate}
\end{example}

\begin{definition}{Característica}{caracteristica}
Sea $A$ un anillo, y recordemos que si $n \in \mathbb{Z}^+$ escribimos $n1 = 1 + \cdots + 1$ ($n$ veces). Si existe $n \in \mathbb{Z}^+$ tal que $n1 = 0$, definimos la característica de $A$ como el menor $n \in \mathbb{Z}^+$ que verifica tal igualdad. Si no existe un tal $n$, decimos que la característica de $A$ es $0$.
\end{definition}

\begin{proposition}{Caracterización de la característica}{caracterizacion_caracteristica}
Sea $A$ un anillo y sea $f : \mathbb{Z} \to A$ el único homomorfismo de anillos (dado por $f(n) = n1$). Para un número natural $n$ las condiciones siguientes son equivalentes:
\begin{enumerate}
    \item $n$ es la característica de $A$.
    \item $n\mathbb{Z}$ es el núcleo de $f$.
    \item El subanillo primo de $A$ es isomorfo a $\mathbb{Z}_n$ (recuérdese que $\mathbb{Z}_0 = \mathbb{Z}$ y $\mathbb{Z}_1 = 0$).
    \item $A$ contiene un subanillo isomorfo a $\mathbb{Z}_n$.
\end{enumerate}
\end{proposition}

\begin{proofbox}
La equivalencia entre (1) y (2) se deja como ejercicio para el lector, y es obvio que (3) implica (4).

(2) implica (3). Se obtiene aplicando el Primer Teorema de Isomorfía y observando que $\Im f$ es el subanillo primo de $A$.

(4) implica (2). Si $B$ es un subanillo de $A$ y $g : \mathbb{Z}_n \to B$ es un isomorfismo, considerando la proyección $\pi : \mathbb{Z} \to \mathbb{Z}_n$ y la inclusión $u : B \hookrightarrow A$ se obtiene un homomorfismo de anillos $u \circ g \circ \pi : \mathbb{Z} \to A$ que debe coincidir con $f$ por su unicidad (Ejemplo 1.12.(3)). Como $u \circ g$ es inyectiva, es elemental ver que $\ker f = n\mathbb{Z}$.
\end{proofbox}

\begin{theorem}{Teorema Chino de los Restos}{chino_restos}
Sea $A$ un anillo y sean $I_1, \ldots, I_n$ ideales de $A$ tales que $I_i + I_j = A$ para todo $i \neq j$. Entonces $I_1 \cap \cdots \cap I_n = I_1 \cdots I_n$. Además
\[
\frac{A}{I_1 \cap \cdots \cap I_n} \cong \frac{A}{I_1} \times \cdots \times \frac{A}{I_n}.
\]
\end{theorem}

\begin{proofbox}
Razonamos por inducción sobre $n$, empezando con el caso $n = 2$. La hipótesis $I_1 + I_2 = A$ nos dice que existen $x_1 \in I_1$ y $x_2 \in I_2$ tales que $x_1 + x_2 = 1$, y entonces para cada $a \in I_1 \cap I_2$ se tiene $a = a x_1 + a x_2 \in I_1 I_2$, de modo que $I_1 \cap I_2 \subseteq I_1 I_2$, y la otra inclusión es clara. Claramente la aplicación $f : A \to A/I_1 \times A/I_2$ dada por $f(a) = (a + I_1, a + I_2)$ es un homomorfismo de anillos con núcleo $I_1 \cap I_2$, y es suprayectiva pues, dado un elemento arbitrario $(a_1 + I_1, a_2 + I_2)$ de $A/I_1 \times A/I_2$, el elemento $a = a_1 x_2 + a_2 x_1$ verifica $f(a) = (a_1 + I_1, a_2 + I_2)$. Ahora el resultado se obtiene aplicando el Primer Teorema de Isomorfía.

En el caso general ($n > 2$) basta ver que las hipótesis implican que $(I_1 \cap \cdots \cap I_{n-1}) + I_n = A$, pues entonces la hipótesis de inducción nos da
\[
I_1 \cap \cdots \cap I_{n-1} \cap I_n = (I_1 \cap \cdots \cap I_{n-1}) I_n = I_1 \cdots I_{n-1} I_n
\]
y
\[
\frac{A}{I_1 \cap \cdots \cap I_n} = \frac{A}{(\cap_{i=1}^{n-1} I_i) \cap I_n} \simeq \frac{A/(\cap_{i=1}^{n-1} I_i)}{I_n} \times \frac{A}{I_n} \simeq \frac{A}{I_1} \times \cdots \times \frac{A}{I_{n-1}} \times \frac{A}{I_n}.
\]

Para ver que $(I_1 \cap \cdots \cap I_{n-1}) + I_n = A$ notemos que, para cada $i \leq n-1$, existen $a_i \in I_i$ y $b_i \in I_n$ tales que $1 = a_i + b_i$, y multiplicando todas esas expresiones se obtiene
\[
1 = \prod_{i=1}^{n-1} (a_i + b_i) = a_1 \cdots a_{n-1} + b,
\]
donde $b$ engloba a todos los sumandos que se obtendrían desarrollando los productos (excepto $a_1 \cdots a_{n-1}$) y está en $I_n$ porque en cada sumando hay al menos un factor del ideal $I_n$. Como además $a_1 \cdots a_{n-1} \in I_1 \cap \cdots \cap I_{n-1}$, deducimos que $1 \in (I_1 \cap \cdots \cap I_{n-1}) + I_n$ y así $(I_1 \cap \cdots \cap I_{n-1}) + I_n = A$, como queríamos ver.
\end{proofbox}