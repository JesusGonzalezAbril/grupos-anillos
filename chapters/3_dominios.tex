\clearpage

\chapter{Divisibilidad en dominios}

\section{Cuerpos y dominios}

\section{Ideales primos y maximales}

\begin{definition}{Ideal primo}{ideal-primo}
    Un ideal propio \(P \leq A, P \neq A\) es primo si para todo \(a, b \in A\):
    \[
    ab \in P \Rightarrow a \in P \text{ o } b \in P
    \]
\end{definition}

\begin{definition}{Ideal maximal}{ideal-maximal}
    Un ideal propio \(M \leq A, M \neq A\) es maximal si no existe ningún ideal \(I\) tal que \(M \subsetneq I \subsetneq A\).
\end{definition}

\begin{lemma}{Caracterizaciones}{caract-primos-maximales}
    \begin{enumerate}
        \item \(P\) es primo si y solo si \(A/P\) es un dominio de integridad
        \item \(M\) es maximal si y solo si \(A/M\) es un cuerpo
        \item Todo ideal maximal es primo
    \end{enumerate}
\end{lemma}

\begin{example}{Ejemplos en \(\mathbb{Z}\)}{}
    \begin{itemize}
        \item Los ideales primos de \(\mathbb{Z}\) son \((0)\) y \((p)\) con \(p\) primo
        \item Los ideales maximales de \(\mathbb{Z}\) son \((p)\) con \(p\) primo
    \end{itemize}
\end{example}
