\clearpage

\chapter{Divisibilidad en dominios}

\section{Cuerpos y dominios}

\begin{definition}{}{}
Un elemento \(a\) de un anillo \(A\) se dice regular si la relación \(ab = ac\) con \(b, c \in A\) implica que \(b = c\); es decir, si \(a\) es cancelable respecto del producto. Claramente, el 0 nunca es regular\footnote{Obsérvese la importancia de la hipótesis \(1 \neq 0\) en este caso.}.

Un cuerpo es un anillo en el que todos los elementos no nulos son invertibles, y un dominio (o dominio de integridad) es un anillo en el que todos los elementos no nulos son regulares.

Un subanillo de un anillo \(A\) que sea un cuerpo se llama un subcuerpo de \(A\), y un homomorfismo de anillos entre dos cuerpos se llama homomorfismo de cuerpos.
\end{definition}

\begin{remark}
    Si $A$ es un dominio y $0 \neq a,b \in A$, luego $a$ y $b$ son regulares. Supongamos que $ab = 0$, entonces
    \[
    ab = 0 = a0 \implies b = 0
    \]
    ya que $a$ es cancelable, pero esto es una contradicción, luego no puede cumplirse $ab = 0$ si $a,b \neq 0$.
    
    En otras palabras 
    \[
    a,b \neq 0 \implies ab \neq 0 
    \]
    el contrarrecíproco de esta afirmación es
    \[
    ab = 0 \implies a = 0 \text{ ó } b = 0
    \]
\end{remark}

\begin{proposition}{}{cuerpos-dominios}
Todo cuerpo es un dominio.
\end{proposition}

\begin{proofbox}
Si \(A\) es un cuerpo y \(a \in A\), \(a \neq 0\), entonces \(a\) es invertible. Si \(ab = ac\), multiplicando por \(a^{-1}\) obtenemos \(b = c\), luego \(a\) es regular. Como esto vale para todo \(a \neq 0\), \(A\) es un dominio.
\end{proofbox}

\begin{proposition}{}{}
Sea \(A\) un anillo.
\begin{enumerate}
\item Las condiciones siguientes son equivalentes:

\begin{enumerate}
    \item \(A\) es un cuerpo.
    \item Los únicos ideales de \(A\) son \(0\) y \(A\).
    \item Todo homomorfismo de anillos \(A \to B\) con \(B \neq 0\) es inyectivo.
\end{enumerate}

\item Un elemento \(a \in A\) es regular si y solo si la relación \(ab = 0\) con \(b \in A\) implica \(b = 0\) (por este motivo, los elementos no regulares se suelen llamar divisores de cero).

\item \(A\) es un dominio si y solo si, para cualesquiera \(a, b \in A\) no nulos, se tiene \(ab \neq 0\).

\item Todo subanillo de un dominio es un dominio.

\item La característica de un dominio es cero o un número primo.
\end{enumerate}
\end{proposition}

\begin{proofbox}

\begin{enumerate}
\item Demostramos las equivalencias.

(a) $\Rightarrow$ (b) Si $A$ es cuerpo e $I$ es un ideal no nulo de $A$, entonces $I$ tiene un elemento $a \neq 0$. Como $A$ es cuerpo, $a$ es invertible, luego $I = A$.

(b) $\Rightarrow$ (c) Si $f: A \to B$ es un homomorfismo con $B \neq 0$, entonces $\ker f$ es un ideal pero $\ker f \neq A$, pues $f(1) = 1 \neq 0$. Entonces, por (b), $\ker f = 0$, luego $f$ es inyectivo.

(c) $\Rightarrow$ (a) Haremos el contrarrecírpoco. Si $A$ no es cuerpo, existe $a \neq 0$ no invertible. Entonces $(a)$ es un ideal propio no nulo, y el homomorfismo canónico
\[
\pi: A \to A/(a),\quad \pi(x) = x + (a)
\]
no es inyectivo ya que
\[
a \in \ker\pi \neq 0.
\]

\item Si $a$ es regular y $ab = 0$, entonces $ab = 0 = a0$, luego $b = 0$. Recíprocamente, si $a$ no es regular, existen $b \neq c$ con $ab = ac$, luego $a(b-c) = 0$ con $b-c \neq 0$.

\item Es consecuencia inmediata de (2).

\item Si $B$ es subanillo de un dominio $A$ y $x, y \in B$ son no nulos, entonces $xy \neq 0$ en $A$, luego también en $B$ ya que su cero es el mismo que el de $A$.

\item Sea $D$ un dominio y consideremos el homomorfismo $f: \mathbb{Z} \to D$ dado por $f(n) = n\cdot 1$. Como $\ker f$ es un ideal de $\mathbb{Z}$, existe $n \geq 0$ tal que $\ker f = (n)$. Si $n = ab$ con $0 < a, b < n$, entonces $f(a)f(b) = f(ab) = 0$, luego $f(a) = 0$ o $f(b) = 0$, contradicción. Así que $n$ es primo o $n = 0$.
\end{enumerate}
\end{proofbox}

\begin{example}{Dominios y cuerpos}{}

\begin{enumerate}
\item Los anillos \(\mathbb{Q}\), \(\mathbb{R}\) y \(\mathbb{C}\) son cuerpos y \(\mathbb{Z}\) es un dominio que no es un cuerpo (aunque es subanillo de un cuerpo).

\item Para \(n \geq 2\), el anillo \(\mathbb{Z}_n\) es un dominio si y solo si es un cuerpo, si y sólo si \(n\) es primo.

\begin{proofbox}
Si $n$ es primo y $\overline{a} \neq 0$ en $\mathbb{Z}_n$, entonces $\mathrm{mcd}(a,n) = 1$, luego existen $x,y$ con $ax + ny = 1$, así que $\overline{a}\overline{x} = \overline{1}$. Recíprocamente, si $n$ no es primo, existen $a,b$ con $1 < a,b < n$ y $n = ab$, luego $\overline{a}\overline{b} = \overline{0}$ con $\overline{a},\overline{b} \neq 0$.
\end{proofbox}

\item Si \(m\) es un entero que no es el cuadrado de un número entero entonces \(\mathbb{Z}[\sqrt{m}]\) es un dominio (subanillo de \(\mathbb{C}\)) que no es un cuerpo (el \(2\) no tiene inverso). Sin embargo, \(\mathbb{Q}[\sqrt{m}]\) sí que es un cuerpo; de hecho, si \(a + b\sqrt{m} \neq 0\), entonces \(q = (a + b\sqrt{m})(a - b\sqrt{m}) = a^2 - b^2 m\) es un número racional no nulo y \((a + b\sqrt{m})^{-1} = \frac{a}{q} - \frac{b}{q}\sqrt{m}\).

\item Un producto de anillos diferentes de \(0\) nunca es un dominio, pues \((1,0)(0,1) = (0,0)\).

\item Los anillos de polinomios no son cuerpos, pues la indeterminada genera un ideal propio y no nulo. Por otra parte, \(A[X]\) es un dominio si y solo si lo es \(A\).

\begin{proofbox}
Si $A$ es dominio y $P,Q \in A[X]$ son no nulos, sean $a_n X^n$ y $b_m X^m$ sus términos de mayor grado. Entonces el coeficiente de $X^{n+m}$ en $PQ$ es $a_n b_m \neq 0$, luego $PQ \neq 0$. El recíproco es claro pues $A$ es subanillo de $A[X]$.
\end{proofbox}

\end{enumerate}
\end{example}

\clearpage
\section{Ideales primos y maximales}

\begin{definition}{Ideal primo}{ideal-primo}
    Un ideal propio \(P \leq A, P \neq A\) es primo si para todo \(a, b \in A\):
    \[
    ab \in P \Rightarrow a \in P \text{ o } b \in P
    \]
\end{definition}

\begin{definition}{Ideal maximal}{ideal-maximal}
    Un ideal propio \(M \leq A, M \neq A\) es maximal si no existe ningún ideal \(I\) tal que \(M \subsetneq I \subsetneq A\).
\end{definition}

\begin{proposition}{Caracterizaciones de ideales maximales y primos}{caracterizacion-maximales-primos}
Sean \(A\) un anillo e \(I\) un ideal propio de \(A\). Entonces:

\begin{enumerate}
\item \(I\) es maximal si y solo si \(A/I\) es un cuerpo.

\item \(I\) es primo si y solo si \(A/I\) es un dominio.

\item Si \(I\) es maximal entonces es primo.

\item \(A\) es un cuerpo si y solo si el ideal \(0\) es maximal.

\item \(A\) es un dominio si y solo si el ideal \(0\) es primo.
\end{enumerate}

\end{proposition}

\begin{proofbox}

\begin{enumerate}
\item Por el teorema de correspondencia, los ideales de $A/I$ corresponden a los ideales de $A$ que contienen a $I$.

Así, si $I$ es maximal entonces el único ideal distinto de $I$ que contiene a $I$ es $A$. Pero entonces, dado un ideal $J/I \leq A/I$ este ha de corresponder a $A/I$ o a $I/I = 0$, luego los únicos ideales de $A/I$ son $0$ y $A/I$, lo cual es una de las caracterizaciones para que un anillo sea un cuerpo. 

De igual manera, si $A/I$ es un cuerpo, entonces los únicos ideales son $0$ y $A/I$, pero entonces los únicos ideales que contienen a $I$ son $I,A$, es decir, $I$ es maximal.

\item Si $I$ es primo y tomamos $(a + I)(b + I) = ab + I = 0 + I$ entonces
\[
ab \in I \implies a \in I \text{ ó } b \in I \implies a + I = 0 + I \text{ ó } b + I = 0 + I
\]
luego $A/I$ es dominio.

Por el contrario, si $A/I$ es dominio entonces dados $a,b$ tales que $ab \in I$
\[
0 + I = ab + I = (a + I)(b + I) \iff a + I = 0 + I \text{ ó } b + I = 0 + I \iff a \in I \text{ ó } b \in I 
\]
es decir, si $ab \in I$ entonces $a \in I$ o $b \in I$.

\item Se sigue de (1) y (2) ya que
\[
I \text{ maximal} \iff A/I \text{ cuerpo} \implies A/I \text{ dominio} \iff I \text{ primo}.
\]

\item Es inmediato aplicando (1) ya que $A \cong A/(0)$.

\item Es inmediato aplicando (2) ya que $A \cong A/(0)$
\end{enumerate}

\end{proofbox}

\begin{remark}
    La parte 3. de la proposición anterior se puede probar directamente. 
    \begin{proofbox}
    Supongamos que $I$ es maximal pero no primo. Entonces existen $a,b \in A$ tales que $ab \in I$ pero $a,b \notin I$. Consideremos entonces el siguiente ideal
    \[
    I + (a) = \{x + ay : x \in I, y \in A\}.
    \]
    Claramente $I \subseteq I + (a) \subseteq A$, y claramente $I \neq I + (a)$ ya que $a = 0 + a1 \in I + (a), a \notin I$. Pero también tenemos $I + (a) \neq A$ ya que si no fuera así entonces existirían $x_0 \in I, y_0 \in A$ tales que
    \[
    x_0 + ay_0 = 1 \implies bx_0 + aby_0 = b \implies b \in I
    \]
    ya que $bx_0, aby_0 \in I$, lo cual es contradictorio.
    
    Luego $I + (a)$ es un ideal propio que contiene a $I$, pero eso contradice la maximalidad de $I$, por tanto $I$ debe ser primo.
    \end{proofbox}
\end{remark}


\begin{example}{Ejemplos en \(\mathbb{Z}\)}{}
    \begin{itemize}
        \item Los ideales primos de \(\mathbb{Z}\) son \((0)\) y \((p)\) con \(p\) primo
        \item Los ideales maximales de \(\mathbb{Z}\) son \((p)\) con \(p\) primo
    \end{itemize}
\end{example}

\begin{proofbox}

\end{proofbox}

\begin{example}{}{}
De la Proposición 1.15 sabemos que todos los ideales de \(\mathbb{Z}\) son principales. Además si \(n\) y \(m\) son enteros entonces \((n) \subseteq (m)\) si y solo si \(m\) divide a \(n\). Por tanto, \((n)\) es un ideal maximal de \(\mathbb{Z}\) si y solo si \(n \not\in \{0,1,-1\}\) y los únicos divisores de \(n\) son \(\pm 1\) y \(\pm n\), o sea si \(n\) es un número primo. En tal caso \((n)\) es ideal primo de \(\mathbb{Z}\) por la Proposición 2.6.(3). Observese que \((0)\) es un ideal primo de \(\mathbb{Z}\) que no es maximal pues \(\mathbb{Z}\) es un dominio que no es un cuerpo. Sin embargo si \(n \neq 0\) y \(n\) no es primo entonces \((n)\) no es un ideal primo de \(\mathbb{Z}\), pues o bien \(n = \pm 1\) en cuyo caso \((n) = \mathbb{Z}\) o bien \(n = ab\) con \(a\) y \(b\) dos divisores propios de \(\mathbb{Z}\), con lo que \(ab \in (n)\) pero ni \(a\) ni \(b\) están en \((n)\).

En resumen, los ideales maximales de \(\mathbb{Z}\) son los de la forma \((n)\) con \(n\) un número primo y los ideales primos \(\mathbb{Z}\) son los de la forma \((n)\) con \(n = 0\) o un número primo.
\end{example}

\begin{proposition}
Todo ideal propio de un anillo está contenido en un ideal maximal.
\end{proposition}

\begin{proof}
Sea \(I\) un ideal propio de \(A\) y sea \(\Omega\) el conjunto de los ideales propios de \(A\) que contienen a \(I\). Observese que la unión de una cadena \(I_1 \subseteq I_2 \subseteq I_3 \subseteq \ldots\) de elementos de \(\Omega\) es un ideal, que además es propio, pues si no lo fuera, contendría a \(1\) y por tanto algún \(I_n\) contendría a \(1\) en contra de que todos los \(I_n\) son ideales propios. Aplicando el Lema de Zorn deducimos que \(\Omega\) tiene un elemento maximal que obviamente es un ideal maximal de \(A\).
\end{proof}

\begin{remark}
El uso del Lema de Zorn en la demostración anterior implica que este resultado depende del Axioma de Elección. En anillos noetherianos (como \(\mathbb{Z}\) o \(K[X]\) con \(K\) cuerpo) se puede demostrar sin el Axioma de Elección.
\end{remark}
