\clearpage
\chapter{Polinomios}

\section{Anillos de polinomios}

\emph{En el resto del capítulo usaremos la siguiente notacion: \(\mathbb{N}_{0} = \mathbb{N}\cup \{0\}\).}

Sea \(A\) un anillo. En el capítulo 1 definimos el anillo de polinomios \(A[X]\) en una indeterminada con coeficientes en \(A\) como el conjunto de las expresiones del tipo
\[
P=P(X)=p_{0}+p_{1}X+p_{2}X^{2}+\cdots+p_{n}X^{n}
\]
donde \(n\) es un numero entero no negativo y \(p_{i}\in A\) para todo \(i\).

Los elementos \(p_{0},p_{1},p_{2},\ldots\) se llaman coeficientes de \(P\). Más precisamente, \(p_{i}X^{i}\) se llama monomio de grado \(i\) del polinomio \(P\) y \(p_{i}\) se llama coeficiente del monomio de grado \(i\) de \(P\). Observese que \(P\) tiene infinitos coeficientes, aunque todos menos un numero finito son iguales a \(0\). Dos polinomios son iguales si sus coeficientes de los monomios del mismo grado son iguales. El polinomio cero o polinomio nulo es el polinomio que tiene todos los coeficientes iguales a \(0\).

La suma y el producto en \(A[X]\) se definen
\[
(a_{0}+a_{1}X+a_{2}X^{2}+\cdots)+(b_{0}+b_{1}X+b_{2}X^{2}+\cdots)=c_{0}+c_{1}X+c _{2}X^{2}+\cdots,
\]
donde cada \(c_{n}=a_{n}+b_{n}\), y
\[
(a_{0}+a_{1}X+a_{2}X^{2}+\cdots)\cdot(b_{0}+b_{1}X+b_{2}X^{2}+\cdots)=d_{0}+d_{1 }X+d_{2}X^{2}+\cdots,
\]
donde cada \(d_{n}=a_{0}b_{n}+a_{1}b_{n-1}+\cdots+a_{n-1}b_{1}+a_{n}b_{0}=\sum_{i=0}^{n}a_{i} b_{n-i}\).

Sea \(A\) un anillo y sea \(p=\sum_{i\in\mathbb{N}_{0}}p_{i}X^{i}\in A[X]\) un polinomio no nulo de \(A[X]\). Entonces, por definicion de polinomio, el conjunto \(\{i\in\mathbb{N}_{0}:p_{i}\neq 0\}\) no es vacio y esta acotado superiormente. Por tanto ese conjunto tiene un maximo, al que llamamos grado del polinomio \(P\) y denotamos por \(\gr(p)\). Es decir,
\[
\gr(p)=\max\{i\in\mathbb{N}_{0}:p_{i}\neq 0\}.
\]

El coeficiente de mayor grado, \(p_{\textrm{gr}(p)}\), se conoce como el coeficiente principal de \(P\), y diremos que \(P\) es monico si su coeficiente principal es \(1\). Por convenio, consideramos que el polinomio \(0\) tiene grado \(-\infty\) y coeficiente principal \(0\). Es claro que los polinomios de grado \(0\) son precisamente los polinomios constantes no nulos. A veces llamaremos lineales a los polinomios de grado \(1\), cuadraticos a los de grado \(2\), cubicos a los de grado \(3\), etcetera.

\begin{lemma}{}{grados_polinomios}
Si $P$ y $Q$ son polinomios no nulos de $A[X]$ y sus términos principales son $p$ y $q$ respectivamente entonces se verifican las siguientes propiedades:

\begin{enumerate}
    \item \(\gr(P + Q) \leq \max(\gr(P), \gr(Q))\), con la desigualdad estricta si y solo si \(\gr(P) = \gr(Q)\) y \(p + q = 0\).
    \item \(\gr(PQ) \leq \gr(P) + \gr(Q)\), con igualdad si y solo si \(pq \neq 0\).
    \item Si \(p\) es regular (por ejemplo, si \(P\) es mónico, o si \(A\) es un dominio), entonces se tiene \(\gr(PQ) = \gr(P) + \gr(Q)\).
    \item Las desigualdades de los apartados 1 y 2 pueden ser estrictas.
\end{enumerate}

\end{lemma}

\begin{proofbox}

Supongamos que los coeficientes de $P$ son \(p_0, p_1, \dots, p_n = p, 0, \dots\) y los de $Q$ son $q_0, q_1, \dots, q_m = q, 0, \dots$, luego $\gr(P) = n, \gr(Q) = m$. Podemos suponer además sin perder generalidad que $n \geq m$.

\begin{enumerate}
    \item Por la definición de grado
    \[
    \gr(P + Q) = \max \{i \in \N_0 : p_i + g_i \neq 0\}.
    \]
    Sabemos que si $i > n \geq m$ entonces $p_i = q_i = 0 \implies p_i + q_i = 0$, luego
    \[
    \gr(P + Q) \leq n = \gr(P) = \max(\gr(P), \gr(Q)).
    \] 
    Además, la desigualdad es estricta si y solo si 
    \[
    p_n + q_n = 0 \iff p_n = -q_n 
    \]
    y como $p_n = p \neq 0$ debe ser $m = n$ y $q = -p$.
    \item Por definición
    \[
    \gr(PQ) = \max \{i \in \N_0 : \sum_{k=0}^{i} p_kq_{i - k} \neq 0\}.
    \]
    Sea $i > n + m$ entonces dado $0 \leq k \leq i$ tenemos \(k > n \implies p_k = 0\) por lo que 
    \[
    \sum_{k=0}^{i} p_kq_{i - k} = \sum_{k=0}^{n} p_kq_{i - k}
    \]
    pero $i - k > m \implies q_{i-k} = 0$ y esta condición es equivalente a
    \[
    i - k \leq m \iff k \geq i - m > n
    \]
    es decir, si $k > n$ los $q_{i-k} = 0$, por tanto nos queda
    \[
    \sum_{k=0}^{n} p_kq_{i - k} = 0
    \]
    esto prueba que 
    \[
    \gr(PQ) \leq n + m = \gr(P) + \gr(Q).
    \]
    La igualdad se da si y solo si
    \[
    0 \neq \sum_{k=0}^{n + m} p_kq_{(n+m) - k} = \sum_{k=n}^{n} p_k q_{(n+m) - k} = p_n q_m 
    \]
    es decir, si y solo si \(pq \neq 0\).
    \item Supongamos que \(p\) es regular y que \(pq = 0\), entonces, como $p$ es regular \(q = 0\), pero esto es imposible pues \(q\) es el término principal de \(Q\). Por tanto
    \[
    pq \neq 0 \implies \gr(PQ) = \gr(P) + \gr(Q)
    \]
    por (2).
    \item Para el apartado (1) consideremos los polinomios
    \[
    P = X + 1, Q = -X,\quad P,Q \in \Z[X],
    \]
    claramente $\gr(P) = \gr(Q) = 1$ pero \(P + Q = 1\), luego
    \[
    \gr(P + Q) = 0 < \max(\gr(P), \gr(Q)) = 1.
    \]

    En cuanto al apartado (2), sean
    \[
    P = 2, Q = 2X + 1,\quad P,Q \in \Z_4[X],
    \]
    entonces $\gr(P) = 0, \gr(Q) = 1$, pero $PQ = 4X + 2 = 2$, luego 
    \[
    \gr(PQ) = 0 < \gr(P) + \gr(Q) = 1.
    \]
\end{enumerate}

\end{proofbox}

Una consecuencia inmediata del Lema \ref{lem:grados_polinomios} es:

\begin{corollary}{}{unidades_dominio_polinomios}
Un anillo de polinomios \(A[X]\) es un dominio si y solo si lo es el anillo de coeficientes \(A\). En este caso se tiene \(A[X]^* = A^*\), es decir, los polinomios invertibles de \(A[X]\) son los polinomios constantes invertibles en \(A\). En particular, los polinomios invertibles sobre un cuerpo son exactamente los de grado 0, y \(A[X]\) nunca es un cuerpo.
\end{corollary}

\begin{proofbox}
Supongamos que \(A[X]\) es un dominio, entonces
\[
PQ = 0 \implies P = 0 ~\text{ o }~ Q = 0.
\]
Sean $p,q \in A$ y supongamos que $pq = 0$, sean $P = p, Q = q$ polinomios en $A[X]$ con grado 0, que claramente cumplen $PQ = pq = 0$. Como \(A[X]\) es un dominio debe ser entonces $P = 0$ o $Q = 0$, es decir, $p = 0$ o $q = 0$, luego \(A\) también es un dominio.

Para el recíproco, supongamos que \(A\) es un dominio. Sean $P,Q \in A[X]$ tales que $PQ = 0$, luego $\gr(PQ) = -\infty$. Por la parte (3) del Lema \ref{lem:grados_polinomios} sabemos que
\[
-\infty = \gr(PQ) = \gr(P) + \gr(Q)
\]
luego o bien \(\gr(P) = -\infty\) o \(\gr(Q) = - \infty\), es decir, $P = 0$ o $Q = 0$. Esto prueba que $A[X]$ es un dominio. 

En cuanto a las unidades, es claro que $A^* \subseteq A[X]^*$ (abusamos del lenguaje identificando $A$ como subanillo de $A[X]$). Sea ahora $P \in A[X]^*$, entonces existe $Q$ tal que
\[
PQ = 1 \implies 0 = \gr(PQ) = \gr(P) + \gr(Q) \implies \gr(P) = \gr(Q) = 0
\]
es decir, $P, Q \in A \setminus \{0\}$, pero entonces $P$ es una unidad en $A$, luego $P \in A^*$ como queríamos ver.
\end{proofbox}

Hemos observado que un anillo \(A\) es un subanillo del anillo de polinomios \(A[X]\), y por tanto la inclusión \(u : A \to A[X]\) es un homomorfismo de anillos. También es claro que el subanillo de \(A[X]\) generado por \(A\) y \(X\) es todo \(A[X]\). Es decir, la indeterminada \(X\) y las constantes de \(A\) (las imágenes de \(u\)) generan todos los elementos de \(A[X]\).

\begin{proofbox}
Ya sabemos que $(A \cup \{X\}) \subseteq A[X]$. Sea $P \in A[X]$, entonces
\[
P = p_0 + p_1 X + \dots + p_n X^n, \quad p_n \neq 0
\]
agrupando términos
\[
P = p_0 + X(p_1 + \dots + p_n X^{n-1}) \in (A \cup \{X\})
\]
luego $A[X] \subseteq (A \cup \{X\})$ como queríamos ver.
\end{proofbox}

El siguiente resultado nos dice que \(A[X]\) puede caracterizarse por una propiedad en la que solo intervienen \(X\) y \(u\).

\begin{proposition}{Propiedad Universal del Anillo de Polinomios, PUAP}{}
Sean \(A\) un anillo, \(A[X]\) el anillo de polinomios con coeficientes en \(A\) en la indeterminada \(X\) y \(u : A \to A[X]\) el homomorfismo de inclusión.

\begin{enumerate}
    \item Para todo homomorfismo de anillos \(f : A \to B\) y todo elemento \(b\) de \(B\) existe un único homomorfismo de anillos \(\overline{f} : A[X] \to B\) tal que \(\overline{f}(X) = b\) y \(\overline{f} \circ u = f\). Para expresar la última igualdad dice que \(\overline{f}\) completa de modo único el diagrama
    
    \[
    \begin{tikzcd}
    A \arrow[r, "u"] \arrow[dr, "f"'] & A[X] \arrow[d, "\overline{f}"] \\
    & B
    \end{tikzcd}
    \]
    
    \item Si dos homomorfismos de anillos \(g, h : A[X] \to B\) coinciden sobre \(A\) y en \(X\) entonces son iguales. Es decir, si \(g \circ u = h \circ u\) y \(g(X) = h(X)\) entonces \(g = h\).
    \item \(A[X]\) y \(u\) están determinados salvo isomorfismos por la PUAP. Explicitamente: supongamos que existen un homomorfismo de anillos \(v:A\to P\) y un elemento \(T\in P\) tales que, para todo homomorfismo de anillos \(f:A\to B\) y todo elemento \(b\in B\), existe un único homomorfismo de anillos \(\overline{f}:P\to B\) tal que \(\overline{f}\circ v=f\) y \(\overline{f}(T)=b\). Entonces existe un isomorfismo \(\phi:A[X]\to P\) tal que \(\phi\circ u=v\) y \(\phi(X)=T\).
\end{enumerate}
\end{proposition}

\begin{proofbox}
\begin{enumerate}
\item Sean \(f:A\to B\) y \(b\in B\) como en el enunciado. Si existe un homomorfismo \(\overline{f}:A[X]\to B\) tal que \(\overline{f}\circ u=f\) y \(\overline{f}(X)=b\), entonces para un polinomio \(P=\sum_{n=0}^{m} p_{n}X^{n}\), se tendra

\[\overline{f}(P)=\overline{f}\left(\sum_{n=0}^{m} u(p_{n})X^{n}\right)=\sum_{n=0}^{m}f(p_{n})b^{n}.\]

Por tanto, la aplicacion dada por \(\overline{f}(P)=\sum_{n=0}^{m}f(p_{n})b^{n}\) es la unica que puede cumplir tales condiciones.

Veamos que es un homomorfismo de anillos, dados \(P,Q \in A[X]\)
\[
\overline{f}(P + Q) = \sum_{n \geq 0}f(c_{n})b^{n} = \sum_{n \geq 0}f(p_{n} + q_n)b^{n} = \sum_{n \geq 0}f(p_{n})b^{n} + \sum_{n \geq 0}f(q_{n})b^{n} = \overline{f}(P) + \overline{f}(Q)
\]
\[
\overline{f}(PQ) = \sum_{n \geq 0}f(d_{n})b^{n} = \sum_{n \geq 0} f(\sum_{i=0}^n p_{i} q_{n-i})b^n = \sum_{n \geq 0} \left(\sum_{i=0}^n f(p_{i}) f(q_{n-i})\right)b^n = \overline{f}(P)\overline{f}(Q)
\]
\[
\overline{f}(1) = f(1)b^0 = f(1) = 1. 
\]
Además, es elemental ver que satisface \(\overline{f}(X)=b\) y \(\overline{f}\circ u=f\).

\item Si ponemos \(f=g\circ u=h\circ u:A\to B\), los homomorfismos \(g\) y \(h\) completan el diagrama del apartado (1). Por la unicidad se tiene \(g=h\).

\item Sean \(v:A\to P\) y \(T\in P\) como en (3). Aplicando (1) y la hipotesis de (3) deducimos que existen homomorfismos \(\overline{v}:A[X]\to P\) y \(\overline{u}:P\to A[X]\) tales que que se verifican las siguientes igualdades:
\[
\overline{v}\circ u=v,\quad\overline{v}(X)=T,\qquad\overline{u}\circ v=u,\quad \overline{u}(T)=X.
\]
Entonces la composicion \(\overline{u}\circ\overline{v}:A[X]\to A[X]\) verifica
\[
(\overline{u}\circ\overline{v})\circ u=\overline{u}\circ v=u\qquad\text{y} \qquad(\overline{u}\circ\overline{v})(X)=\overline{u}(T)=X,
\]
y por (2) se obtiene \(\overline{u}\circ\overline{v} = 1_{A[X]}\). De modo analogo, y observando que \(v\) y \(T\) verifican una condicion similar a (2), se demuestra que \(\overline{v}\circ\overline{u}=1_{P}\), con lo que \(\overline{v}\) es el isomorfismo que buscamos.
\end{enumerate}
\end{proofbox}

La utilidad de la PUAP estriba en que, dado un homomorfismo \(f:A\to B\), nos permite crear un homomorfismo \(A[X]\to B\) que "respeta" a \(f\) y que "se comporta bien" sobre un elemento \(b\in B\) que nos interese. Los siguientes ejemplos son aplicaciones de la PUAP a ciertos homomorfismos que aparecen con frecuencia y son importantes tanto en este capitulo como en algunos de los siguientes (y en otras muchas situaciones que no estudiaremos aqui).

\begin{example}{Aplicaciones de la PUAP (1)}{}
Sean \(A\) un subanillo de \(B\) y \(b\in B\). Aplicando la PUAP a la inclusion \(A\hookrightarrow B\) obtenemos un homomorfismo \(S_{b}:A[X]\to B\) que es la identidad sobre \(A\) (decimos a veces que fija los elementos de \(A\)) y tal que \(S_{b}(X)=b\).

Se le llama el homomorfismo de sustitución (o de evaluación) en \(b\). Dado \(P\in A[X]\), escribiremos a menudo \(P(b)\) en vez de \(S_{b}(X)\). Podemos describir explicitamente la accion de \(S_{b}\) en un polinomio:
\[
P(X)=\sum_{n\geq 0}^{A[X]}p_{n}X^{n} \leadsto S_{b}(P)=P(b)=\sum_{n \geq 0}p_{n}b^{n}.
\]
\end{example}

\begin{example}{Aplicaciones de la PUAP (2)}{}
Sean \(A\) un anillo y \(a \in A\). Si en el ejemplo anterior tomamos \(B = A[X]\) y \(b = X + a\), obtenemos un homomorfismo \(\phi: A[X] \to A[X]\) dado por \[ p(X) \mapsto p(X + a). \] Este homomorfismo es un automorfismo cuyo inverso \(\psi\) viene dado por \(p(X) \mapsto p(X - a)\). En efecto
\[
\phi(\psi(p(X))) = \phi(p(X - a)) = p((X + a) - a) = p(x)
\]
\[
\psi(\phi(p(X))) = \psi(p(X + a)) = p((X - a) + a) = p(x).
\]
\end{example}    

\begin{example}{Aplicaciones de la PUAP (3)}{}
Todo homomorfismo de anillos \(f : A \to B\) induce un homomorfismo entre los correspondientes anillos de polinomios.

Aplicándole la PUAP a la composición de \(f\) con la inclusión \(B \hookrightarrow B[X]\) obtenemos \(\overline{f} : A[X] \to B[X]\) tal que \(\overline{f} |_{A} = f\) y \(\overline{f}(X) = X\). Explicitamente,
\[
\overline{f} \left( \sum_{n \geq 0} p_n X^n \right) = \sum_{n \geq 0} f(p_n) X^n.
\]
Es fácil ver que si \(f\) es inyectivo o suprayectivo, entonces lo es \(\overline{f}\); como casos particulares de esta afirmación se obtienen los dos ejemplos siguientes.
\end{example}

\begin{example}{Aplicaciones de la PUAP (4)}{}
Si \(A\) es un subanillo de \(B\) entonces \(A[X]\) es un subanillo de \(B[X]\).
\end{example}

\begin{example}{Aplicaciones de la PUAP (5)}{}
Si \(I\) es un ideal del anillo \(A\), la proyección \(\pi : A \to A/I\) induce un homomorfismo suprayectivo \(\overline{\pi} : A[X] \to (A/I)[X]\). Si ponemos \(\overline{a} = a + I\), el homomorfismo \(\overline{\pi}\) viene dado explícitamente por \[\overline{\pi} \left( \sum_{n \geq 0} p_n X^n \right) = \sum_{n \geq 0} \overline{p_n} X^n.\] A \(\overline{\pi}\) se le llama el homomorfismo de reducción de coeficientes módulo I. Su núcleo, que es un ideal de \(A[X]\), consiste en los polinomios con coeficientes en \(I\), y lo denotaremos por \(I[X]\). Del Primer Teorema de Isomorfía se tiene que \((A/I)[X] \simeq \frac{A[X]}{I[X]}.\)
\end{example}

\begin{example}{Aplicaciones de la PUAP (6)}{}
Sea \(A\) un subanillo de \(B\) y sea \(S_b : A[X] \to B\) el homomorfismo de sustitución en cierto elemento \(b\) de \(B\). Entonces Im \(S_b\) es el subanillo de \(B\) generado por \(A \cup \{b\}\), y consiste en las ``expresiones polinómicas en \(b\) con coeficientes en \(A\)''; es decir, en los elementos de la forma \[\sum_{i=0}^{n} a_i b^i,\] donde \(n \geq 0\) y \(a_i \in A\) para cada \(i\). Este subanillo se suele denotar por \(A[b]\) y es el menor subanillo de \(B\) que contiene a \(A \cup \{b\}\).
    
% TODO: explicar esto bien
Por ejemplo, si \(A = \mathbb{Z}, B = \mathbb{C}\) y \(b = \sqrt{m}\) para cierto \(m \in \mathbb{Z}\), entonces la notación anterior es compatible con la que se usó anteriormente (es decir, \(\mathbb{Z}[\sqrt{m}]\) representa el mismo subanillo atendiendo a cualquiera de las dos definiciones). Lo mismo ocurre si se toma \(A = \mathbb{Q}\). Si además \(m \equiv 1\) mod 4 entonces \(\mathbb{Z} \left[ \frac{1+\sqrt{m}}{2} \right]\) es el anillo \(A_m = \{\frac{a + b \sqrt{2}}{2} : a \equiv b \text{ mod 2}\}\) y \(\mathbb{Q}[\sqrt{m}] = \mathbb{Q} \left[ \frac{1+\sqrt{m}}{2} \right]\).
\end{example}

\clearpage
\section{Raíces de polinomios}

Empezaremos esta sección con el siguiente lema. Recuérdese que consideramos el polinomio cero como un polinomio de grado $-\infty$.

\begin{lemma}{}{}
Sea $A$ un anillo y sean $f,g\in A[X]$. Si el coeficiente principal de $g$ es invertible en $A$, entonces existen dos únicos polinomios $q,r\in A[X]$ tales que $f = gq + r$ y $\gr(r) < \gr(g)$.

En esta situación, $q$ y $r$ se llaman cociente y resto de la división de $f$ entre $g$.
\end{lemma}

\begin{proofbox}
\emph{Para la existencia usamos el argumento que vimos en el Ejemplo \ref{ex:grado_funcion_euclidea}.}

Sea $m = \gr(g)$ y sea $b$ el coeficiente principal de $g$, que es invertible en $A$ por hipótesis. Dado $f\in A[X]$ vamos a ver, por inducción en $n = \gr(f)$, que existen $q, r\in A[X]$ satisfaciendo las propiedades del Lema. 

Si $n<m$ podemos tomar $q = 0$ y $r = f$. Supongamos pues que $n\geq m$ y que la propiedad se verifica si $f$ se sustituye por un polinomio de grado menor. Si $a$ es el término principal de $f$, es claro que el polinomio $f_{1} = f - ab^{-1} X^{n-m} g\in A[X]$ tiene grado menor que el de $f$. Por hipótesis de inducción existen $q_{1}, r\in A[X]$ tales que $f_{1} = gq_{1} + r$ y $r = 0$ o $\gr(r) < m$. Entonces $f = g(q_{1} + ab^{-1} X^{n-m}) + r$, lo que termina la demostración de la existencia de cociente y resto.

En cuanto a la unicidad, supongamos que $f=gq_{1}+r_{1}=gq_{2}+r_{2}$ con $\gr(r_{i}) < \gr(g)$ para cada $i=1,2$. Como el término principal de $g$ es regular, del Lema \ref{lem:grados_polinomios} se deduce que
\[
gr(g)+gr(q_{1}-q_{2})= \gr(g(q_{1 }-q_{2}))= \gr(r_{2}-r_{1})\leq\max\{\gr(r_{2}), \gr(r_{1})\} < \gr(g).
\]

Luego $\gr(q_{1}-q_{2})<0$ y en consecuencia $q_{1}=q_{2}$, de donde $r_{1}=r_{2}$.
\end{proofbox}

\begin{proposition}{}{}
Sean $A$ un anillo, $a\in A$ y $f\in A[X]$. Entonces:

\begin{enumerate}
\item (Teorema del Resto) El resto de la división de $f$ entre $X-a$ es $f(a)$.
\item (Teorema de Ruffini) $X-a$ divide a $f$ si y solo si $f(a)=0$. En tal caso se dice que $a$ es una raíz de $f$.
\end{enumerate}
\end{proposition}

\begin{proofbox}
Dividiendo $f$ entre $X-a$ (podemos hacerlo puesto que el coeficiente principal de $X-a$ es $1$) tenemos $f=q(X-a)+r$ con $\gr(r)<1$, por lo que $r$ es constante y así $r=r(a)=f(a)-q(a)(a-a)=f(a)$. Esto demuestra (1), y (2) es entonces inmediato.
\end{proofbox}

Fijemos $a \in A$. Como, para cada $k \in N_0$, el polinomio $(X - a)^k$ es monico de grado $k$, se tiene $gr((X - a)^k q) = k + gr(q)$ para cada $q \in A[X]$. Por tanto, para cada $f \in A[X]$ no nulo, existe un mayor $m \in N_0$ tal que $(X - a)^m$ divide a $f$. Este entero $m$, que verifica $0 \leq m \leq gr(f)$, se llama la multiplicidad de $a$ en $f$. Por el Teorema de Ruffini, $a$ es raíz de $f$ precisamente si $m \geq 1$. Cuando $m = 1$ se dice que $a$ es una raíz simple de $f$, y cuando $m > 1$ se dice que $a$ es una raíz múltiple de $f$.

\begin{lemma}{}{}
Sean $a \in A$ y $f \in A[X]$. La multiplicidad de $a$ en $f$ es el único entero no negativo $m$ tal que $f = (X - a)^m g$ para algún polinomio $g \in A[X]$ del que $a$ no es raíz.
\end{lemma}

\begin{proofbox}
Sea $n$ la multiplicidad de $a$ en $f$. Entonces
\[
f = (X - a)^n h
\]
para un polinomio $h$, pero $(X - a)^{n+1}$ no divide a $f$. Si $a$ es raíz de $h$, entonces $X - a$ divide a $h$ y por tanto $(X - a)^{n+1}$ divide a $f$, que acabamos de decir que no pasa. Luego $a$ no es raíz de $h$.

Recíprocamente, supongamos que
\[
f = (X - a)^m g
\]
con $g \in A[X]$ y $g(a) = 0$. Por la definición de multiplicidad, $m \leq n$. Como $X - a$ es monico también es cancelable en $A[X]$, y por tanto de
\[
(X - a)^m g = (X - a)^n h
\]
deducimos que $(X - a)^{n-m} h = g$. Si $n > m$ entonces $g(a) = 0$, en contra de la suposición. Luego $m = n$.
\end{proofbox}

Cuando $D$ es un dominio, del Teorema de Ruffini se deduce que $X - a$ es primo para cualquier $a \in D$. En efecto
\[
(X - a) \mid fg \implies f(a)g(a) = 0 \implies f(a) = 0 ~\text{ ó }~ g(a) = 0 \implies (X - a) \mid f ~\text{ ó }~ (X - a) \mid g.
\]
Esto es esencial en la demostración del siguiente resultado.

\begin{proposition}{Acotación de raíces}{acotacion_raices}
Sean $D$ un dominio y $0 \neq f \in D[X]$. Entonces:

\begin{enumerate}
\item Si $a_1, \ldots, a_n \in D$ son distintos dos a dos y $\alpha_1, \ldots, \alpha_n \geq 1$ son enteros con cada $(X - a_i)^{\alpha_i}$ dividiendo a $f$, entonces $(X - a_1)^{\alpha_1} \cdots (X - a_n)^{\alpha_n}$ divide a $f$. Por tanto $\sum_{i=1}^n \alpha_i \leq gr(f)$.

\item La suma de las multiplicidades de todas las raíces de $f$ es menor o igual que $gr(f)$. En particular, el número de raíces distintas de $f$ es menor o igual que $gr(f)$.
\end{enumerate}
\end{proposition}

\begin{proofbox}
Es claro que basta con demostrar la primera afirmación de (1), cosa que hacemos por inducción en $s = \sum_{i=1}^n \alpha_i$. El caso $s = 1$ es inmediato, ya que por hipótesis $(X - a_1)^\alpha_1$ divide a $f$.

Cuando $s > 1$, por hipótesis $(X - a_1)^{\alpha_1}$ divide a $f$, y por hipótesis de inducción
\[
(X - a_1)^{\alpha_1 - 1}(X - a_2)^{\alpha_2} \cdots (X - a_{n})^{\alpha_{n}} \mid f
\]
también. Por tanto, sabemos que existen polinomios $g$ y $h$ tales que
\[
g(X - a_1)^{\alpha_1} = f = h(X - a_1)^{\alpha_1-1}(X - a_2)^{\alpha_2} \cdots (X - a_n)^{\alpha_n}.
\]
Cancelando $(X - a_1)^{\alpha_1-1}$ y usando el hecho de que $X - a_1$ es primo y no divide a ningún otro $X - a_i$ (pues $a_1 \neq a_i$ para $i \geq 2$), deducimos que $X - a_1$ divide a $h$, y esto nos da el resultado.
\end{proofbox}

Si $D$ no es un dominio, siempre podemos encontrar un polinomio en $D[X]$ para el que falle la acotación de raíces (es decir, "con más raíces que grado"). En efecto, si $0 \neq a, b \in D$ y $ab = 0$, entonces $aX$ es un polinomio de grado 1 con al menos 2 raíces, 0 y $b$. Otro ejemplo se obtiene considerando el polinomio $X^2 - 1$, que tiene 4 raíces en $\mathbb{Z}_8$.

El siguiente corolario evidente de la Proposición \ref{prop:acotacion_raices} se conoce como el principio de las identidades polinómicas. Ya hemos comentado que su segundo apartado falla sobre cualquier anillo finito.

\begin{corollary}{Principio de las identidades polinómicas}{}
Sea $D$ un dominio, y sean $f, g \in D[X]$. Entonces:

\begin{enumerate}
\item Si las funciones polinomicas $f,g:D\to D$ coinciden en $m$ elementos de $D$ y se tiene que $m>gr(f)$ y $m>gr(g)$, entonces $f=g$ (como polinomios).

\item Si $D$ es infinito entonces dos polinomios distintos definen funciones polinomicas distintas en $D$.
\end{enumerate}
\end{corollary}

\begin{proofbox}

\begin{enumerate}
\item Si $f \neq g$, entonces $f - g \neq 0$ y tiene grado a lo sumo $\max\{\gr(f), \gr(g)\}$. Pero $f-g$ tiene al menos $m$ raíces, lo que contradice la Proposición \ref{prop:acotacion_raices}.

\item Si $f$ y $g$ definen la misma función polinómica, entonces $f-g$ se anula en todo $D$, que es infinito, luego por (1) se tiene $f=g$.
\end{enumerate}

\end{proofbox}

La necesidad de la hipotesis de infinitud del dominio $D$ en el Corolario anterior resulta obvia si observamos que si $K$ es un cuerpo (recuerdese que todo dominio finito es un cuerpo) entonces hay infinitos polinomios con coeficientes en $K$ pero solo un numero finito de aplicaciones de $K$ en $K$.

Para un ejemplo explícito recordemos el Pequeño Teorema de Fermat que afirma que si $p$ es primo, entonces $a^{p}\equiv a \bmod p$. Eso implica que todos los elementos del cuerpo $\mathbb{Z}/p\mathbb{Z}$ son raíces del polinomio no nulo $X^{p}-X$.

El siguiente concepto es útil para calcular multiplicidades.

\begin{definition}{Derivada de un polinomio}{}
Sea $A$ un anillo. La derivada de $P=a_{0}+a_{1}X+\cdots+a_{n}X^{n}\in A[X]$ se define como
\[
D(P)=P^{\prime}=a_{1}+2a_{2}X+3a_{3}X^{2}+\cdots+na_{n}X^{n-1}.
\]
\end{definition}

Observese que la derivada no se ha definido a partir de ningún concepto métrico. Por ejemplo, no es cierto en general que un polinomio con derivada nula sea constante (considérese por ejemplo $X^{n}\in\mathbb{Z}_{n}[X]$). Sin embargo, esta derivada formal satisface las mismas propiedades algebraicas que la derivada del Análisis.

\begin{lemma}{}{derivadas_polinomicas}
Dados $a,b\in A$ y $P,Q\in A[X]$:
\begin{enumerate}
\item $(aP+bQ)^{\prime}=aP^{\prime}+bQ^{\prime}$.
\item $(PQ)^{\prime}=P^{\prime}Q+PQ^{\prime}$.
\item $(P^{n})^{\prime}=nP^{n-1}P^{\prime}$.
\end{enumerate}
\end{lemma}

\begin{proofbox}
Sean
\[
P = p_0 + p_1 X + \dots + p_n X^n, Q = q_0 + q_1 X + \dots + q_m X^m
\]
y supongamos sin perder generalidad que $n \geq m$.
\begin{enumerate}
\item Es inmediato que
\[
aP + bQ = c_0 + c_1 X + \dots + c_n X^n
\]
donde $c_k = ap_k + bq_k$. Entonces
\[
(aP+bQ)' = c_1 + 2c_2 X + \dots + nc_n X^{n-1} 
\]
mientras que 
\[
P' = p_1 + \dots + np_n X^{n-1},\quad Q' = q_1 + \dots + mq_m X^{m-1}
\]
de donde es claro que $aP' + bQ'$ tiene coeficientes
\[
ap_1 + bq_1, 2(ap_2 + bq_2), \dots, n(ap_n + bq_n)
\]
es decir, los mismos que $(aP+bQ)'$, como queríamos ver.
\end{enumerate}

Las otras dos me dan pereza, quedan como ejercicio para el lector.
\end{proofbox}

\begin{proposition}{}{}
Un elemento $a\in A$ es una raiz multiple de $P\in A[X]$ si y solo si $P(a)=P^{\prime}(a)=0$.
\end{proposition}

\begin{proofbox}
Ya sabemos que $a$ es una raiz de $P$ si y solo si $P(a)=0$. Si $a$ es raiz simple se tiene $P=(X-a)Q$ para cierto $Q\in A[X]$ con $Q(a)\neq 0$, por lo que, del Lema \ref{lem:derivadas_polinomicas} tenemos que
\[
P^{\prime}=Q+(X-a)Q^{\prime}
\]
y asi $P^{\prime}(a)=Q(a)\neq 0$.

Si $a$ es raiz multiple se tiene $P=(X-a)^{2}Q$ para cierto $Q\in A[X]$, por lo que $P^{\prime}=2(X-a)Q+(X-a)^{2}Q^{\prime}$ y asi $P^{\prime}(a)=0$.
\end{proofbox}

En dominios de caracteristica cero, la idea de la demostracion anterior puede usarse para determinar la multiplicidad de $a$ en $P$ (no solo para decidir si $a$ es simple o multiple). Para ello, necesitamos considerar las derivadas sucesivas de un polinomio: Para cada $n\geq 0$ se define la derivada $n$-esima $P^{(n)}$ de $P\in A[X]$, de forma recurrente, por las formulas:
\[
P^{(0)} = P,\quad P^{(n+1)} = \left( P^{(n)} \right)'
\]

\begin{proposition}{}{}
Sea $D$ un dominio de característica 0, y sean $P \in D[X]$ y $a \in D$. Entonces la multiplicidad de $a$ en $P$ es el menor $m \in \N_0$ tal que $P^{(m)}(a) \neq 0$.
\end{proposition}

\begin{proofbox}
Haremos inducción en la multiplicidad $m$ de $a$ en $P$. Si $m = 0$ el resultado es inmediato puesto que $a$ no es raíz de $P$, luego $P^{(0)}(a) = P(a) \neq 0$.

Supongamos ahora que se cumple para $m-1$. Sea $a$ raíz de $P$ con multiplicidad $m$, entonces $P = (X - a)Q$ para cierto $Q \in D[X]$.Claramente, la multiplicidad de $a$ en $Q$ es $m - 1$, y por hipótesis de inducción para todo $i < m - 1$ se tiene $Q^{(i)}(a) = 0$ mientras que $Q^{(m-1)}(a) \neq 0$.

Además, usando el Lema \ref{lem:derivadas_polinomicas} es fácil demostrar por inducción que para cada $t \geq 1$ se tiene
\[
P^{(t)} = tQ^{(t-1)} + (X - a)Q^{(t)}.
\]
Entonces es inmediato que para todo $i < m$
\[
P^{(i)}(a) = iQ^{(i-1)}(a) + (a - a)Q^{(i)} = 0,
\]
mientras que 
\[
P^{(m)}(a) = mQ^{(m-1)}(a) + (a - a)Q^{(i)} = mQ^{(m-1)}(a) \neq 0.
\]
\end{proofbox}

La hipótesis sobre la característica de $D$ en la Proposición anterior es necesaria. Por ejemplo, si $p$ es un número primo, $K = \Z_p$ y $P = X^p$, entonces $P' = 0$ y así $P^{(n)} = 0$ para todo $n$. No todos los polinomios con coeficientes en un anillo $A$ tienen raíces en $A$. Por ejemplo, los polinomios de grado 0 no tienen ninguna raíz, y un polinomio lineal $aX + b$ (con $a \neq 0$) tiene una raíz en $A$ si y solo si a divide a $b$. En particular, todo polinomio lineal sobre un cuerpo tiene una raíz, pero puede haber polinomios de grado positivo sin raíces: por ejemplo, $X^2 + 1$ no tiene raíces en $\R$, y para todo entero primo $p$ y todo $n \geq 2$ el polinomio $X^n - p$ no tiene raíces en $\Q$.

\clearpage
\section{Divisibilidad en anillos de polinomios}

La siguiente proposición caracteriza cuándo un anillo de polinomios es un DIP o dominio euclídeo y cuáles son los irreducibles en tal caso.

\begin{proposition}{}{}
Para un anillo $A$, las condiciones siguientes son equivalentes:

\begin{enumerate}
\item $A[X]$ es un dominio euclídeo.
\item $A[X]$ es un dominio de ideales principales.
\item $A$ es un cuerpo.
\end{enumerate}

En este caso, un polinomio $f \in A[X]$ es irreducible si y solo si es primo si y solo si $\operatorname{gr}(f) > 0$ y $f$ no es producto de dos polinomios de grado menor; es decir, si una igualdad $f = gh$ en $A[X]$ implica que $\operatorname{gr}(g) = \operatorname{gr}(f)$ (y $\operatorname{gr}(h) = 0$) ó $\operatorname{gr}(h) = \operatorname{gr}(f)$ (y $\operatorname{gr}(g) = 0$).
\end{proposition}

\begin{proofbox}

\begin{itemize}
\item Ya sabemos que (1) implica (2) por el Teorema \ref{thm:DE_implica_DIP} y que (3) implica (1) por el Ejemplo \ref{ex:grado_funcion_euclidea}.

\item Para ver (2) implica (3) notemos que laramente el polinomio $X$ es irreducible, con lo que si $A[X]$ es DIP entonces el ideal $(X)$ es maximal. Si $a \in A \setminus \{0\}$ entonces $a \notin (X)$ con lo que de la maximalidad de $(X)$ deducimos que $(a, X) = A[X]$ y por tanto $1 = aP + XQ$ para ciertos $P, Q \in A[X]$. Luego $1 = aP(0)$, con lo que $a$ es invertible en $A$. Esto demuestra que $A$ es un cuerpo.
\end{itemize}

En caso de que se verifiquen las equivalencias, por \ref{prop:propiedades_DIP} $f \in A[X]$ es irreducible si y solo si es primo.

Para la otra caracterización recordemos que por el Corolario \ref{cor:unidades_dominio_polinomios} $A[X]^* = A^*$, es decir, las únicas unidades son los polinomios de grado $0$ (todos ellos ya que $A$ es cuerpo).

\begin{itemize}
\item Supongamos que $f$ es irreducible, entonces no puede ser 0 ni una unidad, luego $\gr(f) > 0$, pues si fuera $\gr(f) = 0$ entonces $f$ sería una unidad. Además, si $f = gh$ entonces $g \in A[X]^*$ o $h \in A[X]^*$, es decir, $g$ o $h$ tienen grado 0, luego los grados verifican lo esperado.

\item Supongamos por el contrario que $f$ cumple las condiciones descritas. Como $\gr(f) > 0$ sabemos que $0 \neq f \in A \setminus A^*$. Además, si $f = gh$ entonces $g$ o $h$ tienen grado 0, es decir, uno de los dos es una unidad.
\end{itemize}

\end{proofbox}

Obsérvese que si $a \in A$ y $f \in A[X]$ entonces
\[
a \mid f \iff f = ag \iff f_0 + f_1X + \dots + f_nX^n = ag_0 + ag_1X + \dots + ag_nX^n \iff f_k = a g_k \iff a \mid f_k
\]
es decir, $a \mid f$ si y solo si $a$ divide a todos los coeficientes de $f$.

\begin{lemma}{}{}
Sea $D$ un dominio y sea $p\in D$.

\begin{enumerate}
    \item $p$ es irreducible en $D$ si y solo si lo es en $D[X]$.
    \item Si $p$ es primo en $D[X]$ entonces lo es en $D$.
    \item Si ademas $D$ es un DFU entonces las condiciones siguientes son equivalentes:
    \begin{enumerate}
        \item $p$ es irreducible en $D$.
        \item $p$ es irreducible en $D[X]$.
        \item $p$ es primo en $D$.
        \item $p$ es primo en $D[X]$.
    \end{enumerate}
\end{enumerate}

\end{lemma}

\begin{proofbox}
\begin{enumerate}
    \item Es consecuencia del Corolario \ref{cor:unidades_dominio_polinomios}.
    \item Supongamos que $p \mid ab$ en $D$, entonces $p \mid ab$ en $D[X]$. Como $p$ es primo en $D[X]$ podemos suponer sin perder generalidad que $p \mid a$, en tal caso
    \[
    a = pf,\quad f \in D[X]
    \] 
    pero entonces comparando grados $0 = \gr(a) = \gr(p) + \gr(f) = \gr(f)$, luego $f \in D$ y por tanto $p \mid a$ en $D$.
    \item Por (1) sabemos que (a) y (b) son equivalentes. Además, por la Proposición \ref{prop:primo_implica_irreducible} ser primo implica ser irreducible, luego (c) implica (a) y (d) implica (b) por otro lado, por el Lema \ref{lem:DFU_irreducible} (a) implica (c). En resumen: (a), (b), (c) son equivalentes y (d) implica (b), luego solo falta ver que (c) implica (d).

    Supongamos por tanto que $p$ es primo en $D$, y veamos que lo es en $D[X]$. Para ello, sean
    \[
    a=a_{0}+\cdots+a_{n}X^{n}\qquad\text{y}\qquad b=b_{0}+\cdots+b_{m}X^{m}
    \]
    polinomios de $D[X]$ tales que $p\nmid a$ y $p\nmid b$, y veamos que $p\nmid ab$. Recordemos que $p$ divide a un polinomio si y solo si divide a todos sus factores, luego por hipótesis, existen un menor indice $i$ tal que $p\nmid a_{i}$, y un menor indice $j$ tal que $p\nmid b_{j}$. El coeficiente de grado $i+j$ de $ab$ es
    \[
    c_{i+j}=a_{0}b_{i+j}+\cdots+a_{i-1}b_{j+1}+a_{i}b_{j}+a_{i+1}b_{j-1}+\cdots+a_{i +j}b_{0},
    \]
    y las condiciones dadas implican que $p$ divide a todos los sumandos excepto a $a_{i}b_{j}$, por lo que $p\nmid c_{i+j}$ y en consecuencia $p\nmid ab$.
\end{enumerate}

\end{proofbox}

\clearpage
\subsection{Polinomios sobre dominios de factorización única}

\emph{En el resto de la seccion $D$ sera un DFU y $K$ su cuerpo de fracciones.}

Consideremos la funcion
\[
\varphi:D\setminus\{0\}\to\mathbb{N}_{0}
\]
que a cada $0\neq a\in D$ le asocia el numero $\varphi(a)$ de factores irreducibles en la expresion de $a$ como producto de irreducibles de $D$, contando repeticiones. Por ejemplo, si $D=\mathbb{Z}$ entonces $\varphi(12)=3$ y $\varphi(-80)=5$. Es claro que, si $a,b\in D\setminus\{0\}$, entonces
\[
\varphi(ab)=\varphi(a)+\varphi(b)\qquad\text{y}\qquad\varphi(a)=0\Leftrightarrow a \in D^{*}.
\]

\begin{lemma}{}{}
Si $a\in D$ y $f,g,h\in D[X]$ verifican $af=gh\neq 0$, entonces existen $g_{1},h_{1}\in D[X]$ tales que

\[f=g_{1}h_{1},\qquad\operatorname{gr}(g_{1})=\operatorname{gr}(g),\qquad\operatorname{gr}(h _{1})=\operatorname{gr}(h).\]
\end{lemma}

\begin{proofbox}
Razonamos por induccion en $\varphi(a)$. Si $\varphi(a)=0$ podemos tomar $g_{1}=a^{-1}g$ y $h_{1}=h$. Si $\varphi(a)>0$, existen $p,b\in D$ tales que $a=pb$ y $p$ es primo. Entonces $p\mid af=gh$ en $D[X]$ y, por el Lema 3.14, podemos asumir que $p\mid g$ en $D[X]$. Es decir, existe $\overline{g}\in D[X]$ tal que $g=p\overline{g}$, de donde $\operatorname{gr}(g)=\operatorname{gr}(\overline{g})$. Cancelando $p$ en la igualdad $pbf=af=gh=p\overline{g}h$ obtenemos $bf=\overline{g}h$. Como $\varphi(b)=\varphi(a)-1<\varphi(a)$, la hipotesis de induccion nos dice que existen $g_{1},h_{1}\in D[X]$ tales que $f=g_{1}h_{1}$, $\operatorname{gr}(g_{1})=\operatorname{gr}(\overline{g})=\operatorname{gr}(g)$, y $\operatorname{gr}(h_{1})=\operatorname{gr}(h)$, lo que nos da el resultado.
\end{proofbox}

El siguiente resultado relaciona la irreducibilidad de un polinomio sobre $D$ con su irreducibilidad sobre $K$. Aunque su recíproco es falso en general (piensese en $2X$ como polinomio sobre $\mathbb{Z}$), pronto veremos que es válido con una condicion extra sobre el polinomio (Proposicion 3.20).

\begin{lemma}{}{}
Si $f\in D[X]\setminus D$ es irreducible en $D[X]$, entonces es irreducible (y primo) en $K[X]$.
\end{lemma}

\begin{proofbox}
Supongamos que $f$ no es irreducible en $K[X]$. Por la Proposicion 3.13, existen $G,H\in K[X]$ tales que

\[f=GH,\qquad\operatorname{gr}(G)>0,\qquad\operatorname{gr}(H)>0.\]

Si $0\neq b\in D$ es un multiplo comun de los denominadores de los coeficientes de $G$, se tiene $g=bG\in D[X]$, y analogamente existe $0\neq c\in D$ tal que $h=cH\in D[X]$. Aplicando el Lema 3.15 a la igualdad $(bc)f=gh$ obtenemos $g_{1},h_{1}\in D[X]$ tales que $f=g_{1}h_{1}$, $\operatorname{gr}(g_{1})=\operatorname{gr}(g)=\operatorname{gr}(G)>0$, y $\operatorname{gr}(h_{1})=\operatorname{gr}(h)=\operatorname{gr}(H)>0$, lo que nos da una factorizacion no trivial de $f$ en $D[X]$.
\end{proofbox}

Podemos ya demostrar el resultado principal de esta seccion:

\begin{theorem}{}{}
$D$ es un DFU si y solo si lo es $D[X]$.
\end{theorem}

\begin{proofbox}
Supongamos primero que $D[X]$ es un DFU. Entonces $D$ es un dominio (Corolario 3.2), y cada $0\neq a\in D\setminus D^{*}$ es producto de irreducibles de $D[X]$, que tendran grado 0 pues lo tiene $a$. Por el Lema 3.14, esa sera una factorizacion de $a$ en irreducibles de $D$. Del mismo lema se deduce que todo irreducible de $D$ es primo en $D$, por lo que $D$ es un DFU.

Supongamos ahora que $D$ es un DFU y veamos que lo es $D[X]$. Empezaremos demostrando que cada $a=a_{0}+\cdots+a_{n}X^{n}\in D[X]$ (con $a_{n}\neq 0$) no invertible es producto de irreducibles, y lo haremos por induccion en $n+\varphi(a_{n})$. Observese que $a$ es invertible si y solo si $n+\varphi(a_{n})=0$. El caso $n+\varphi(a_{n})=1$ se resuelve facilmente. Supongamos pues que $n+\varphi(a_{n})>1$ y que $a$ no es irreducible. Entonces existen

\[b=b_{0}+\cdots+b_{m}X^{m}\quad(b_{m}\neq 0)\qquad\text{y}\qquad c=c_{0}+\cdots +c_{k}X^{k}\quad(c_{k}\neq 0)\]

en $D[X]$, no invertibles, con $a=bc$ y $b$ y $c$ elementos de $D[X]$ que no son unidades de $D[X]$. Entonces

\[0<m+\varphi(b_{m}),\quad 0<k+\varphi(c_{k})\quad\text{y}\quad n+\varphi(a_{n})= m+k+\varphi(b_{m})+\varphi(c_{k}).\]

En consecuencia, podemos aplicar la hipotesis de induccion a $b$ y $c$, y pegando las factorizaciones asi obtenidas conseguimos una factorizacion en irreducibles de $a$.

Por la Proposicion 2.22, solo falta demostrar que todo irreducible $f$ de $D[X]$ es primo, y por el Lema 3.14 podemos suponer que $\operatorname{gr}(f)\geq 1$. Sean pues $g,h\in D[X]$ tales que $f\mid gh$ en $D[X]$, y veamos que $f\mid g$ o $f\mid h$ en $D[X]$. Obviamente, $f\mid gh$ en $K[X]$, y como $f$ es primo en $K[X]$ por el Lema 3.16, podemos asumir que $f\mid g$ en $K[X]$. Es decir, existe $G\in K[X]$ tal que $g=fG$, y si demostramos que $G\in D[X]$ habremos terminado. Para ello, tomando $a\in D$ con $aG\in D[X]$ y $\varphi(a)$ mínimo, basta ver que $\varphi(a)=0$. Supongamos que $\varphi(a)>0$ y sean $p,b\in D$ con $a=pb$ y $p$ primo. Entonces, en $D[X]$, se tiene $p\mid ag=f(aG)$. Como $p$ es primo en $D[X]$ (Lema 3.14) y $p\nmid f$ (pues $f$ es irreducible y $\operatorname{gr}(f)\geq 1$), deducimos que $p\mid aG$ en $D[X]$. Si $g_{1}\in D[X]$ verifica $aG=pg_{1}$ entonces $bG=g_{1}\in D[X]$, contra la minimalidad de $\varphi(a)$, y esta contradiccion termina la demostracion.
\end{proofbox}

De la Proposicion 3.13 y el Teorema 3.17 se deduce que ${\mathbb{Z}}[X]$ es un DFU pero no un DIP, lo que muestra que el reciproco del Teorema 2.25 no es cierto.

En el resto de la seccion suponemos que $D$ es un DFU y $K$ es su cuerpo de fracciones.

Definimos una relacion de equivalencia $\sim$ en $K$ de la siguiente forma para $x,y\in K$:

\[x\sim y\Leftrightarrow y=ux\text{ para algun }u\in D^{*}.\]

Claramente la clase de equivalencia que contiene a $x$ es $xD^{*}=\{xu:u\in D^{*}\}$. En particular, si $x\in D$ entonces la clase de equivalencia que contiene a $x$ esta formada por los elementos que son asociados de $x$ en $D$. Por ejemplo, $0D^{*}=\{0\}$, $1D^{*}=D^{*}$. Observese que $xyD^{*}=\{xa:a\in yD^{*}\}=x(yD^{*})$.

Podemos definir una multiplicacion de elementos de $K$ por elementos de $K/\sim$ poniendo

\[a(bD^{*})=(ab)D^{*}.\]

Esto esta bien definido pues si $b_{1}\sim b_{2}$ entonces $ab_{1}\sim ab_{2}$. Ademas se verifica $a(b(cD^{*}))=(ab)(cD^{*})$.

Vamos a definir una aplicacion

\[c:K[X]\to K/\sim\]

Empezamos definiendo $c(p)$ para $p\in D[X]$ como la clase que contiene a un maximo comun divisor de los coeficientes de $p$, o sea, si $p=\sum_{i\geq 0}p_{i}X^{i}$ entonces

\[c(p)=\operatorname{mcd}(p_{i}:i\geq 0)D^{*}.\]

Para definir $c(p)$ para un elemento $p\in K[X]$ elegimos $a\in D\setminus\{0\}$ con $ap\in D[X]$ y definimos

\[c(p)=a^{-1}c(ap).\]

Esto esta bien definido pues si $a_{1}p,a_{2}p\in D[X]$ entonces $c(a_{1}a_{2}p)=a_{1}c(a_{2}p)=a_{2}c(a_{1}p)$ con lo que $a_{1}^{-1}c(a_{1}p)=a_{2}^{-1}c(a_{2}p)$.

Si $c(p)=aD^{*}$, entonces decimos que $a$ es el contenido y abusaremos de la notacion escribiendo $a=c(p)$. En realidad deberiamos decir "un contenido" pero estamos abusando de la notacion, de la misma forma que lo haciamos al hablar "del maximo comun divisor" o "el mínimo comun multiplo". En todos los casos se trata de un concepto que es unico salvo multiplicacion por unidades de $D$.

Observese que si $a\in D$ y $p\in D[X]$ entonces las notaciones $a\mid c(p)$ y $c(p)\mid a$ no son ambiguas pues todos los valores posibles para $c(p)$ son asociados.

Veamos ahora algunas propiedades del contenido.

\begin{proposition}{}{}
Sean D un DFU y K su cuerpo de fracciones. Sean a $\in$ K y p $\in$ K[X].

\begin{enumerate}
\item[(1)] Si a $\in$ D y p $\in$ D[X] entonces a $|$ p en D[X] si y solo si a $|$ c(p) en D.
\item[(2)] c(ap) = ac(p).
\item[(3)] p $\in$ D[X] si y solo si c(p) $\in$ D.
\end{enumerate}
\end{proposition}

\begin{proofbox}
Pongamos b = c(p).
(1) Supongamos que a $\in$ D y p $\in$ D[X]. Entonces b es máximo común divisor de los coeficientes de p. Luego a $|$ p en D[X] si y solo si a divide a cada uno de los coeficientes de p (en D) si y solo si a divide a b.
(2) Es consecuencia inmediata de la fórmula $\operatorname{mcd}(ap_0, ap_1, \ldots, ap_n) = a \operatorname{mcd}(p_0, p_1, \ldots, p_n)$.
(3) Obviamente si p $\in$ D[X] entonces b $\in$ D. Para demostrar la otra implicación ponemos p = $\sum_{i=0}^k s_i X^i$ donde cada $s_i$ es una fracción reducida, entendiendo que si $r_i = 0$ entonces $s_i = 1$. Supongamos que p $\notin$ D[X]. Eso implica que algún $s_i$ no es unidad de D con lo que es divisible por un irreducible q y por tanto q no divide a $r_i$. Ponemos $s_i = q^{n_i} h_i$ con q no divide a $h_i$ para cada i y tomamos n = $\max(n_0, n_1, \ldots, n_k) \geq 1$ y m = $\operatorname{mcm}(s_0, \ldots, s_k)$. Entonces m = $q^n h$ con h $\in$ D y q no divide a h en D. Además $mp \in D[X] y c(mp) = mbD^*$. Pero $m s_i r_i = \frac{h r}{h_i}$ es el coeficiente de $X^i$ en $mp$, que es un elemento de D que no es múltiplo de q. Luego mb, que es el máximo común divisor de los coeficientes de mp, no es múltiplo de q en D. Pero m es un elemento de D que sí es múltiplo de q en D. Por tanto b $\notin$ D.
\end{proofbox}

Diremos que un polinomio es primitivo si c(p) = 1. Es decir p $\in$ D[X] es primitivo si los únicos divisores de p en D[X] que tienen grado 0 son las unidades de D[X]. Obsérvese que para todo 0 $\neq$ p $\in$ D[X] se tiene que p/c(p) es primitivo y de hecho c = c(p) si y solo si p = cp1 con p1 $\in$ D[X], primitivo.

\begin{lemma}{Lema de Gauss}{}
Si f, g $\in$ K[X], entonces c(fg) = c(f)c(g). En particular, si f y g son primitivos entonces fg es primitivo y si además f, g $\in$ D[X] entonces se verifica el recíproco.
\end{lemma}

\begin{proofbox}
Tenemos f = c(f)f1 y g = c(g)g1 con f1 y g1 primitivos. Por tanto fg = c(f)c(g)f1g1, luego para demostrar que c(fg) = c(f)c(g) basta probar que f1g1 es primitivo. En caso contrario c(f1g1) tendría un divisor irreducible p en D. Eso implica que p$|$f1g1. Por el Lema 3.14, p es primo en D[X] y por tanto p$|$f1 ó p$|$g1, lo que implica que p$|$c(f1) ó p$|$c(g1), en contra de que c(f1) = c(g1) = 1.
\end{proofbox}

\begin{proposition}{}{}
Para un polinomio primitivo f $\in$ D[X] $\setminus$ D, las condiciones siguientes son equivalentes:

\begin{enumerate}
\item[(1)] f es irreducible en D[X].
\item[(2)] f es irreducible en K[X].
\item[(3)] Si f = GH con G, H $\in$ K[X] entonces $\operatorname{gr}(G) = 0$ ó $\operatorname{gr}(H) = 0$.
\end{enumerate}
\end{proposition}