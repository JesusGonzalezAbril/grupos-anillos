\chapter{Grupos Abelianos Finitos}

En este capítulo vamos a describir todos los grupos abelianos finitos salvo isomorfismos. La mayoría de los grupos que aparecen en esta sección son abelianos y en general usaremos notación aditiva y los denotaremos \( A, B, A_i, B_i, \ldots \), mientras que utilizaremos siempre notación multiplicativa para grupos no necesariamente abelianos y los denotaremos \( G, H, G_i, H_i, \ldots \).

Un modo habitual de estudiar un objeto matemático consiste en descomponerlo en objetos más sencillos, estudiar éstos y recomponer entonces el objeto inicial. Lo que se entiende por objeto sencillo y la manera de descomponer y recomponer un objeto dependen de cada caso. En este capítulo el objeto estudiado será un grupo abeliano finito \( A \), y los objetos sencillos serán los grupos cíclicos, que ya conocemos bien. En este contexto, el proyecto sugerido al principio del párrafo funciona porque existe un método muy efectivo para descomponer \( A \) de modo que es muy fácil conocer \( A \) a partir de sus componentes. Se trata de la suma directa de subgrupos, que analizamos en la primera sección. Al final del capítulo demostraremos el Teorema Fundamental de los Grupos Abelianos Finitos que describe todos los grupos abelianos finitos salvo isomorfismos a partir de los grupos cíclicos (nuestros objetos sencillos) y sumas directas (nuestra forma de recomponer).

Durante todo el capítulo usaremos la notación $\langle g \rangle_n$ para indicar que el grupo cíclico generado por $g$ tiene orden $n$. Es importante tener en cuenta que al usar esta notación estamos sobreentendiendo que se ha justificado previamente que el orden de $\langle g \rangle$ es verdaderamente $n$.

\section{Sumas directas}

Comenzamos con una proposición que dará lugar al concepto de familia independiente de subgrupos. Si tenemos una familia \( (B_i)_{i \in I} \) de subgrupos de un grupo abeliano aditivo entonces los elementos de \( \sum_{i \in I} B_i \) tiene la forma \( \sum_{i \in I} b_i \) con \( b_i \in B_i \) para todo \( i \) y \( b_i = 0 \) para casi todo \( i \in I \), o sea el conjunto \( \{ i \in I : b_i \neq 0 \} \) es finito. Con el fin de no recargar el discurso, en el futuro cada vez que tengamos una suma \( \sum_{i \in I} b_i \), se entiende que \( b_i = 0 \) para casi todo \( i \), sin necesidad de decirlo explícitamente.

\begin{proposition}{Caracterización de independencia de subgrupos}{independencia_subgrupos}
Sean \( (B_i)_{i \in I} \) una familia de subgrupos de un grupo abeliano \( A \). Entonces las condiciones siguientes son equivalentes:

\begin{enumerate}
    \item El $0$ se expresa de manera única como suma de elementos de los \( B_i \). Es decir, si \( \sum_{i \in I} b_i = 0 \) con cada \( b_i \in B_i \), entonces se tiene \( b_i = 0 \) para todo \( i \in I \).
    \item Cada elemento de \(\sum_{i \in I} B_i\) se expresa de manera única como suma de elementos de los \(B_i\). Es decir, si \(\sum_{i \in I} b_i = \sum_{i \in I} b'_i\) con cada \(b_i \in B_i\) y cada \(b'_i \in B_i\), entonces se tiene \(b_i = b'_i\) para todo \(i \in I\).
    \item Para cada \(j \in I\) se verifica \(B_j \cap (\sum_{i \in I \setminus \{j\}} B_i) = 0\).
\end{enumerate}
\end{proposition}

\begin{proofbox}
(1) \(\Rightarrow\) (2): Supongamos que un elemento $x$ de $\sum_{i \in I} B_i$ se expresa de dos maneras como suma de elementos de los $B_i$
\[
x = \sum_{i \in I} b_i = \sum_{i \in I} c_i
\]
entonces, llamando $d_i = b_i - c_i$,
\[
0 = \sum_{i \in I} d_i
\]
por tanto, por hipótesis, $d_i = 0 \implies b_i = c_i$, lo que prueba que la expresión de $x$ es única.

(2) \(\Rightarrow\) (3): Sea $j \in I$ y supongamos que existe \(x \in B_j \cap (\sum_{i \in I \setminus \{j\}} B_i)\). Entonces existen $c_j \in B_j$ y unos $b_i \in B_i, i\neq j$ tales que
\[
x = c_j = \sum_{i \in I \setminus\{j\}} b_i
\]
entonces, por hipótesis, $c_j = 0$ y $\forall i,\ b_i = 0$, lo que prueba que $x = 0$.

(2) \(\Rightarrow\) (3): Supongamos que $0$ se expresa como suma de elementos de $B_i$
\[
0 = \sum_{i \in I} b_i.
\]
Entonces para cada $j \in I$ tenemos
\[
b_j = \sum_{i \in I \setminus \{j\}} b_i \in B_j \cap (\sum_{i \in I \setminus \{j\}} B_i) \implies b_j = 0
\]
por tanto, la única expresión de $0$ es $\forall i,\ b_i = 0$.
\end{proofbox}

\begin{definition}{Familia independiente y suma directa}{suma_directa}
Si se verifican las condiciones equivalentes de la Proposición \ref{prop:independencia_subgrupos} se dice que la familia de subgrupos \((B_i)_{i \in I}\) es independiente, o que los subgrupos \(B_i\) son independientes. Su suma,
\[
\sum_{i \in I} B_i,
\]
se llama entonces la suma directa de la familia \((B_i)_{i \in I}\), y se denota por 
\[
\oplus_{i \in I} B_i.
\]
En el caso en que se trate de una familia finita \((B_1, \ldots, B_n)\) también se denota \(\oplus_{i=1}^n B_i = B_1 \oplus \cdots \oplus B_n\).
\end{definition}

Un subgrupo \(B\) de \(A\) es un sumando directo de \(A\) si existe otro subgrupo \(C\) de \(A\) tal que \(A = B \oplus C\); es decir, tal que \(A = B + C\) y \(B \cap C = 0\). En este caso se dice que \(C\) es un complemento directo de \(B\).

\subsection{Ejemplos de subgrupos independientes y sumas directas}

\begin{example}{}{}
En el grupo \(A = \mathbb{Z}_6\) los subgrupos \(B = \langle 2 \rangle \) y \(C = \langle 3 \rangle\) son independientes y se tiene \(A = B \oplus C\).
\end{example}

\begin{proofbox}
Veamos que el $0$ se expresa de manera única en $B \oplus C$ ya que
\[
0 = b + c = [2n] + [3m] \implies n = 3, m = 2 \implies b = 0 = c
\]
luego $B, C$ son independientes. Además, es claro que $B \oplus C \subseteq A$, y si $[n] \in A$ entonces $[n] = -[2n] + [3n] \in B \oplus C$, por lo que $A = B \oplus C$.
\end{proofbox}

\begin{example}{}{Suma directa en grupos multiplicativos}
En el grupo multiplicativo $\R^*$, que es abeliano, se tiene $\R^* = \langle -1 \rangle \oplus \R^+$. Para grupos multiplicativos a veces se escribe $\otimes$ en lugar de $\oplus$.
\end{example}

\begin{proofbox}
Obviamente, cualquier elemento de $\R^*$ se puede expresar como producto de un elemento de $\langle -1 \rangle$ y otro de $\R^+$: si $x\in\R^*$ es positivo $x = 1|x|$ y si es negativo $x = -|x|$. Por otro lado, $\langle -1 \rangle \cap \R^+ = \{1\}$, que es precisamente el grupo trivial (equivalente a $0$ en la Definición \ref{defn:suma_directa}, que está escrita para notación aditiva).
\end{proofbox}

\begin{example}{}{}
Si \(A\) y \(B\) son grupos abelianos, entonces el grupo producto \(A \times B\) es la suma directa de los subgrupos \(A \times 0\) y \(0 \times B\).
\end{example}

\begin{example}{}{}
El complemento directo de un sumando directo no es, en general, único. Por ejemplo, para cualquier \(a \in \mathbb{Z}\) se tiene \(\mathbb{Z} \times \mathbb{Z} = \langle(1,0)\rangle \oplus \langle(a,1)\rangle\): la intersección es claramente nula, y un elemento arbitrario \((x,y)\) de \(\mathbb{Z} \times \mathbb{Z}\) se puede expresar como \((x,y) = y(a,1) + (x - ya)(1,0)\).
\end{example}

\begin{example}{}{Q_Z_indescomponibles}
En \(\mathbb{Q}\) no hay dos subgrupos no triviales que sean independientes. En efecto, si \(A\) y \(B\) son subgrupos no nulos y elegimos elementos no nulos \(\frac{a}{n} \in A\) y \(\frac{b}{m} \in B\), entonces
\[
0 = bn \frac{a}{n} - am \frac{b}{m}
\]
nos da una expresión no trivial del 0 como suma de elementos de \(A\) y \(B\). En \(\mathbb{Z}\) ocurre lo mismo, por un argumento similar.
\end{example}

Cuando sólo consideramos familias finitas, existe una estrecha relación entre los conceptos de suma directa y producto directo de grupos, que describimos a continuación dejando los detalles a cargo del lector.

Supongamos primero que \(A=B_{1}\oplus\cdots\oplus B_{n}\). Entonces, viendo cada \(B_{i}\) como grupo y considerando su producto \(B_{1}\times\cdots\times B_{n}\), la aplicación \(B_{1}\times\cdots\times B_{n}\to A\) dada por
\[
(b_{1},\ldots,b_{n})\mapsto b_{1}+\cdots+b_{n}
\]
es un isomorfismo de grupos. Es decir, si \(A\) es la suma directa de los \(B_{i}\), entonces \(A\) es isomorfo al producto directo de los \(B_{i}\).

Recíprocamente, sean \(B_{1},\ldots,B_{n}\) grupos abelianos y sea \(A\) el grupo producto, \(A=B_{1}\times\cdots\times B_{n}\). Si denotamos por \(\hat{B}_{i}\) al subgrupo de \(A\) formado por los elementos que llevan ceros en todas las coordenadas excepto tal vez en la \(i\)-ésima (o sea \(\hat{B}_{i}=0\times\cdots\times 0\times B_{i}\times 0\times\cdots\times 0\)), entonces es elemental ver que cada \(\hat{B}_{i}\) es isomorfo a \(B_{i}\) y que \(A=\hat{B}_{1}\oplus\cdots\oplus\hat{B}_{n}\). Es decir, si \(A\) es el producto directo de los \(B_{i}\), entonces \(A\) es la suma directa de los \(\hat{B}_{i}\), que son isomorfos a los \(B_{i}\).

En vista de esto, a menudo identificaremos \(B_{1}\oplus\cdots\oplus B_{n}\) con \(B_{1}\times\cdots\times B_{n}\).

En el caso infinito identificamos \(\oplus_{i\in I}B_{i}\) como el subgrupo de \(\prod_{i\in I}B_{i}\) formado por las \(I\)-uplas \((b_{i})_{i\in I}\) para las que \(\{i\in I:b_{i}\neq 0\}\) es finito.

% \begin{definition}{Suma directa de grupos (no necesariamente abelianos)}{suma_directa_general}
% Sea \( \{G_i\}_{i \in I} \) una familia de subgrupos (no necesariamente abelianos) de un grupo $G$. Si los subgrupos verifican
% \begin{enumerate}
% \item Para cada $i \in I$, $G_i$ es un subgrupo normal de $G$.
% \item Para cada $i \in I$, $G_i \cap \langle \cup_{j\neq i} G_j \rangle = \langle e \rangle$.
% \item $G = \langle \cup_{i \in I} G_i \rangle$.
% \end{enumerate}
% Decimos que $G$ es la suma directa de los $G_i$, y escribimos $G = \oplus_{i \in I} G_i$.
% \end{definition}

\clearpage
\section{Grupos indescomponibles y $p$-grupos}

\begin{definition}{Grupo indescomponible}{grupo_indescomponible}
Un grupo abeliano no nulo se dice que es indescomponible si no es suma directa de dos subgrupos propios. Es decir, $A$ es indescomponible si $A = X \oplus Y$ implica $X = 0$ ó $Y = 0$ (y por tanto $X = A$ ó $Y = A$).
\end{definition}

\begin{proposition}{Existencia de descomposición en indescomponibles}{descomposicion_indescomponibles}
Todo grupo abeliano finito y no nulo $A$ es una suma directa de subgrupos indescomponibles.
\end{proposition}

\begin{proofbox}
Podemos razonar por inducción sobre el orden de $A$. Si $A$ tiene orden $2$ solo tiene un subgrupo propio, el trivial, luego $A$ es indescomponible puesto que no se puede expresar como suma directa de subgrupos propios.

Supongamos que se cumple para grupos de orden menor que $n$ y sea $A$ un grupo con orden $n$. Si $A$ es indescomponible, entonces hemos terminado. En caso contrario existen $B,C < A$ tales que $A = B \oplus C$. Como ambos grupos son propios, por hipótesis se pueden expresar como suma directa de indescomponibles, luego
\[
B = B_1 \oplus \dots \oplus B_k,\quad C = C_1 \oplus \dots \oplus C_l
\]
y por tanto
\[
A = B_1 \oplus \dots \oplus B_k \oplus C_1 \oplus \dots \oplus C_l
\]
donde todos los factores son indescomponibles.
\end{proofbox}

\begin{example}{}{}
$\mathbb{Z}$ y $\mathbb{Q}$ son indescomponibles por el argumento usado en el Ejemplo \ref{ex:Q_Z_indescomponibles}.
\end{example}

\begin{example}{}{p_grupo_ciclico}
Todo grupo de orden primo es obviamente indescomponible ya que su único subgrupo propio es el trivial. Más generalmente, supongamos que $G=\langle g\rangle$ es un grupo cíclico de orden $p^{n}$ con $p$ un primo y $n\in\mathbb{N}$. Entonces los subgrupos de $G$ forman una cadena:
\[
1<\langle g^{p^{n-1}}\rangle_{p}<\langle g^{p^{n-2}}\rangle_{p^{2}}<\cdots< \langle g^{p^{2}}\rangle_{p^{n-2}}<\langle g^{p}\rangle_{p^{n-1}}<\langle g \rangle_{p^{n}}=G.
\]
Donde $\langle a \rangle_l$ indica que el grupo cíclico es de orden $l$.

Supongamos entonces que $G = H \oplus K$, entonces o bien $H < K$ o $K < H$, ya que todos los subgrupos de $G$ se encuentran en la cadena antes descrita .En cualquiera de los dos casos, $H \cap K \neq \{0\}$, lo que contradice que $H$ y $K$ son independientes.

Sin embargo, si $G$ es cíclico de orden $n$ pero $n$ no es una potencia de un primo entonces existen enteros coprimos $h$ y $k$ y mayores que $1$, con $n=hk$. Por tanto, $G$ tiene un grupo cíclico $H$ de orden $h$ y otro $K$ de orden $k$. Entonces $G=H\oplus K$, por tanto $G$ no es indescomponible.

Esta es otra reencarnación del Teorema Chino de los Restos: si $n$ se puede factorizar como producto de dos términos coprimos $n = pq$ entonces $C_n \cong C_p \oplus C_q$.
\end{example}

En el ejemplo anterior hemos caracterizado los grupos cíclicos finitos indescomponibles como aquellos cuyo orden es una potencia de un primo: todo grupo isomorfo a $C_{p^n}$ es indescomponible. El resto de la sección lo dedicamos a ver que no hay más grupos abelianos finitos indescomponibles.

\begin{definition}{Exponente}{exponente}
Sea $G$ un grupo no necesariamente abeliano ni finito.

Si existe un entero positivo $n$ tal que $g^{n}=1$ para todo $g\in G$ entonces al menor entero que cumple esa propiedad se le llama exponente o periodo de $G$. Denotaremos ese número por $\Exp(G)$ y en el caso que no exista tal número pondremos $\Exp(G)=\infty$ y diremos que $G$ tiene periodo infinito.
\end{definition}

\begin{definition}{Grupo periodico}{grupo_periodico}
Decimos que un grupo arbitrario $G$ (no necesariamente abeliano ni finito) es periódico o de torsión si para todo elemento de $G$ tiene orden finito, o sea si para todo $g\in G$ se verifica que $g^{n}=1$ para algún entero positivo.
\end{definition}

\begin{definition}{$p$-grupo}{p_grupo}
Sea $p$ un número primo. Un grupo en el que todo elemento tiene orden potencia de $p$ se dice que es un $p$-grupo.
\end{definition}

\begin{remark}
Nótese que esta definición de $p$-grupo es distinta de la dada en la Sección \ref{sec:p_grupos}. Sin embargo, veremos que ambas definiciones coinciden cuando el grupo es finito.
\end{remark}

Claramente si un grupo es finito entonces tiene periodo finito y si tiene periodo finito entonces el grupo es periódico. Sin embargo los recíprocos no se verifican. Por ejemplo, una suma directa infinita de copias de $\mathbb{Z}_{2}$
\[
\oplus_{i=1}^{\infty} \Z_2
\]
es periódico, ya que todo elemento cumple $a + a = 0$, pero no es finito. Por otro lado, la suma directa de todos los grupos de la forma $\mathbb{Z}_{n}$ con $n\geq 1$
\[
\oplus_{n=1}^{\infty} \Z_n
\]
es periódico, ya que los elementos son cero en casi todas sus componentes y por tanto para algún $k$ se cumple $ka = 0$, pero tiene periodo infinito, ya que fijado un $k$ siempre podemos encontrar un elemento en la suma directa tal que $ka \neq 0$.

También está claro que todo $p$-grupo es periódico. Sin embargo $\oplus_{n\in\mathbb{N}}\mathbb{Z}_{p^{n}}$ es un $p$-grupo de orden infinito.

Además, todo grupo que tenga orden potencia de $p$ es un $p$-grupo ya que por el Corolario \ref{cor:corolario_orden_lagrange} cualquier $g \in G$ debe cumplir
\[
|g| \Big \vert |G| = p^n.
\] 
Para el caso finito se verifica el recíproco: todo $p$-subgrupo finito tiene orden potencia de $p$. Eso es consecuencia inmediata del \hyperref[thm:teorema_cauchy]{Teorema de Cauchy}. Ponemos esto en un lema para uso futuro.

\begin{lemma}{Caracterización de $p$-grupos finitos}{caracterizacion_p_grupos_finitos}
Sean $G$ un grupo finito y $p$ un número primo. Entonces $G$ es un $p$-grupo si y solo si $|G|$ es una potencia de $p$.
\end{lemma}

\begin{proofbox}
Si $|G|$ es una potencia de $p$, por el Corolario \ref{cor:corolario_orden_lagrange} cualquier $g \in G$ debe cumplir
\[
|g| \Big\vert |G| = p^n
\] 
por lo que todos los elementos tienen orden potencia de $p$.

Por el contrario, sea $G$ un $p$-grupo finito. Si $q$ no es potencia de $p$ y divide a $|G|$, por el \hyperref[thm:teorema_cauchy]{Teorema de Cauchy} existe un elemento $g$ de orden $q$, pero esto contradice el hecho de que $G$ es un $p$-grupo.
\end{proofbox}

\begin{definition}{Subgrupo de $p$-torsion}{p_torsion}
Dados un grupo abeliano $A$ y un entero primo $p$, el subgrupo de $p$-torsión de $A$ es
\[
t_{p}(A)=\{a\in A:\text{ existe }n\in\mathbb{N}\text{ tal que }p^{n}a=0\}=\{a \in A:|a|\text{ es una potencia de }p\}.
\]
\end{definition}

Veamos que ambos conjuntos son iguales y que forman un subgrupo de $A$.
\begin{proposition}{}{}
Ambas definiciones de $t_p(A)$ son equivalentes y $t_p(A) < A$.
\end{proposition}

\begin{proofbox}
Denotemos
\[
X = \{a\in A:\text{ existe }n\in\mathbb{N}\text{ tal que }p^{n}a=0\},\ Y =\{a \in A:|a|\text{ es una potencia de }p\}.
\]
Si $a \in Y$ entonces $|a| = p^n$, por tanto
\[
\langle a \rangle = \{0,a,2a,\dots,(p^{n}-1)a\}
\]
como $p^n a \in \langle a \rangle$ debe ser $p^n a = 0$, ya que en caso contrario se tendría
\[
p^n a = k a \implies (p^n - k) a = 0 
\]
con $0 < p^n - k < p^{n}-1$, lo cual implica que $|a| < p^n$, contradiciendo la hipótesis. De esto se deduce que $a \in X$.

Por el contrario, si $a \in X$ entonces existe $n \in \N$ tal que $p^n a = 0$. De hecho, por el principio de buena ordenación podemos elegir $n$ como el menor natural tal que se verifica $p^n a = 0$. Supongamos que $|a| = k$ no es potencia de $p$, entonces $k$ es el menor natural tal que $ka = 0$ por un argumento similar al de la implicación anterior. Pero entonces $k \leq p^n$ y por el algoritmo de la división $p^n = qk + r$ y, por tanto,
\[
p^n a = qka + ra = q0 + ra = ra \implies ra = 0 \implies r = 0,
\]
luego $p^n = qk$. Si $q \neq 1$ entonces $k$ es una potencia de $p$, $k = p^m$ con $m<n$, lo cual contradice la minimalidad de $p^n$. De esto se deduce que debe ser $q=1$ y por tanto $|a| = k = p^n$.

Para ver que es un subgrupo basta notar que si $a,b \in t_p(A)$ entonces existen $n,m$ tales que
\[
p^n a = 0 = p^m b \implies p^{n+m} (a+b) = 0
\]
luego $a + b \in t_p(A)$. Que existe elemento neutro e inversos es inmediato ya que
\[
p^n(-a) = -p^n a = 0,\ 0 \in X.
\]
\end{proofbox}

De hecho, si $A$ es finito, $t_{p}(A)$ es claramente el mayor $p$-subgrupo de $A$ (es decir, el mayor subgrupo de $A$ que es un $p$-grupo).
\begin{proposition}{Descomposición primaria en $p$-grupos}{descomposicion_primaria_p}
Sea $A$ un grupo abeliano finito y sean $p_1, \ldots, p_k$ los divisores primos de $|A|$. Entonces
\[
A = t_{p_1}(A) \oplus \cdots \oplus t_{p_k}(A),
\]
con cada $t_{p_i}(A) \neq 0$.
\end{proposition}

\begin{proofbox}
Sea $a \in A$ y sea $|a| = n = p_1^{\alpha_1} \cdots p_k^{\alpha_k}$ (el orden de $a$ solo puede tener estos primos porque $|a|$ divide a $|A|$). Para cada $i = 1, \ldots, k$ sea $q_i = n/p_i^{\alpha_i}$. Es claro que ningún primo divide a la vez a todos los $q_i$, por lo que $\mcd(q_1, \ldots, q_k) = 1$ y por tanto existen $m_1, \ldots, m_k \in \mathbb{Z}$ tales que $m_1q_1 + \cdots + m_kq_k = 1$. Como $p_i^{\alpha_i} q_i a = na = 0$\footnote{Recordemos que si $n = |g|$ entonces $g^n = 1$, en notación aditiva $ng = 0$.}, se tiene $q_i a \in t_{p_i}(A)$, luego
\[
a = m_1q_1a + \cdots + m_kq_k a \in t_{p_1}(A) + \cdots + t_{p_k}(A).
\]
En consecuencia, $A = t_{p_1}(A) + \cdots + t_{p_k}(A)$.

Para ver que la suma es directa, supongamos que $a_1 + \cdots + a_k = 0$ con cada $a_i \in t_{p_i}(A)$. Por tanto, para cada $i = 1, \ldots, k$, existe $\beta_i$ tal que $p_i^{\beta_i} a_i = 0$. Sea $m = p_1^{\beta_1} \cdots p_k^{\beta_k}$. Para cada índice $i$ ponemos $t_i = m/p_i^{\beta_i}$, de modo que $t_i a_j = 0$ cuando $i \neq j$, y así
\[
t_i a_i = -t_i \sum_{j\neq i} a_j = -t_i \sum_{j=1}^{k} a_j = 0
\]
donde hemos introducido en la suma el factor $-t_i a_i = 0$. Entonces, como $t_i a_i = 0$, $|a_i|$ divide a $t_i$ y, de igual manera, $|a_i|$ divide a $p_i^{\beta_i}$. Como estos son coprimos, se tiene $|a_i| = 1$ y por tanto $a_i = 0$. Esto prueba que la familia es independiente.

Por último, de la igualdad $A = t_{p_1}(A) \oplus \cdots \oplus t_{p_k}(A)$ se deduce que $|A| = |t_{p_1}(A)| \cdots |t_{p_k}(A)|$. Como el orden de cada $t_{p_i}(A)$ es una potencia de $p_i$ (Lema \ref{lem:caracterizacion_p_grupos_finitos}) y cada $p_i$ divide a $|A|$, deducimos que ese orden es mayor que 1 y por tanto $t_{p_i}(A) \neq 0$.
\end{proofbox}

El siguiente corolario es inmediato:

\begin{corollary}{}{corolario_indescomponible_p_grupo}
Un grupo finito e indescomponible es un $p$-grupo para cierto primo $p$.
\end{corollary}

\begin{example}{Descomposición en suma directa de $p$-grupos}{ejemplos_descomposicion_p}
\begin{enumerate}
    \item Sea $n = p_1^{\alpha_1} \cdots p_k^{\alpha_k}$ una factorización prima irredundante del entero $n$. Por el Teorema Chino de los Restos, $\mathbb{Z}_n \cong \mathbb{Z}_{p_1}^{\alpha_1} \times \cdots \times \mathbb{Z}_{p_k}^{\alpha_k}$ y claramente los factores de esta descomposición van a corresponder con los factores $t_p(\mathbb{Z}_n)$ de la descomposición de la Proposición \ref{prop:descomposicion_primaria_p}. Más concretamente, si $q_i = n/p_i^{\alpha_i}$ para cada $i = 1, \ldots, k$, entonces $\overline{q_i} = q_i + n\mathbb{Z}$ genera un grupo de orden $p_i^{\alpha_i}$, y por tanto $t_{p_i}(\mathbb{Z}_n) = \langle \overline{q_i} \rangle$.
    
    \item Sea $B$ un anillo commutativo. Definimos en el producto cartesiano $B^* \times B$ la siguiente operación:
    \[
    (u,a)(v,b) = (uv,ub + va)
    \]
    Dejamos que el lector compruebe que esto define un grupo abeliano que denotamos $B^* \times B$ y que $(u,a)^n = (u^n, nu^{n-1}a)$, de lo que se deduce que $|(u,a)|$ es el mínimo común múltiplo del orden de $u$ en $(B^*,\cdot)$ y el orden de $a$ en $(B,+)$ pues, como $u$ es invertible, $nu^{n-1}a=0$ si y solo si $na=0$. Por tanto $t_{p}(B^{*}\times B)=t_{p}(B^{*})\times t_{p}(B)$ para todo primo $p$.
\end{enumerate}
\end{example}

Sea $B$ un subgrupo del grupo abeliano $A$, y sea $a\in A$. Si $na=0$ (con $n\in\mathbb{N}$), entonces, en $A/B$, se tiene $n(a+B)=0$. Eso implica que el orden de $a+B$ divide al orden de $a$. En general estos órdenes no coinciden; por ejemplo, no lo hacen si $a$ es un elemento no nulo de $B$.

\begin{lemma}{Propiedades de $p$-grupos finitos}{propiedades_p_grupos_finitos}
Sean $A$ un $p$-grupo finito. Entonces:

\begin{enumerate}
    \item Existe $a\in A$ tal que $|a|=\operatorname{Exp}(A)$.
    \item Si $B=\langle a\rangle$ (donde $a$ es el del apartado anterior) entonces todo elemento del cociente $A/B$ tiene un representante con el mismo orden. Es decir, para todo $\gamma\in A/B$ existe $x\in A$ tal que $x+B=\gamma$ y $|x|=|\gamma|$.
\end{enumerate}
\end{lemma}

\begin{proofbox}
\begin{enumerate}
\item Como $A$ es finito, su exponente es el mínimo común múltiplo de los órdenes de sus elementos. Como $A$ es un $p$-grupo finito, sus $k$ elementos tienen ordenes
\[
|a_1| = p^{n_1}, |a_2| = p^{n_2}, \dots, |a_k| = p^{n_k},
\]
por tanto, $\operatorname{Exp}(A)$ es una potencia de $p$. De hecho, es elemental ver que
\[
\Exp(A) = \mcm(p^{n_1},p^{n_2},\dots,p^{n_k}) = p^{m}
\]
con $m = n_i$ para algún $i\in\{1,\dots,k\}$. Por tanto, $a_i$ es el elemento buscado con $|a_i|=p^m=\Exp(A)$.

\item Sea $a$ como en (1), $|a| = p^m = \operatorname{Exp}(A)$. Sea $\gamma \in A/B$ y sea $y \in A$ tal que $y + B = \gamma$. Sean $|y| = p^s$, $|\gamma| = p^k$; sabemos que $k \leq s$ por la observación de antes de la Proposición y también $s \leq m$. Si $k = s$, tomamos $x = y$ y hemos terminado. Supongamos $k < s$. Como $p^k(y+B) = p^k \gamma = 0$, se tiene $p^k y \in B = \langle a \rangle$, luego $p^k y = q a$ para algún $q \in \mathbb{Z}$. Escribimos $q = r p^t$ con $\operatorname{mcd}(p,r) = 1$, es decir, separando la parte de la factorización de $q$ que contiene los factores con $p$ del resto. Entonces
\[
p^{m+k-t} y = p^{m-t} p^k y = p^{m-t} q a = r p^m a = 0,
\]
luego $s \leq m+k-t$. Por otro lado,
\[
p^{m+k-t-1} y = p^{m-t-1} q a = r p^{m-1} a \neq 0,
\]
de donde $s = m+k-t$. Tomemos $x = y - r p^{m-s} a$. Entonces $x+B = y+B = \gamma$, y
\[
p^k x = p^k y - r p^{m+k-s} a = p^k y - r p^t a = 0,
\]
mientras que $p^{k-1} x = p^{k-1} y - r p^{m+k-1-s} a \neq 0$ (pues $p^{k-1} y \notin B$). Por tanto $|x| = p^k = |\gamma|$.
\end{enumerate}
\end{proofbox}

Ahora podemos caracterizar los grupos abelianos finitos que son indescomponibles.

\begin{proposition}{Caracterización de grupos abelianos indescomponibles}{caracterizacion_indescomponibles}
Un grupo abeliano finito es indescomponible si y solo si es un $p$-grupo cíclico.
\end{proposition}

\begin{proofbox}
($\Leftarrow$) Ya hemos visto en el Ejemplo \ref{ex:p_grupo_ciclico} que todo $p$-grupo cíclico es indescomponible.

($\Rightarrow$) Sea $A$ indescomponible. Por el Corolario \ref{cor:corolario_indescomponible_p_grupo}, $A$ es un $p$-grupo para algún primo $p$. Probaremos por inducción sobre $|A|$ que $A$ es cíclico. Si $|A| = p$, entonces $A$ es cíclico por el Corolario \ref{cor:corolario_grupos_primo}.

Supongamos $|A| = p^N > p$ , por el Lema \ref{lem:propiedades_p_grupos_finitos}, existe $a \in A$ con $|a| = \Exp(A) = p^m$. Sea $B = \langle a \rangle$ y $C = A/B$. Por la Proposición \ref{prop:descomposicion_indescomponibles}, $C$ se descompone en suma directa de subgrupos indescomponibles, todos de orden menor que $p^N$, luego por hipótesis de inducción son cíclicos. Es decir, existen $x_1, \dots, x_r \in A$ tales que
\[
C = \langle x_1 + B \rangle \oplus \cdots \oplus \langle x_r + B \rangle,
\]
y por la parte (2) del Lema \ref{lem:propiedades_p_grupos_finitos} podemos suponer $|x_i| = |x_i + B|$ para cada $i$. Claramente $A = B + \langle x_1 \rangle + \cdots + \langle x_r \rangle$. Veamos que esta suma es directa. Si
\[
b + m_1 x_1 + \cdots + m_r x_r = 0 \quad (b \in B, \; m_i \in \mathbb{Z}),
\]
entonces en $C$ tenemos $m_1(x_1+B) + \cdots + m_r(x_r+B) = 0$, luego cada $m_i(x_i+B) = 0$. Por tanto $|x_i+B| = |x_i|$ divide a $m_i$, así que $m_i x_i = 0$ y en consecuencia $b = 0$. Esto prueba que
\[
A = B \oplus \langle x_1 \rangle \oplus \cdots \oplus \langle x_r \rangle.
\]
Como $A$ es indescomponible y $B \neq 0$, debemos tener $r = 0$, es decir $A = B = \langle a \rangle$, luego $A$ es cíclico.
\end{proofbox}

Combinando las Proposiciones \ref{prop:descomposicion_indescomponibles} y \ref{prop:caracterizacion_indescomponibles} se obtiene:

\begin{corollary}{Descomposición en $p$-grupos cíclicos}{descomposicion_p_ciclicos}
Todo grupo abeliano finito es suma directa de subgrupos cíclicos cada uno de los cuales tiene orden potencia de un primo.
\end{corollary}

\clearpage
\section{Descomposiciones primarias e invariantes}

El Corolario \ref{cor:descomposicion_p_ciclicos} va a ser fundamental para clasificar los grupos abelianos finitos salvo isomorfismos. La idea es que cada clase de isomorfía de grupos abelianos finitos estará dada por una lista de números que van a representar los cardinales de los factores que aparecen en una descomposición de cualquiera de los elementos de la clase como suma directa de grupos cíclicos. Vamos a elegir dos tipos de listas de números: En la primera los números que admitimos son potencias de primos; en la segunda los números van a ser números naturales arbitrarios pero con la exigencia de que cada uno de ellos divida a los anteriores.

\subsection{Descomposición primaria}

\begin{definition}{Descomposición primaria}{descomposicion_primaria}
Sea $A$ un grupo abeliano finito. Una descomposición primaria o indescomponible de $A$ es una expresión de $A$ como suma directa de subgrupos indescomponibles. Como cada uno de estos es un $p$-grupo cíclico para un primo $p$, siempre podemos reordenarlos de modo que se tenga
\begin{gather*}
A = \langle\alpha_{11}\rangle_{p_{1}^{\alpha_{11}}}\oplus\langle\alpha_{12}\rangle_{p_{1}^{\alpha_{12}}}\oplus\cdots\oplus\langle\alpha_{1m_{1}}\rangle_{p_{1}^{\alpha_{1m_{1}}}}\\
\cdots \\
\oplus\langle\alpha_{k1}\rangle_{p_{k}^{\alpha_{k1}}}\oplus\langle\alpha_{k2}\rangle_{p_{k}^{\alpha_{k2}}}\oplus\cdots\oplus\langle\alpha_{km_{k}}\rangle_{p_{k}^{\alpha_{km_{k}}}}
\end{gather*}
donde
\[
p_{1}<p_{2}<\cdots<p_{k}
\]
son los primos que dividen al orden de $A$ y ciertos enteros positivos $\alpha_{ij}$ con
\[
\alpha_{i1}\geq\alpha_{i2}\geq\cdots\geq\alpha_{im_{i}}\geq 1
\]
para cada $i=1,\ldots,k$.
\end{definition}

Con esta terminología, la Proposición \ref{prop:descomposicion_indescomponibles} se renuncia como:

\begin{theorem}{Existencia de descomposición primaria}{existencia_descomposicion_primaria}
Todo grupo abeliano finito tiene una descomposición primaria.
\end{theorem}

Para obtener una descomposición primaria de un grupo abeliano finito seguimos los pasos indicados en la sección anterior; es decir, dado un grupo abeliano finito $A$:

\begin{enumerate}
    \item Se calcula $t_{p}(A)$ para cada divisor primo de $|A|$; entonces $A=t_{p_{1}}(A)\oplus\cdots\oplus t_{p_{k}}(A)$ (Proposición \ref{prop:descomposicion_primaria_p}).
    \item Para cada divisor primo $p$ de $|A|$ se calcula $a\in t_{p}(A)$ tal que $|a|$ coincida con el periodo de $t_{p}(A)$ (Lema \ref{lem:propiedades_p_grupos_finitos}) y pasamos a estudiar $t_{p}(A)/\langle a\rangle$, que tiene orden menor que el de $t_{p}(A)$. Por recurrencia vamos pasando a grupos de orden cada vez más pequeño hasta obtener un grupo cíclico. Volvemos para atrás siguiendo la demostración de la Proposición \ref{prop:caracterizacion_indescomponibles} y así obtendremos una descomposición primaria de $t_{p}(A)$, que ocupará una fila en la ordenación de los sumandos según la Definición \ref{defn:descomposicion_primaria}.
\end{enumerate}

\subsection{Ejemplos de descomposiciones primarias}

\begin{example}{}{}
Consideremos el número complejo $\omega = \frac{-1+\sqrt{-3}}{2}$ y sea $G$ el subgrupo multiplicativo de $\mathbb{C}^*$ generado por $i$ y $\omega$:
\[
G = \langle i, \omega \rangle.
\]
Observamos que $|i| = 4$ y $|\omega| = 3$, es decir, $i$ es una raíz cuarta de la unidad y $\omega$ una raíz cúbica. Por el Teorema Chino de los Restos para grupos, tenemos que
\[
G = \langle i \rangle_4 \otimes \langle \omega \rangle_3,
\]
$t_2(G) = \langle i \rangle$ y $t_3(G) = \langle \omega \rangle$. Esta es la descomposición primaria de $G$.

Recordamos que al escribir $\otimes$ queremos indicar que $G = \langle i \rangle_4 \times \langle \omega \rangle_3$ y que $\langle i \rangle_4 \cap \langle \omega \rangle_3 = \{1\}$.
\end{example}
    
\begin{example}{}{}
Consideremos ahora $A = \mathbb{Z}_{560}^*$. Por el Teorema Chino de los Restos para Anillos tenemos que $\mathbb{Z}_{560} \cong \mathbb{Z}_{16} \otimes \mathbb{Z}_{5} \otimes \mathbb{Z}_{7}$ y por tanto $A \cong \mathbb{Z}_{16}^* \oplus \mathbb{Z}_{5}^* \oplus \mathbb{Z}_{7}^*$. Usando el Problema 4.5.4 deducimos que $|\mathbb{Z}_{16}^*| = \phi(16) = 8$, $|\mathbb{Z}_{5}^*| = \phi(5) = 4$ y $|\mathbb{Z}_{7}^*| = \phi(7) = 6$. Por tanto $\mathbb{Z}_{16}^*$ y $\mathbb{Z}_{5}^*$ son $2$-grupos de órdenes $8$ y $4$ respectivamente y la descomposición primaria de $\mathbb{Z}_{7}^*$ es una suma directa de un grupo de orden $2$ y otro de orden $3$. Luego $t_2(A) \cong \mathbb{Z}_{16}^* \times \mathbb{Z}_{5}^* \times t_2(\mathbb{Z}_{7}^*)$ y $t_3(A) \cong t_3(\mathbb{Z}_{7}^*) \cong C_3$.
\end{example}

\subsection{Descomposición invariante}

\begin{definition}{Descomposición invariante}{descomposicion_invariante}
Sea $A$ un grupo abeliano finito. Una descomposición invariante de $A$ es una expresión del tipo
\[
A = \bigoplus_{i=1}^{n} \langle a_i \rangle,
\]
tal que $|a_i|$ divide a $|a_{i-1}|$ para cada $i=2,\ldots,n$.
\end{definition}

Es fácil ver que el periodo de un grupo abeliano finito es igual al orden del primer sumando en una descomposición invariante suya (Problema 5.3.1).

Utilizando el Teorema 5.17 podemos obtener también:

\begin{theorem}{Existencia de descomposición invariante}{existencia_descomposicion_invariante}
Todo grupo abeliano finito tiene una descomposición invariante.
\end{theorem}

\begin{proofbox}
Sea $A$ un grupo abeliano finito. Añadiendo sumandos triviales a una descomposición primaria suya, tenemos
\[
A = \langle a_{11} \rangle_{p_1^{\alpha_{11}}} \oplus \langle a_{12} \rangle_{p_1^{\alpha_{12}}} \oplus \cdots \oplus \langle a_{1m_1} \rangle_{p_1^{\alpha_{1m}}} \oplus \cdots \langle a_{k_1} \rangle_{p_k^{\alpha_{k_1}}} \oplus \langle a_{k_2} \rangle_{p_k^{\alpha_{k_2}}} \oplus \cdots \oplus \langle a_{km_k} \rangle_{p_k^{\alpha_{km}}}
\]
para ciertos primos positivos distintos $p_1, p_2, \ldots, p_k$ y ciertos enteros $\alpha_{ij}$ que satisfacen
\[
\alpha_{i1} \geq \alpha_{i2} \geq \cdots \geq \alpha_{im} \geq 0 \quad \text{para todo } i = 1, \ldots, k. \tag{5.3}
\]
Los $\alpha_{ij}$ que valen cero se corresponden con los sumandos triviales que hemos añadido para que, en cada fila de la descomposición de $A$, a partir de la segunda, haya el mismo número de sumandos.

Para obtener la descomposición invariante basta con "agrupar los sumandos por columnas", a partir de la segunda fila. Explícitamente, para cada $j = 1, \ldots, m$, sean
\[
b_j = a_{1j} + a_{2j} + \cdots + a_{kj} \quad \text{y} \quad d_j = p_1^{\alpha_{1j}} p_2^{\alpha_{2j}} \cdots p_k^{\alpha_{kj}}.
\]

Por el Corolario 4.25 tenemos que
\[
\langle b_j \rangle_{d_j} = \langle a_{1j} \rangle_{p_1^{\alpha_{1j}}} \oplus \langle a_{2j} \rangle_{p_2^{\alpha_{2j}}} \oplus \cdots \oplus \langle a_{kj} \rangle_{p_k^{\alpha_{kj}}}.
\]
Entonces,
\[
A = \langle b_1 \rangle_{d_1} \oplus \langle b_2 \rangle_{d_2} \oplus \cdots \oplus \langle b_m \rangle_{d_m},
\]
es una descomposición invariante, pues como consecuencia de las desigualdades (5.3) se tiene que $d_j \mid d_{j-1}$ para todo $j = 2, \ldots, n$.
\end{proofbox}

La demostración del Teorema 5.20 nos dice cómo se obtiene una descomposición invariante a partir de una descomposición primaria.

\begin{definition}{Semejanza de descomposiciones}{semejanza_descomposiciones}
Sean $A$ y $B$ dos grupos abelianos finitos.

Dos descomposiciones primarias de $A$ y $B$ son \emph{semejantes} si los sumandos que intervienen son isomorfos dos a dos. Si ordenamos las descomposiciones como se ha indicado en la Definición 5.16, digamos

\[
A = (\bigoplus_{j=1}^{m_1} A_{1j}) \oplus \cdots \oplus (\bigoplus_{j=1}^{m_k} A_{kj})
\]

y

\[
B = (\bigoplus_{j=1}^{m'_1} B_{1j}) \oplus \cdots \oplus (\bigoplus_{j=1}^{m'_k} B_{kj}),
\]

es claro que éstas son semejantes si y sólo si $k = k'$, cada $m_i = m'_i$ y $|A_{ij}| = |B_{ij}|$ para cada posible par de índices.

Dos descomposiciones invariantes $A = \bigoplus_{i=1}^n A_i$ y $B = \bigoplus_{i=1}^n B_i$ son \emph{semejantes} si los sumandos que intervienen son isomorfos dos a dos, lo que claramente equivale a que tengan el mismo número de sumandos ($n = n'$) y las mismas listas de órdenes ($|A_i| = |B_i|$ para todo $i = 1, \ldots, n$).
\end{definition}

Es fácil ver que, si $A$ y $B$ tienen descomposiciones primarias (o invariantes) semejantes, entonces $A$ y $B$ son isomorfos. El siguiente teorema nos dice, esencialmente, que se verifica el recíproco:

\begin{theorem}{Unicidad de descomposiciones}{unicidad_descomposiciones}
Sea $A$ un grupo abeliano finito. Entonces:

\begin{enumerate}
    \item Todas las descomposiciones primarias de $A$ son semejantes.
    \item Todas las descomposiciones invariantes de $A$ son semejantes.
\end{enumerate}
\end{theorem}

\begin{proofbox}
En vista de que se puede pasar de una descomposición primaria a una invariante y viceversa, bastará con demostrar una de las dos afirmaciones. Demostraremos la primera.

Sea

\[
A = (\bigoplus_{j=1}^{m_1} A_{1j}) \oplus \cdots \oplus (\bigoplus_{j=1}^{m_k} A_{kj})
\]

una descomposición primaria de $A$ con $|A_{ij}| = p_i^{\alpha_{ij}}$ para ciertos enteros primos positivos $p_1 < p_2 < \cdots < p_k$ y ciertos enteros positivos $\alpha_{ij}$ con $\alpha_{i1} \geq \alpha_{i2} \geq \cdots \geq \alpha_{im_i} \geq 1$ para cada $i = 1, \ldots, k$. Obsérvese que para cada $i = 1, \ldots, k$, se tiene

\[
\bigoplus_{j=1}^{m_i} A_{ij} = t_p(A),
\]

por lo que estos subgrupos también están determinados por $A$. En consecuencia, podemos limitarnos a demostrar la unicidad asumiendo que $A$ es un $p$-grupo finito.

En esta situación, dos descomposiciones primarias de $A$ serán de la forma

\[
A = A_1 \oplus \cdots \oplus A_n = B_1 \oplus \cdots \oplus B_m,
\]

donde cada sumando es cíclico y, si ponemos $|A_i| = p^{\alpha_i}$ y $|B_i| = p^{\beta_i}$, se tiene $\alpha_1 \geq \alpha_2 \geq \cdots \geq \alpha_n$ y $\beta_1 \geq \beta_2 \geq \cdots \geq \beta_m$. Vamos a ver, por inducción en $i$, que $\alpha_i = \beta_i$ para cada $i$.

Obsérvese que $p^{\alpha_1} = \operatorname{Exp}(A) = p^{\beta_1}$, lo que resuelve el caso $i = 1$. Supongamos pues que $\alpha_j = \beta_j$ para cada $j = 1, \ldots, i - 1$, y veamos que $\alpha_i = \beta_i$. Podemos suponer sin pérdida de generalidad que $\alpha_i \leq \beta_i$.

Observemos lo siguiente: Sea $C$ un grupo cíclico de orden $p^r$ y sea $s \in \mathbb{N}$. Se tiene $p^s C = 0$ si y sólo si $s \geq r$. Por otra parte, si $s \leq r$, entonces $p^s C$ es cíclico de orden $p^{r-s}$ por la Proposición 4.22. En consecuencia, si ponemos $q = p^{\alpha_i}$, se tiene

\[
qA \cong qA_1 \oplus \cdots \oplus qA_{i-1} \cong (qB_1 \oplus \cdots \oplus qB_{i-1}) \oplus (qB_i \oplus \cdots \oplus qB_m).
\]

Como $qA_1 \oplus \cdots \oplus qA_{i-1}$ y $qB_1 \oplus \cdots \oplus qB_{i-1}$ tienen el mismo cardinal, deducimos que $qB_i \oplus \cdots \oplus qB_m = 0$. En particular $0 = qB_i = p^{\alpha_i} B_i$, de modo que $\alpha_i \geq \beta_i$, y por tanto $\alpha_i = \beta_i$, como queríamos ver.
\end{proofbox}

\begin{definition}{Divisores elementales y factores invariantes}{divisores_factores}
Sea $A$ un grupo abeliano finito. Sea

\[
A = \bigoplus_{i=1}^n \langle a_i \rangle_{m_i} \tag{5.4}
\]

una descomposición primaria ordenada como en la Definición 5.16. La lista $(m_1, \ldots, m_n)$ (que no depende de la descomposición primaria elegida, por el Teorema 5.24) se conoce como la lista de los \emph{divisores elementales} de $A$.

Análogamente, si (5.4) es una descomposición invariante, entonces la lista $(m_1, \ldots, m_n)$ (que tampoco depende de la descomposición invariante elegida) se conoce como la lista de los \emph{factores invariantes} de $A$.
\end{definition}

Todo lo visto en esta sección se resume en el siguiente Teorema:

\begin{theorem}{Teorema de Estructura de Grupos Abelianos Finitos}{teorema_estructura_grupos_abelianos_finitos}
\begin{enumerate}
    \item Todo grupo abeliano finito tiene una descomposición primaria y una descomposición invariante.
    
    \item Las siguientes condiciones son equivalentes para dos grupos abelianos finitos:
    \begin{enumerate}
        \item Son isomorfos.
        \item Tienen descomposiciones primarias semejantes.
        \item Tienen descomposiciones invariantes semejantes.
        \item Tienen la misma lista de divisores elementales.
        \item Tienen la misma lista de factores invariantes.
    \end{enumerate}
\end{enumerate}
\end{theorem}

Por el Teorema de Estructura 5.27, todo grupo abeliano finito es suma directa de cíclicos. Esto no es cierto para grupos abelianos en general, considérese $\mathbb{Q}$; ni siquiera para grupos abelianos de torsión.

El siguiente teorema que es una generalización del Teorema 5.27, sobrepasa los contenidos de esta asignatura. Necesitamos extender la noción de descomposiciones primaria e invariante y de descomposiciones semejantes. La única diferencia es que admitimos sumandos directos que sean cíclicos de orden infinito. Obsérvese que para que un grupo abeliano tenga una descomposición primaria o una descomposición invariante es necesario que sea finitamente generado pero además:

\begin{theorem}{Teorema de Estructura de Grupos Abelianos Finitamente Generados}{teorema_estructura_grupos_abelianos_fg}
\begin{enumerate}
    \item Todo grupo abeliano finitamente generado tiene una descomposición primaria y una descomposición invariante.
    
    \item Las siguientes condiciones son equivalentes para dos grupos abelianos finitos:
    \begin{enumerate}
        \item Son isomorfos.
        \item Tienen descomposiciones primarias semejantes.
        \item Tienen descomposiciones invariantes semejantes.
    \end{enumerate}
\end{enumerate}
\end{theorem}

De esta manera asociamos a cada grupo abeliano finitamente generado una lista de divisores elementales $(k; p_{1}^{\alpha_{11}}, \ldots, p_{1}^{\alpha_{1m_{1}}}, \ldots, p_{k}^{\alpha_{k1}}, \ldots, p_{k}^{\alpha_{km_{k}}})$ y una lista de factores invariantes $(k; d_1, \ldots, d_n)$, donde la única diferencia con el caso finito es que $k$ es el número de sumandos cíclicos infinitos bien en una descomposición primaria o en una descomposición invariante.