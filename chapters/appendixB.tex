\chapter{Preorden de divisibilidad}

\section{Relación de divisibilidad}

Sea $A$ un anillo conmutativo. Definimos la relación  divisibilidad como:
\[
a \preceq b \quad \text{si y solo si} \quad a \mid b
\]

\begin{proposition}{}{}
La relación $\preceq$ es un preorden, es decir, verifica las propiedades:
\begin{enumerate}
    \item Reflexiva: $a \preceq a$ para todo $a \in A$
    \item Transitiva: Si $a \preceq b$ y $b \preceq c$, entonces $a \preceq c$
\end{enumerate}
\end{proposition}

La relación no es antisimétrica, por lo que no es un orden parcial. De hecho $a \preceq b$ y $b \preceq a$ si y solo si $a$ y $b$ son asociados.

Definamos ahora unos cuantos conceptos importantes en conjuntos con un preorden.

\begin{definition}{Cota inferior}{}
Sea \((P, \preceq)\) un conjunto preordenado y \(S \subseteq P\). Un elemento \(c \in P\) es una cota inferior de \(S\) si \(c \preceq s\) para todo \(s \in S\).
\end{definition}

\begin{definition}{Cota superior}{}
Sea \((P, \preceq)\) un conjunto parcialmente ordenado y \(S \subseteq P\). Un elemento \(c \in P\) es una cota superior de \(S\) si \(s \preceq c\) para todo \(s \in S\).
\end{definition}

\begin{definition}{Ínfimo}{}
Sea \((P, \preceq)\) un conjunto preordenado y \(S \subseteq P\). El ínfimo de \(S\), denotado \(\inf S\), es la mayor cota inferior de \(S\), es decir, un elemento \(i \in P\) tal que:
\begin{enumerate}
    \item \(i\) es cota inferior de \(S\)
    \item Si \(c\) es cota inferior de \(S\), entonces \(c \preceq i\)
\end{enumerate}
\end{definition}

\begin{definition}{Supremo}{}
Sea \((P, \preceq)\) un conjunto preordenado y \(S \subseteq P\). El supremo de \(S\), denotado \(\sup S\), es la menor cota superior de \(S\), es decir, un elemento \(s \in P\) tal que:
\begin{enumerate}
    \item \(s\) es cota superior de \(S\)
    \item Si \(c\) es cota superior de \(S\), entonces \(s \preceq c\)
\end{enumerate}
\end{definition}

\section{Máximo común divisor y mínimo común múltiplo}

\begin{definition}{}{}
Sea $S \subseteq A$ un conjunto no vacío. Un máximo común divisor de $S$ es un elemento $d \in A$ tal que:
\begin{enumerate}
    \item $d \preceq s$ para todo $s \in S$ \quad ($d$ es cota inferior)
    \item Si $c \preceq s$ para todo $s \in S$, entonces $c \preceq d$ \quad ($d$ es la mayor cota inferior)
\end{enumerate}
\end{definition}

En términos del preorden de la divisibilidad, podemos expresar esto de manera mucho más simple
\[
\mcd(S) = \inf_{\preceq} S
\]

\begin{definition}{}{}
Sea $S \subseteq A$ un conjunto no vacío. Un mínimo común múltiplo de $S$ es un elemento $m \in A$ tal que:
\begin{enumerate}
    \item $s \preceq m$ para todo $s \in S$ \quad ($m$ es cota superior)
    \item Si $s \preceq c$ para todo $s \in S$, entonces $m \preceq c$ \quad ($m$ es la menor cota superior)
\end{enumerate}
\end{definition}

En términos del preorden:
\[
\mathrm{mcm}(S) = \sup_{\preceq} S
\]

Si definimos
\[
a \sim b \iff a \text{ es asociado de } b
\]
tenemos el siguiente resultado:
\begin{proposition}{}{}
Si $d$ y $d'$ son ambos $\mcd$ de $S$, entonces $d \sim d'$. Análogamente, si $m$ y $m'$ son ambos $\mcm$ de $S$, entonces $m \sim m'$.
\end{proposition}

\begin{proofbox}
Si $d$ y $d'$ son $\mcd$, entonces por definición:
\begin{itemize}
    \item $d \preceq s$ y $d' \preceq s$ para todo $s \in S$
    \item Como $d$ es mcd y $d'$ es divisor común: $d' \preceq d$
    \item Como $d'$ es mcd y $d$ es divisor común: $d \preceq d'$
\end{itemize}
Por tanto, $d \sim d'$. La demostración para el $\mcm$ es análoga.
\end{proofbox}

\begin{example}{}{}
En $\mathbb{Z}$, consideremos $S = \{6, 10\}$ con el preorden $a \preceq b \iff a \mid b$.
\begin{itemize}
    \item $\inf S = \{2, -2\}$
    \item $\sup S = \{30, -30\}$
\end{itemize}

Si representamos $a \preceq b$ mediante una flecha obtenemos el siguiente diagrama (ignorando asociados).
\[
\begin{tikzcd}
    & 1 \arrow[d] \arrow[dl] \arrow[dr] & \\
    \mathbf{2} \arrow[d] \arrow[drr] & 3 \arrow[dl] & 5 \arrow[d] \\
    6 \arrow[d] \arrow[dr] & & 10 \arrow[dl] \arrow[d]\\
    12 \arrow[dr] & \mathbf{30} \arrow[d] & 20 \arrow[dl] \\
    & 60 &
\end{tikzcd}
\]
\end{example}
