\chapter{Grupos de permutaciones}

En este capítulo estudiaremos dos de las familias más importantes de grupos: los grupos simétricos o de permutaciones y los grupos alternados.  
Recordemos que si \( A \) es un conjunto entonces \( S_A \) denota el grupo formado por las biyecciones de \( A \) en sí mismo con la composición como operación. Si \( B \) es un conjunto con el mismo cardinal que \( A \) entonces existe una biyección \( f : A \to B \) que se puede usar para dar un isomorfismo  
\[
S_A \quad \to \quad S_B, \qquad \sigma \quad \mapsto \quad f \sigma f^{-1}.
\]  
Por tanto, las propiedades de \( S_A \) sólo dependen del cardinal de \( A \). En particular, si \( n = |A| \) lo mismo nos da estudiar \( S_A \) que \( S_n = S_{N_n} \) donde \( N_n = \{1, 2, \ldots, n\} \).

\section{Ciclos y trasposiciones}

Describiremos a veces un elemento \( \sigma \in S_n \) dando la lista de sus imágenes en la forma  
\[
\begin{pmatrix}
1 & 2 & \cdots & n \\
\sigma(1) & \sigma(2) & \cdots & \sigma(n)
\end{pmatrix}.
\]
Sin embargo pronto veremos una forma más eficiente de representar los elementos de \( S_n \).

\begin{definition}{Conjunto de elementos movidos}{conjunto_movidos}
Diremos que una permutación \( \sigma \in S_n \) fija un entero \( i \in N_n \) si \( \sigma(i) = i \); en caso contrario diremos que \( \sigma \) cambia o mueve \( i \), y denotaremos por \( M(\sigma) \) al conjunto de los enteros cambiados por \( \sigma \):
\[
M(\sigma) = \{i \in N_n : \sigma(i) \neq i\}.
\]
\end{definition}

Es claro que \( M(\sigma) \) es vacío si y solo si \( \sigma = 1 \), y que \( M(\sigma) \) no puede tener exactamente un elemento.

Diremos que dos permutaciones \( \sigma \) y \( \tau \) de \( S_n \) son disjuntas si lo son los conjuntos \( M(\sigma) \) y \( M(\tau) \). Es decir, si todos los elementos que cambia una de ellas son fijados por la otra.

Cuando digamos que ciertas permutaciones \( \sigma_1, \ldots, \sigma_r \) son disjuntas entenderemos que lo son dos a dos.

\begin{lemma}{}{}
Una propiedad importante que usaremos con frecuencia es que $k \in M(\sigma) \implies \sigma(k)\in M(\sigma)$.
\end{lemma}

\begin{proofbox}
En efecto, como $\sigma$ es una biyección debe existir $\sigma^{-1}$ y entonces
\[
k \neq \sigma^{-1}(k) \iff \sigma(k) \neq k
\]
de aquí se deduce que $M(\sigma) = M(\sigma^{-1})$. Finalmente, si $k \in M(\sigma)$, entonces
\[
\sigma^{-1}(\sigma(k)) = k \neq \sigma(k),
\]
lo que implica que $\sigma(k) \in M(\sigma^{-1}) = M(\sigma)$.
\end{proofbox}

\begin{lemma}{Permutaciones disjuntas}{permutaciones_disjuntas}
Si \( \sigma \) y \( \tau \) son permutaciones disjuntas entonces \( \sigma\tau = \tau\sigma \) y \( M(\sigma\tau) = M(\sigma) \cup M(\tau) \).
\end{lemma}

\begin{proofbox}
Sea $k \in \N_n$, como las permutaciones son disjuntas podemos distinguir tres casos:
\begin{enumerate}
    \item $k \notin M(\sigma)\cup M(\tau)$
    \item $k \in M(\sigma)$
    \item $k \in M(\tau)$
\end{enumerate}
El caso (1) es el más fácil, puesto que se tiene $\sigma(k) = k = \tau(k)$, luego $(\sigma\tau)(k)=k=(\tau\sigma)(k)$.

En el caso (2), como las permutaciones son disjuntas
\[
\tau(k)=k \implies (\sigma\tau)(k)=\sigma(k)
\]
y por el lema anterior $\sigma(k)\in M(\sigma)$, por tanto, $\sigma(k)\notin M(\tau)$, luego $(\tau\sigma)(k)=\sigma(k)=(\sigma\tau)(k)$.

Para el caso (3) razonamos como en el caso (2).

Para la igualdad entre conjuntos, basta notar que por el razonamiento del caso (1)
\[
k \notin M(\sigma) \cup M(\tau) \implies k \notin M(\sigma\tau),
\]
el contrarrecíproco nos da $M(\sigma\tau) \subseteq M(\sigma) \cup M(\tau)$. Por otro lado, los argumentos de los casos (2) y (3) muestran que
\[
k \in M(\sigma) \implies \sigma\tau(k) = \sigma(k) \neq k,\quad k \in M(\tau) \implies \sigma\tau(k) = \tau(k) \neq k,
\]
luego $M(\sigma) \cup M(\tau) \subseteq M(\sigma\tau)$.
\end{proofbox}

\begin{definition}{Ciclo}{ciclo}
La permutación \( \sigma \in S_n \) es un ciclo de longitud \( s \) (o un \( s \)-ciclo) si \( M(\sigma) \) tiene \( s \) elementos y éstos pueden ordenarse de manera que se tenga \( M(\sigma) = \{i_1, i_2, \dots, i_s\} \) y
\[
\sigma(i_1) = i_2, \quad \sigma(i_2) = i_3, \quad \dots \quad \sigma(i_{s-1}) = i_s, \quad \sigma(i_s) = i_1.
\]
Este \( s \)-ciclo \( \sigma \) se denota como
\[
\sigma = (i_1 \; i_2 \; i_3 \; \dots \; i_s) \quad \text{ó} \quad \sigma = (i_1, i_2, i_3, \dots, i_s).
\]
Los \( 2 \)-ciclos también se llaman trasposiciones.
\end{definition}

Por ejemplo, los siguientes elementos de \( S_4 \) son ciclos de longitudes 2, 3 y 4, respectivamente:
\[
(1 \; 4) = \begin{pmatrix}
1 & 2 & 3 & 4 \\
4 & 2 & 3 & 1 
\end{pmatrix}, \quad 
(2 \; 4 \; 3) = \begin{pmatrix}
1 & 2 & 3 & 4 \\
1 & 4 & 2 & 3 
\end{pmatrix}, \quad 
(1 \; 3 \; 4 \; 2) = \begin{pmatrix}
1 & 2 & 3 & 4 \\
3 & 1 & 4 & 2 
\end{pmatrix}.
\]

\begin{example}{Acción de un ciclo}{}
Sea $\sigma = (2\ 4\ 3) \in S_4$, entonces $\sigma$ actúa sobre los elementos de $\N_4$ de la siguiente manera:
\[
\sigma(1) = 1,\ \sigma(2) = 4,\ \sigma(3) = 2,\ \sigma(4) = 3.
\]
Por otro lado, $\tau =(2\ 3\ 4)$ actua de manera que
\[
\tau(1) = 1,\ \tau(2) = 3,\ \tau(3) = 4,\ \tau(4) = 2.
\]
\end{example}

\begin{lemma}{Propiedades de los ciclos}{propiedades_ciclos}
Sea \( \sigma = (i_1 \; \dots \; i_s) \) un ciclo de longitud \( s \) en \( S_n \).
\begin{enumerate}
    \item Para cada \( t \in \{1, 2, \dots, s\} \) se tiene \( \sigma = (i_t \; i_{t+1} \; \dots \; i_s \; i_1 \; \dots \; i_{t-1}) \) y \( i_t = \sigma^{t-1}(i_1) \).
    \item El orden de \( \sigma \) (como elemento del grupo simétrico) coincide con su longitud \( s \).
\end{enumerate}
\end{lemma}

\begin{proofbox}
\begin{enumerate}
    \item Sea \( \sigma' = (i_t \; i_{t+1} \; \dots \; i_s \; i_1 \; \dots \; i_{t-1}) \). Sea $k \in \N_n$ arbitrario, si para todo $l=1,\dots,s$, $k \neq i_l$ entonces $\sigma'(k)=k=\sigma(k)$.
    
    Por otro lado, si para cierto $l$, $k = i_l$ entonces $\sigma'(k) = \sigma(k)$ ya que $\sigma', \sigma$ actúan igual sobre estos elementos. En conclusión, $\sigma' = \sigma$.

    Por otro lado, es claro que $i_1 = 1(i_1) = \sigma^0(i_1)$, si suponemos que $i_t = \sigma^{t-1}(i_1)$ para $t$ arbitrario entonces $i_{t+1} = \sigma(i_t) = \sigma(\sigma^t(i)) = \sigma^{t+1}(i_1)$ como queríamos ver.

    \item El orden de \( \sigma \), $k=|\sigma|$ es el menor natural tal que $\sigma^k = 1$. Si $\sigma$ tiene longitud $s$ entonces
    \[
    i_s = \sigma^{s-1}(i_1) \implies i_1 = \sigma^s(i_1)
    \]
    luego para cualquier $i_l$
    \[
    \sigma^s(i_l) = \sigma^{s+l-1}(i_1)=\sigma^{l-1}(\sigma^s(i_1))=\sigma^{l-1}(i_1)=i_l
    \]
    lo que prueba que $\sigma^s = 1$. Como $k$ es el orden de $\sigma$ debe ser $0 < k \leq s$, supongamos que $k < s$, entonces
    \[
    \sigma^k = 1 \implies i_{k+1} = \sigma^k(i_1) = i_1
    \]
    lo cual es imposible ya que $i_{k+1}\neq i_1$ si $0<k<s$, por tanto, ha de ser $k=s$.
\end{enumerate}
\end{proofbox}

\begin{theorem}{Factorización en ciclos disjuntos}{factorizacion_ciclos_disjuntos}
Toda permutación \( \sigma \neq 1 \) de \( S_n \) se puede expresar de forma única (salvo el orden) como producto de ciclos disjuntos.
\end{theorem}

\begin{proofbox}
Razonamos por inducción en \( |M(\sigma)| \). Como \( \sigma \neq 1 \), tenemos que \( |M(\sigma)| \geq 2 \), con lo que el primer caso es cuando \( M(\sigma) = \{i, j\} \) que implica que \( \sigma = (i \; j) \). Además si \( \sigma = \tau_1 \dots \tau_k \) con \( \tau_1, \dots, \tau_k \) ciclos disjuntos, del Lema \ref{lem:permutaciones_disjuntas} se deduce que \( k = 1 \), lo que demuestra la unicidad.

Supongamos que $|M(\sigma)| = n$ y que la propiedad se verifica para $m < n$, es decir, para toda permutación que mueve menos elementos que \( \sigma \). Fijemos \( i \in M(\sigma) \) y definamos recursivamente 
\[
i_0 = i, \quad i_j = \sigma(i_{j-1}).
\]
Como los \( i_j \) pertenecen a $M(\sigma)$, que es finito, debe existir $k > 0$ tal que $i_0 = i_k$, veamos por qué:
\begin{itemize}
    \item Como $M(\sigma)$ es finito debe existir un $k > 0$ tal que $i_k = i_l$ con $l < k$. De hecho, podemos tomar $k$ como el menor natural con esta propiedad, de manera que
    \[
    i_0, i_1, \dots, i_{k-1}
    \]
    son todos distintos entre sí y $i_k = i_l$ para algún $0 \leq l \leq k-1$.
    \item De hecho, debe ser $l = 0$. Si no fuera así, entonces $l>0$, luego
    \[
    i_{l-1} = \sigma^{-1}(i_l) = \sigma^{-1}(i_k) = i_{k-1}
    \]
    lo cual contradice que $k$ es el primer indice para el que $i_k$ coincide con alguno de los $i_j$ anteriores.
\end{itemize}
Entonces, \( i_0, i_1, \dots, i_{k-1} \) son distintos y \( i_0 = i_k = \sigma(i_{k-1}) \). Luego \( \tau = (i_0 \; i_1 \; \dots \; i_{k-1}) \) es un \( k \)-ciclo, donde es obvio que $k \leq n$. Definimos \( \rho \in S_n \) poniendo:
\[
\rho(j) = 
\begin{cases}
j, & \text{si } j \in \{i_0, i_1, \dots, i_{k-1}\} \text{ o } j \not\in M(\sigma); \\
\sigma(j), & \text{en caso contrario}.
\end{cases}
\]

Es fácil ver que \( \tau \) y \( \rho \) son disjuntas, que \( |M(\rho)| = |M(\sigma)| - k < |M(\sigma)| \) y \( \sigma = \tau\rho \). Aplicando la hipótesis de inducción deducimos que \( \rho = \rho_1 \dots \rho_l \) con \( \rho_1, \dots, \rho_l \) ciclos disjuntos dos a dos. Del Lema \ref{lem:permutaciones_disjuntas} deducimos que $M(\rho) = \bigcup_{i=1}^{l}M(\rho_i)$, luego
\[
M(\tau) \cap M(\rho_i) \subseteq M(\tau) \cap M(\rho) = \emptyset,
\]
o sea, \( \tau \) y \( \rho_i \) son disjuntos para todo \( i \). Por tanto \( \sigma = \tau\rho_1 \dots \rho_l \), un producto de ciclos disjuntos.

Por otro lado, si \( \sigma = \tau_1 \dots \tau_m \) con \( \tau_1, \dots, \tau_m \) ciclos disjuntos entonces \( i \in M(\tau_j) \) para un único \( j = 1, \dots, m \). Como los \( \tau_j \) conmutan podemos suponer que \( j = 1 \). Entonces \( \tau(i_l) = \sigma(i_l) = \tau_1(i_l) \), con lo que \( \tau = \tau_1 \). Luego \( \rho = \tau_2 \dots \tau_m \). Usando la unicidad de la factorización de \( \rho \) como producto de ciclos disjuntos, se deduce la unicidad de la de \( \sigma \).
\end{proofbox}

\begin{definition}{Tipo de una permutación}{def:tipo_permutacion}
El tipo de una permutación \( \sigma \neq 1 \) de \( S_n \) es la lista \( [s_1, \dots, s_k] \) de las longitudes de los ciclos que aparecen en su factorización en ciclos disjuntos, ordenadas en forma decreciente. Por convenio, la permutación identidad tiene tipo \( [1] \).
\end{definition}

Por ejemplo, el tipo de un \( s \)-ciclo es \( [s] \), el de la permutación \( (1 \; 2)(3 \; 4 \; 5)(6 \; 7) \in S_7 \) es \( [3, 2, 2] \), y el de la permutación de \( S_{11} \) del Ejemplo \ref{ex:factorizacion_permutacion} es \( [4, 3, 2] \).

La última demostración deja claro cómo obtener la factorización de una permutación como producto de ciclos disjuntos y en consecuencia su tipo. Lo ilustramos con un ejemplo:

% Pequeño ajuste en el número máximo de columnas de una matriz
\setcounter{MaxMatrixCols}{11}

\begin{example}{Factorización de una permutación como producto de ciclos disjuntos}{factorizacion_permutacion}
Consideremos la permutación de \( S_{11} \)
\[
\sigma = \begin{pmatrix}
1 & 2 & 3 & 4 & 5 & 6 & 7 & 8 & 9 & 10 & 11 \\
6 & 5 & 1 & 4 & 2 & 7 & 3 & 8 & 11 & 9 & 10
\end{pmatrix}.
\]

Elegimos un elemento arbitrario cambiado por \( \sigma \), por ejemplo el \( 1 \), y calculamos sus imágenes sucesivas por \( \sigma \):

\[
\sigma(1) = 6, \quad \sigma^2(1) = \sigma(6) = 7, \quad \sigma^3(1) = \sigma(7) = 3, \quad \sigma^4(1) = \sigma(3) = 1.
\]

Entonces \( (1 \; 6 \; 7 \; 3) \) es uno de los factores de \( \sigma \). Elegimos ahora un elemento de \( M(\sigma) \) que no haya aparecido aún, por ejemplo el \( 2 \), y le volvemos a seguir la pista, lo que nos da un nuevo factor \( (2 \; 5) \). Empezando ahora con el \( 9 \) obtenemos un tercer ciclo \( (9 \; 11 \; 10) \) que agota el proceso (el \( 4 \) y el \( 8 \) son fijados por \( \sigma \)) y nos dice que 
\[
\sigma = (1 \; 6 \; 7 \; 3)(2 \; 5)(9 \; 11 \; 10).
\]
Por tanto \( \sigma \) tiene tipo \( [4, 3, 2] \).
\end{example}
\begin{remark}
Por lo general, aunque podríamos escribir $\sigma = (1 \; 6 \; 7 \; 3)(2 \; 5)(9 \; 11 \; 10)(4)(8)$, omitimos los $1$-ciclos al factorizar una permutación. 
\end{remark}

El tipo de una permutación determina muchas de sus propiedades; por ejemplo determina su orden. Ya hemos visto que cualquier $s$-ciclo tiene orden $s$, y cuando obtenemos la factorización de una permutación cualquiera $\sigma$ en ciclos $\tau_i$, tenemos $\sigma^n = 1$ siempre que cada uno de los ciclos cumple $\tau_i^n = 1$.

\begin{proposition}{Orden de una permutación}{prop:orden_permutacion}
El orden de una permutación es el mínimo común múltiplo de las componentes de su tipo.
\end{proposition}

\begin{proofbox}
Sea \( \sigma = \tau_1 \dots \tau_k \) la factorización de una permutación \( \sigma \) como producto de ciclos disjuntos, y sea \( s_i \) la longitud del ciclo \( \tau_i \). Sea \( m \in \mathbb{N} \). Como los \( \tau_i \) conmutan entre sí (por ser disjuntos), se tiene \( \sigma^m = \tau_1^m \dots \tau_k^m \). Por otra parte, para cada \( i \) se tiene \( M(\tau_i^m) \subseteq M(\tau_i) \) y por tanto los \( \tau_i^m \) son disjuntos. Esto implica, por la unicidad en el Teorema \ref{thm:factorizacion_ciclos_disjuntos}, que \( \sigma^m = 1 \) precisamente si cada \( \tau_i^m = 1 \) para todo \( i \), lo cual pasa si y solo si \( s_i \mid m \) para todo \( i \), o equivalentemente si \( \operatorname{mcm}(s_1, \dots, s_k) \) divide a \( m \).
\end{proofbox}

\subsection{Conjugación en \( S_n \)}

A continuación vamos a describir las clases de conjugación de \( S_n \).

\begin{theorem}{Clases de conjugación en \( S_n \)}{clases_conjugacion_sn}
Dos elementos de \( S_n \) son conjugados precisamente si tienen el mismo tipo. En consecuencia, cada clase de conjugación de \( S_n \) está formada por todos los elementos de un mismo tipo.
\end{theorem}

\begin{proofbox}
Fijemos una permutación \( \alpha \). Para un \( s \)-ciclo \( \tau = (i_1 \; i_2 \; \dots \; i_s) \), se comprueba fácilmente que
\begin{equation}
\alpha \tau \alpha^{-1} = \alpha (i_1 \; i_2 \; \dots \; i_s) \alpha^{-1} = (\alpha(i_1) \; \alpha(i_2) \; \dots \; \alpha(i_s)), \label{eq:s_ciclo}.
\end{equation}
En efecto, dado $k \in \N_n$, si $k \neq \alpha(i_l)$ entonces $\alpha^{-1}(k)\neq i_l$, por lo que $\tau(\alpha^{-1}(k))=\alpha^{-1}(k)$, es decir,
\[
\alpha\tau\alpha^{-1}(k)=\alpha\alpha^{-1}(k)=k.
\]
Por otro lado, si $k = \alpha(i_l)$ entonces
\[
\alpha\tau\alpha^{-1}(k)=\alpha\tau(i_l)=\alpha(i_{l+1})
\]
sobreentendiendo que si $l=n$ en realidad es $\alpha(i_1)$. En cualquier caso, $\alpha\tau\alpha^{-1}$ actúa igual que $(\alpha(i_1) \; \alpha(i_2) \; \dots \; \alpha(i_s))$, por lo que ambos deben ser el mismo ciclo.

En particular, deducimos que \( \alpha \tau \alpha^{-1} \) es un \( s \)-ciclo. Usando la misma fórmula es fácil ver que, si dos ciclos \( \tau_1 \) y \( \tau_2 \) son disjuntos, entonces lo son \( \alpha \tau_1 \alpha^{-1} \) y \( \alpha \tau_2 \alpha^{-1} \). Como, en general, \( \alpha (\tau_1 \cdots \tau_k) \alpha^{-1} = (\alpha \tau_1 \alpha^{-1}) \cdots (\alpha \tau_k \alpha^{-1}) \), es claro que dos elementos conjugados de \( S_n \) tienen el mismo tipo.

Recíprocamente, supongamos que \( \sigma \) y \( \sigma' \) tienen el mismo tipo. Entonces las descomposiciones de \( \sigma \) y \( \sigma' \) en producto de ciclos disjuntos son de la forma \( \sigma = \tau_1 \tau_2 \cdots \tau_k \) y \( \sigma' = \tau_1' \tau_2' \cdots \tau_k' \), donde \( \tau_i \) y \( \tau_i' \) tienen la misma longitud. Por tanto, si \( \tau_i = (j_1 \; j_2 \; \dots \; j_s) \) y \( \tau_i' = (j_1' \; j_2' \; \dots \; j_s') \), entonces las aplicaciones \( \alpha_i : M(\tau_i) \rightarrow M(\tau_i') \) definidas por 
\[
\alpha_i(j_t) = j_t'
\]
para todo \( t \), son biyecciones. Además, como \( |M(\sigma)| = |M(\sigma')| \), existe una biyección \( \beta : N_n \setminus M(\sigma) \rightarrow N_n \setminus M(\sigma') \). Sea ahora \( \alpha \in S_n \) la biyección que se obtiene “pegando” las \( \alpha_i \) y \( \beta \). Es decir, \( \alpha(x) = \alpha_i(x) \) si \( x \in M(\tau_i) \) y \( \alpha(x) = \beta(x) \) si \( x \notin M(\sigma) \). De \eqref{eq:s_ciclo} se deduce que \( \tau_i' = \alpha \tau_i \alpha^{-1} \) para todo \( i \) y, por tanto \( \sigma' = \alpha \sigma \alpha^{-1} \).
\end{proofbox}

El siguiente corolario se deduce fácilmente del resultado anterior

\begin{corollary}{}{permutaciones_equivalentes}
La factorización en ciclos disjuntos de \( \alpha \sigma \alpha^{-1} \) se obtiene sustituyendo, en la de \( \sigma \), cada elemento \( i \) por \( \alpha(i) \). Por tanto la factorización de \( \sigma^\alpha = \alpha^{-1} \sigma \alpha \) se obtiene sustituyendo, en la de \( \sigma \), cada elemento \( i \in N_n \) por \( \alpha^{-1}(i) \).
\end{corollary}

Por ejemplo, si \( \alpha = (1 \; 4 \; 3)(2 \; 5 \; 6) \) y \( \sigma = (1 \; 3)(2 \; 4 \; 7) \), entonces \( \alpha \sigma \alpha^{-1} = (4 \; 1)(5 \; 3 \; 7) \) y \( \sigma^\alpha = (3 \; 4)(6 \; 1 \; 7) \).

\begin{example}{Clases de conjugación de \( S_3 \)}{clases_conjugacion_s3}
Las 6 permutaciones de \( S_3 \) se dividen en una permutación de tipo \( [1] \) (la identidad), tres 2-ciclos o permutaciones de tipo \( [2] \) (a saber, \( (1 \; 2) \), \( (1 \; 3) \) y \( (2 \; 3) \)), y dos 3-ciclos o permutaciones de tipo \( [3] \) (a saber, \( (1 \; 2 \; 3) \) y \( (1 \; 3 \; 2) \)).
\end{example}

\begin{example}{Clases de conjugación de \( S_4 \)}{clases_conjugacion_s4}
En \( S_4 \) hay más variedad, y en particular aparecen permutaciones que no son ciclos. Sus 24 permutaciones se dividen en los siguientes tipos:

\begin{center}
\begin{tabular}{c|l}
Tipo & Permutaciones \\
\hline
$[1]$ & $1$ \\
$[2]$ & $(1 \; 2), (1 \; 3), (1 \; 4), (2 \; 3), (2 \; 4), (3 \; 4)$ \\
$[3]$ & $(1 \; 2 \; 3), (1 \; 3 \; 2), (1 \; 2 \; 4), (1 \; 4 \; 2), (1 \; 3 \; 4), (1 \; 4 \; 3), (2 \; 3 \; 4), (2 \; 4 \; 3)$ \\
$[4]$ & $(1 \; 2 \; 3 \; 4), (1 \; 2 \; 4 \; 3), (1 \; 3 \; 2 \; 4), (1 \; 3 \; 4 \; 2), (1 \; 4 \; 2 \; 3), (1 \; 4 \; 3 \; 2)$ \\
$[2,2]$ & $(1 \; 2)(3 \; 4), (1 \; 3)(2 \; 4), (1 \; 4)(2 \; 3)$
\end{tabular}
\end{center}

Por tanto, cada fila de elementos a la derecha de la barra es una clase de conjugación de \( S_4 \).
\end{example}

\begin{example}{Clases de conjugación de \(S_5\) y \(S_6\)}{clases_conjugacion_s5_s6}
Además de los ciclos, en \( S_5 \) hay permutaciones de los tipos \( [2,2] \) y \( [3,2] \); y en \( S_6 \) las hay de los tipos \( [2,2] \), \( [3,2] \), \( [2,2,2] \), \( [3,3] \) y \( [4,2] \). En estos casos, por el gran número de elementos en los grupos, es pesado construir tablas como la que acabamos de dar para \( S_4 \), pero se puede calcular cuántas permutaciones hay de cada tipo.
\end{example}

\begin{proposition}{Conjuntos generadores de \( S_n \)}{generadores_sn}
Para \( n \geq 2 \), los siguientes son conjuntos generadores de \( S_n \):
\begin{enumerate}
    \item El conjunto de todos los ciclos.
    \item El conjunto de todas las trasposiciones.
    \item El conjunto de \( n-1 \) trasposiciones: \( \{(1 \; 2), (1 \; 3), (1 \; 4), \dots, (1 \; n-1), (1 \; n)\} \).
    \item El conjunto de \( n-1 \) trasposiciones: \( \{(1 \; 2), (2 \; 3), (3 \; 4), \dots, (n-1 \; n)\} \).
    \item El conjunto de una trasposición y un \( n \)-ciclo: \( \{(1 \; 2), (1 \; 2 \; 3 \; \dots \; n-1 \; n)\} \).
\end{enumerate}
\end{proposition}

\begin{proofbox}
\begin{enumerate}
    \item Es una consecuencia inmediata del Teorema \ref{thm:factorizacion_ciclos_disjuntos}.
    
    Para demostrar el resto de apartados bastará con comprobar que los elementos del conjunto dado en cada apartado se expresan como productos de los elementos del conjunto del apartado siguiente.
    
    \item Cada ciclo \( \sigma = (i_1 \; i_2 \; \dots \; i_s) \) puede escribirse como producto de trasposiciones (no disjuntas):
    \[
    \sigma = (i_1 \; i_s)(i_1 \; i_{s-1}) \cdots (i_1 \; i_3)(i_1 \; i_2).
    \]
    
    \item Es consecuencia de la igualdad \( (i \; j) = (1 \; i)(1 \; j)(1 \; i) \).
    
    \item Dado \( j \geq 2 \), sea \( \alpha = (2 \; 3)(3 \; 4)(4 \; 5) \cdots (j-1 \; j) \). Como $\alpha^{-1}(1)=1, \alpha^{-1}(2)=j$, por el Corolario \ref{cor:permutaciones_equivalentes} deducimos que \( (1 \; 2)^\alpha = (1 \; j) \).
    
    \item Sean \( \tau = (1 \; 2) \) y \( \sigma = (1 \; 2 \; \dots \; n-1 \; n) \). Como \( \sigma^{j-1}(1) = j \) y \( \sigma^{j-1}(2) = j+1 \), un argumento similar al del Corolario \ref{cor:permutaciones_equivalentes} nos permite afirmar que \( \sigma^{j-1} \tau \sigma^{1-j} = (j \; j+1) \).
\end{enumerate}
\end{proofbox}

\begin{corollary}{Subgrupo con trasposición y \( p \)-ciclo}{cor:subgrupo_trasposicion_p-ciclo}
Sean \( p \) un número primo y \( H \) un subgrupo de \( S_p \). Si \( H \) contiene una trasposición y un \( p \)-ciclo, entonces \( H = S_p \).
\end{corollary}

\begin{proofbox}
Podemos suponer que \( H \) contiene a \( (1 \; 2) \) y un \( p \)-ciclo \( \sigma = (a_1 \; a_2 \; \dots \; a_p) \). Por el Lema \ref{lem:propiedades_ciclos}, podemos suponer que \( a_1 = 1 \). Como $p$ es primo, $\sigma^k$ es un $p$-ciclo para todo $1 \leq k < p$.

Por tanto, si \( a_i = 2 \), entonces \( \sigma^{i-1} = (1 \; 2 \; b_3 \; \dots \; b_p) \), y podemos renombrar los \( b_i \) de forma que \( b_i = i \), notemos que hay que renombrar una vez que hemos asegurado que los dos primeros elementos de $\sigma$ son los mismos que aparecen en la trasposición $(1 \; 2)$. Por tanto \( (1 \; 2), (1 \; 2 \; \dots \; p) \in H \). Deducimos de la Proposición \ref{prop:generadores_sn} que \( H = S_p \).
\end{proofbox}

Aunque toda permutación de \( S_n \) se puede expresar como un producto de trasposiciones, estas expresiones no tienen las buenas propiedades que vimos en las descomposiciones en ciclos disjuntos. Por una parte, no podemos esperar que una permutación arbitraria sea producto de trasposiciones disjuntas (tendría orden 2). Por otra, tampoco se tiene conmutatividad (por ejemplo, \( (1 \; 3)(1 \; 2) \neq (1 \; 2)(1 \; 3) \)) ni unicidad, ni siquiera en el número de factores; por ejemplo
\[
(1 \; 2 \; 3) = (1 \; 3)(1 \; 2) = (2 \; 3)(1 \; 3) = (1 \; 3)(2 \; 4)(1 \; 2)(1 \; 4) = (2 \; 3)(2 \; 3)(1 \; 3)(2 \; 4)(1 \; 2)(1 \; 4).
\]
Nótese que en todas estas factorizaciones de \( (1 \; 2 \; 3) \) hay un número par de trasposiciones; esto es consecuencia de un hecho general que analizaremos en la sección siguiente.

\clearpage
\section{El grupo alternado}

Fijemos un entero positivo \( n \geq 2 \) y una permutación \( \sigma \in S_n \). Definamos un homomorfismo de anillos \( \bar{\sigma} : \mathbb{Z}[X_1, \dots, X_n] \to \mathbb{Z}[X_1, \dots, X_n] \) tal que \( \bar{\sigma}(X_i) = X_{\sigma(i)} \) para cada \( i \). Este homomorfismo simplemente intercambia unas indeterminadas con otras siguiendo la permutación $\sigma$, es decir, dado un polinomio \( Q \), su imagen \( \bar{\sigma}(Q) \) se obtiene sustituyendo cada \( X_i \) por \( X_{\sigma(i)} \) en la expresión de \( Q \).

En lo que sigue, \( P \) designará al polinomio de \( \mathbb{Z}[X_1, \dots, X_n] \) dado por
\[
P = \prod_{1 \leq i < j \leq n} (X_j - X_i).
\]
Por ejemplo, en $\Z[X,Y,Z]$,
\[
P = (Z-X)(Y-X)(Z-Y).
\]
Si \( i \) y \( j \) son dos elementos distintos de \( N_n \), en el producto anterior aparece \( X_i - X_j \) ó \( X_j - X_i \) pero no los dos. Como \( \bar{\sigma} \) es un homomorfismo de anillos, se tiene
\[
\bar{\sigma}(P) = \prod_{i<j} \bar{\sigma}(X_j - X_i) = \prod_{i<j} (X_{\sigma(j)} - X_{\sigma(i)}).
\]
Como \( \sigma \) es una biyección, de nuevo para cada dos elementos distintos \( i \) y \( j \) de \( N_n \) en el producto anterior aparece \( X_{\sigma(i)} - X_{\sigma(j)} \) ó \( X_{\sigma(j)} - X_{\sigma(i)} \) pero no ambos. Más concretamente, si \( i < j \) entonces en \( P \) se da uno de los dos siguientes casos:
\begin{itemize}
    \item Que sea \( \sigma(i) < \sigma(j) \), en cuyo caso el factor \( X_{\sigma(j)} - X_{\sigma(i)} \) aparece en \( \bar{\sigma}(P) \) igual que en \( P \).
    \item Que sea \( \sigma(i) > \sigma(j) \), en cuyo caso el factor \( X_{\sigma(j)} - X_{\sigma(i)} \) aparece en \( \bar{\sigma}(P) \) pero en \( P \) aparece su opuesto; en este caso diremos que \( \sigma \) presenta una inversión para el par \( (i,j) \).
\end{itemize}
Como cada inversión se traduce en un cambio de signo en \( \bar{\sigma}(P) \) con respecto a \( P \), se tiene \( \bar{\sigma}(P) = \pm P \), donde el signo es \( + \) si y solo si el número de pares \( (i,j) \) (con \( i < j \)) para los que \( \sigma \) presenta una inversión es par. Esto sugiere las definiciones que siguen.

\begin{definition}{Permutación par e impar}{permutacion_par_impar}
La permutación \( \sigma \in S_n \) es par si \( \bar{\sigma}(P) = P \); es decir, si \( \sigma \) presenta un número par de inversiones; y es impar si \( \bar{\sigma}(P) = -P \); es decir, si \( \sigma \) presenta un número impar de inversiones.

El signo de \( \sigma \) se define como \( \operatorname{sgn}(\sigma) = (-1)^k \), donde \( k \) es el número de inversiones que presenta \( \sigma \). Es decir, \( \operatorname{sgn}(\sigma) = 1 \) si \( \sigma \) es par y \( \operatorname{sgn}(\sigma) = -1 \) si \( \sigma \) es impar. Por el comentario previo a esta definición se tiene \( \bar{\sigma}(P) = \operatorname{sgn}(\sigma) P \).
\end{definition}

De hecho, el uso del anillo de polinomios en varias variables no es estrictamente necesario puesto que la única información relevante que extraemos tiene que ver con los índices de las indeterminadas. Si simplemente consideramos el orden natural en $\N_n$, entonces, dados $i,j\in\N_n$ distintos, o bien $i < j$ o $j < i$. Podemos considerar por simplicidad que $i < j$, la pregunta ahora es si $\sigma$ respeta el orden de esta pareja, es decir
\[
\sigma(i) < \sigma(j) \text{ o bien } \sigma(i) > \sigma(j).
\]
Podemos estudiar el efecto de $\sigma$ sobre todas las parejas $i < j$, de lo cual deduciremos que invierte el orden en un número de parejas $N$, para determinar el orden basta ver si $N$ es par o impar. Este acercamiento a la paridad es algo más simple, aunque menos operativo a la hora de demostrar propiedades interesantes.

\begin{example}{}{}
Para estudiar el signo de $\sigma = (1\; 3\; 2\; 4) \in S_4$ notemos que cumple:
\begin{gather*}
\sigma(1) = 3 < 4 = \sigma(2),\quad \sigma(1) = 3 > 2 = \sigma(3),\quad \sigma(1) = 3 > 1 = \sigma(4)\\
\sigma(2) = 4 > 2 = \sigma(3),\quad \sigma(2) = 4 > 1 = \sigma(4)\\
\sigma(3) = 2 > 1 = \sigma(4)
\end{gather*}
por tanto $\sigma$ invierte el orden en $5$ casos, luego es una permutación impar.
\end{example}

\begin{proposition}{El signo es un homomorfismo}{signo_homomorfismo}
La "aplicación signo" \( \operatorname{sgn} : S_n \to \mathbb{Z}^* = \{1, -1\} \) es un homomorfismo de grupos.
\end{proposition}

\begin{proofbox}
Sean \( \sigma, \tau \in S_n \). Es claro que \( \overline{\sigma \circ \tau} = \bar{\sigma} \circ \bar{\tau} \), por tanto
\[
\operatorname{sgn}(\sigma \circ \tau) P = \overline{\sigma \circ \tau}(P) = \bar{\sigma}(\bar{\tau}(P)) = \bar{\sigma}(\operatorname{sgn}(\tau) P) = \operatorname{sgn}(\tau) \bar{\sigma}(P) = \operatorname{sgn}(\tau) \operatorname{sgn}(\sigma) P,
\]
usando que $\bar\sigma$ es homomorfismo de anillos, luego \( \operatorname{sgn}(\sigma \circ \tau) = \operatorname{sgn}(\sigma) \operatorname{sgn}(\tau) \).
\end{proofbox}

\begin{proposition}{Propiedades del signo}{propiedades_signo}
En \( S_n \) se verifica:
\begin{enumerate}
    \item El signo de una permutación \( \sigma \) es el mismo que el de su inversa \( \sigma^{-1} \) y que el de cualquiera de sus conjugadas \( \sigma^\alpha \).
    \item Toda trasposición es impar.
    \item Si \( \sigma = \tau_1 \cdots \tau_r \), donde las \( \tau_i \) son trasposiciones, entonces \( \operatorname{sgn}(\sigma) = (-1)^r \).
    \item Una permutación \( \sigma \) es par (respectivamente impar) si y solo si es producto de un número par (respectivamente impar) de trasposiciones.
    \item Un ciclo de longitud \( s \) tiene signo \( (-1)^{s-1} \); es decir, un ciclo de longitud par es impar, y viceversa.
    \item La paridad de una permutación coincide con la del número de componentes pares de su tipo.
\end{enumerate}
\end{proposition}

\begin{proofbox}
\begin{enumerate}
    \item Basta usar que $\operatorname{sgn}$ es un homomorfismo de grupos y que los elementos de $\Z^*$ son sus propios inversos, luego $\operatorname{sgn}(\sigma^{-1}) = \operatorname{sgn}(\sigma)^{-1} = \operatorname{sgn}(\sigma)$. Para las conjugadas
    \[
    \operatorname{sgn}(\alpha^{-1}\sigma\alpha) = \operatorname{sgn}(\alpha^{-1})\operatorname{sgn}(\sigma)\operatorname{sgn}(\alpha) = \operatorname{sgn}(\alpha)^2\operatorname{sgn}(\sigma) = \operatorname{sgn}(\sigma)
    \] 
    usando de nuevo que los elementos de $\Z^*$ son sus propios inversos ($a^2 = 1$).

    \item Solo hay que fijarse en que toda transposición tiene un número impar de inversiones.

    \item Como $\operatorname{sgn}$ es un homomorfismo de grupos
    \[
    \operatorname{sgn}\sigma = \operatorname{sgn}(\tau_1) \cdots \operatorname{sgn}(\tau_r) = (-1)^r
    \]
    ya que $\operatorname{sgn}(\tau_i) = -1$ por (2).

    \item Dada una permutación cualquiera $\sigma$ podemos factorizarla en trasposiciones por la Proposición \ref{prop:generadores_sn}. Como $\operatorname{sgn}$ es un homomorfismo, la permutación es par si y solo si tiene un numero par de trasposiciones.
    
    \item Si el ciclo es $\sigma = (i_1 \; \dots \; i_s)$ podemos factorizarlo como
    \[
    \sigma = (i_1 \; i_s)(i_1 \; i_{s-1}) \cdots (i_1 \; i_3)(i_1 \; i_2)
    \]
    es decir, como producto de $s-1$ trasposiciones, luego $\operatorname{sgn}(\sigma) = (-1)^{s-1}$.

    \item Por (5) un $k$-ciclo es par si $k$ es impar, e impar si $k$ es par (para recordar esto basta notar que las trasposiciones son impares). Por tanto, al factorizar una permutación el signo solo depende de los ciclos pares, si hay un número par de ellos la permutación será par, en caso contrario será impar. 
\end{enumerate}
\end{proofbox}

\begin{example}{Calculando la paridad en función del tipo}{paridad_tipo}
De los Ejemplos \ref{ex:clases_conjugacion_s3}, \ref{ex:clases_conjugacion_s4}, \ref{ex:clases_conjugacion_s5_s6} y del último apartado de la Proposición \ref{prop:propiedades_signo} se deduce que, además de la identidad, las permutaciones pares de \( S_3 \) son las de tipo \( [3] \); las de \( S_4 \) son las de los tipos \( [3] \) ó \( [2,2] \); las de \( S_5 \) son las de los tipos \( [3] \), \( [5] \), o \( [2,2] \); y las de \( S_6 \) son las de los tipos \( [3] \), \( [5] \), \( [2,2] \), \( [4, 2]\) ó \( [3,3] \).
\end{example}

\begin{definition}{Grupo alternado}{grupo_alternado}
El grupo alternado en \( n \) elementos, denotado por \( A_n \), es el núcleo del homomorfismo \( \operatorname{sgn} : S_n \to \mathbb{Z}^* = \{1, -1\} \). Es decir, es el subgrupo de \( S_n \) formado por las permutaciones pares.
\end{definition}

\begin{proposition}{Propiedades de \( A_n \)}{propiedades_an}
\( A_n \) es un subgrupo normal de \( S_n \), y para \( n \geq 2 \) se tiene:
\[
[S_n : A_n] = 2, \quad |A_n| = \frac{n!}{2}, \quad \text{y} \quad \frac{S_n}{A_n} \simeq \{1, -1\} \simeq \mathbb{Z}_2.
\]
\end{proposition}

\begin{proofbox}
Al estar definido como el núcleo de un homomorfismo, \( A_n \) es normal en \( S_n \). El resto es consecuencia del Primer Teorema de Isomorfía si vemos que, para \( n \geq 2 \), el homomorfismo \( \operatorname{sgn} \) es supravectivo, para lo que basta notar que \( \operatorname{sgn}(1) = 1 \) y \( \operatorname{sgn}((1 \; 2)) = -1 \).
\end{proofbox}

Es elemental ver que \( A_2 \) es el grupo trivial y que \( A_3 \) es el subgrupo cíclico de \( S_3 \) generado por el 3-ciclo \( (1 \; 2 \; 3) \), y por tanto \( A_3 \cong C_3 \). En el caso general, tenemos dos maneras sencillas de describir conjuntos de generadores de \( A_n \).

\begin{proposition}{Sistemas de generadores de \( A_n \)}{generadores_an}
Los siguientes son sistemas de generadores de \( A_n \):
\begin{enumerate}
    \item El conjunto de todos los productos de dos trasposiciones (disjuntas o no).
    \item El conjunto de todos los 3-ciclos.
\end{enumerate}
\end{proposition}

\begin{proofbox}
\begin{enumerate}
    \item Por el apartado (4) de la Proposición \ref{prop:propiedades_signo}, una permutación está en $A_n$ si y solo si es producto de un número par de trasposiciones, agrupando este número par en parejas de dos trasposiciones deducimos que cualquier permutación par es producto de elementos de la forma $\tau = (a\; b)(c\; d)$.
    
    \item Por la misma Proposición, todos los 3-ciclos están en \( A_n \); por tanto, usando (1) para demostrar (2) solo hay que probar que cada producto de dos trasposiciones distintas (disjuntas o no) se puede escribir como producto de 3-ciclos, lo que se sigue de las igualdades
    \[
    (i \; j)(i \; k) = (i \; k \; j) \quad \text{e} \quad (i \; j)(k \; l) = (j \; l \; k)(i \; k \; j),
    \]
    donde asumimos que \( i, j, k, l \) son distintos dos a dos.
\end{enumerate}
\end{proofbox}
\begin{remark}
Observese que, como el conjunto vacío genera el subgrupo trivial, la Proposición \ref{prop:generadores_an} es válida incluso cuando \( n = 1 \) ó \( n = 2 \).
\end{remark}

A continuación describimos los subgrupos de \( A_4 \). Esto nos dará un ejemplo en el que no se verifica el recíproco del Teorema de Lagrange: \( A_4 \) tiene orden 12, pero no tiene subgrupos de orden 6.

\begin{example}{Subgrupos de \( A_4 \)}{subgrupos_a4}
En virtud del Ejemplo \ref{ex:paridad_tipo}, la siguiente es la lista completa de los elementos de \( A_4 \):
\begin{align*}
& 1, \quad \sigma = (1 \; 2)(3 \; 4), \quad \tau = (1 \; 3)(2 \; 4), \quad \eta = (1 \; 4)(2 \; 3), \\
& \alpha = (1 \; 2 \; 3), \quad \beta = (1 \; 2 \; 4), \quad \gamma = (1 \; 3 \; 4), \quad \delta = (2 \; 3 \; 4), \\
& \alpha^2 = (1 \; 3 \; 2), \quad \beta^2 = (1 \; 4 \; 2), \quad \gamma^2 = (1 \; 4 \; 3), \quad \delta^2 = (2 \; 4 \; 3).
\end{align*}

Por el Teorema de Lagrange, los subgrupos propios y no triviales de \( A_4 \) han de tener orden 2, 3, 4, ó 6. Los de orden 2 han de ser cíclicos y estar generados por elementos de orden 2, y por tanto son:
\[
\langle \sigma \rangle = \{1, \sigma\}, \quad \langle \tau \rangle = \{1, \tau\}, \quad \langle \eta \rangle = \{1, \eta\}.
\]
Como \( \sigma^\alpha = \tau \notin \langle \sigma \rangle \), deducimos que \( \langle \sigma \rangle \) no es normal en \( A_4 \), y del mismo modo se ve que no lo son \( \langle \tau \rangle \) ni \( \langle \eta \rangle \). Los subgrupos de orden 3 también deben ser cíclicos y han de estar generados por elementos de orden 3, por tanto son:
\begin{align*}
\langle \alpha \rangle = \langle \alpha^2 \rangle &= \{1, \alpha, \alpha^2\}, \\
\langle \beta \rangle = \langle \beta^2 \rangle &= \{1, \beta, \beta^2\}, \\
\langle \gamma \rangle = \langle \gamma^2 \rangle &= \{1, \gamma, \gamma^2\}, \\
\langle \delta \rangle = \langle \delta^2 \rangle &= \{1, \delta, \delta^2\}.
\end{align*}

Un subgrupo de orden 4 no puede contener a ninguno de los elementos de orden 3; como el resto de elementos forman un subgrupo
\[
N = \{1, \sigma, \tau, \eta\},
\]
este es el único subgrupo de orden 4, que además es normal en \( S_n \) por el Teorema \ref{thm:clases_conjugacion_sn}.

Por último, veamos que no hay subgrupos de orden 6. Un tal subgrupo \( H \) sería normal en \( A_4 \) por tener índice 2. Además, debería contener algún $3$-ciclo (puesto que solo hay 4 elementos que no son $3$-ciclos). Entonces, $H$ debería contener a todos los tres ciclos por ser normal y contener a uno de ellos, ya que todos los elementos con el mismo tipo son conjugados (Teorema \ref{thm:clases_conjugacion_sn}), pero esto es claramente una contradicción ya que hay 8 $3$-ciclos distinos y $H$ tiene solo 6 elementos.
\end{example}

\clearpage
\section{El Teorema de Abel}

\begin{definition}{Grupo simple}{def:grupo_simple}
Un grupo no trivial \( G \) es simple si sus únicos subgrupos normales son \( \{1\} \) y \( G \).
\end{definition}

Como consecuencia inmediata de la Proposición \ref{prop:grupos_ciclicos} se tiene que un grupo abeliano es simple si y solo si tiene orden primo. Para ver esto notemos que si es abeliano y simple, entonces cualquier elemento $x\neq 1$ debe generarlo, por lo que es cíclico, como no puede tener subgrupos normales propios, debe ser de orden primo. Por otro lado, si tiene orden primo es claramente abeliano y simple, puesto que debe ser isomorfo a $\Z_p$.

Obsérvese que \( A_3 \cong C_3 \) es simple, pero \( A_4 \) no lo es, como muestra el Ejemplo \ref{ex:subgrupos_a4}. El Teorema de Abel demuestra que 4 es el único número para el que pasa esto.

\begin{lemma}{Subgrupo normal que contiene un $3$-ciclo}{normal_con_3-ciclo}
Si un subgrupo normal \( H \) de \( A_n \) (\( n \geq 5 \)) contiene un $3$-ciclo, entonces \( H = A_n \).
\end{lemma}

\begin{proofbox}
Sea \( \sigma \) un 3-ciclo en \( H \). Por la Proposición \ref{prop:generadores_an}, basta ver que cualquier otro 3-ciclo \( \sigma' \) está en \( H \). Sabemos por el Teorema \ref{thm:clases_conjugacion_sn} que existe \( \alpha \in S_n \) tal que \( \sigma' = \sigma^\alpha \), de modo que si \( \alpha \in A_n \) entonces \( \sigma' \in H \), por la normalidad de \( H \) en \( A_n \); en consecuencia, podemos suponer que \( \alpha \) es una permutación impar. Como \( \sigma \) sólo cambia 3 elementos y \( n \geq 5 \), existe una trasposición \( \beta \) disjunta con \( \sigma \), por lo que \( \sigma^\beta = \sigma \). Por tanto
\[
\sigma^{\beta\alpha} = (\sigma^\beta)^\alpha = \sigma^\alpha = \sigma',
\]
y como \( \beta\alpha \) está en \( A_n \) por ser el producto de dos permutaciones impares, la normalidad de \( H \) en \( A_n \) implica que \( \sigma' \in H \), como queríamos ver.

Observese que la hipótesis \( n \geq 5 \) en el Lema anterior es superflua, pues para \( n \leq 3 \) es obvio que se verifica el Lema y para \( n = 4 \) es consecuencia del Ejemplo \ref{ex:subgrupos_a4}.
\end{proofbox}

\begin{theorem}{Teorema de Abel}{thm:abel}
Si \( n \geq 5 \), entonces \( A_n \) es un grupo simple.
\end{theorem}

\begin{proofbox}
Supongamos que \( H \neq \{1\} \) es un subgrupo normal de \( A_n \) y veamos que \( H = A_n \). Por el Lema \ref{lem:normal_con_3-ciclo}, bastará probar que \( H \) contiene un $3$-ciclo.

Sea \( 1 \neq \sigma \in H \) tal que \( r = |M(\sigma)| \) sea mínimo, es decir, \( |M(\nu)| \leq r \) para todo \( 1 \neq \nu \in H \). Ahora veremos que debe tenerse \( r = 3 \), por lo que \( \sigma \) será un $3$-ciclo en \( H \) y habremos terminado.

Desde luego, no puede ser \( r = 1 \) porque ninguna permutación cambia exactamente un elemento, ni tampoco \( r = 2 \) porque todas las permutaciones de \( H \) son pares. Supongamos pues, en busca de una contradicción, que \( r > 3 \). Se tienen entonces dos posibilidades:

\begin{enumerate}
    \item Que, en la factorización de \( \sigma \) en ciclos disjuntos, aparezca alguno de longitud \( \geq 3 \).
    \item Que \( \sigma \) sea un producto de (al menos dos) trasposiciones disjuntas.
\end{enumerate}

\textbf{Caso 1.} \( \sigma \) debe cambiar al menos 5 elementos (si sólo cambiase 4, como en la factorización de \( \sigma \) aparece un ciclo de longitud \( \geq 3 \), \( \sigma \) sería un $4$-ciclo, lo que contradice el hecho de que \( \sigma \in A_n \)). Podemos suponer, sin pérdida de generalidad (¿por qué?), que \( 1,2,3,4,5 \in M(\sigma) \) y que alguno de los ciclos disjuntos que componen \( \sigma \) es de la forma \( (1 \; 2 \; 3 \; \dots) \) (con longitud al menos 3). Sea \( \alpha = (3 \; 4 \; 5) \). Como \( \alpha \in A_n \) y \( H \) es normal en \( A_n \), deducimos que \( \sigma^\alpha \in H \), y así \( \beta = \sigma^{-1} \sigma^\alpha \in H \). Si \( \sigma(i) = i \) entonces \( i > 5 \) y por tanto \( \alpha(i) = i \), de donde se sigue que \( \beta(i) = i \); por tanto \( M(\beta) \subseteq M(\sigma) \), y la inclusión es estricta pues \( \sigma(1) = 2 \) mientras que \( \beta(1) = 1 \). En consecuencia, \( \beta \in H \) cambia menos de \( r \) elementos, así que debe ser \( \beta = 1 \), por la elección de \( r \). Esto significa que \( \sigma^\alpha = \sigma \), y por tanto \( \alpha\sigma = \sigma\alpha \). Pero esto es falso, pues \( \alpha\sigma(2) = 4 \) y \( \sigma\alpha(2) = 3 \), de manera que la primera de las dos posibilidades consideradas nos lleva a una contradicción.

\textbf{Caso 2.} Reordenando los elementos de \( N_n \) podemos asumir que \( \sigma = (1 \; 2)(3 \; 4) \cdots \) (puede haber más trasposiciones en el producto o no). Sea de nuevo \( \alpha = (3 \; 4 \; 5) \). Como antes, tomamos \( \beta = \sigma^{-1} \sigma^\alpha \in H \). Si \( i \neq 5 \) y \( \sigma(i) = i \) entonces \( i \neq 3,4,5 \) y por tanto \( \alpha(i) = i \), de donde se sigue que \( \beta(i) = i \); por tanto \( M(\beta) \subseteq M(\sigma) \cup \{5\} \). Pero el 1 y el 2 son fijados por \( \beta \) y cambiados por \( \sigma \), de modo que \( \beta \) cambia menos de \( r \) elementos y así \( \beta = 1 \), o sea \( \sigma\alpha = \alpha\sigma \). Sin embargo \( \sigma\alpha(3) = 3 \neq 5 = \alpha\sigma(3) \). En cualquier caso, pues, llegamos a la contradicción que buscábamos.
\end{proofbox}