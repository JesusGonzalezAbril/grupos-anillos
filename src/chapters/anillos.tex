\chapter{Anillos}

\section{Anillos}

\defn{Anillo}{anillo}{
    Un anillo es una terna \((A, +, \cdot)\) formada por un conjunto no vacío \(A\) y dos operaciones \(+\) (suma) y \(\cdot\) (producto) en \(A\) que verifican:
    \begin{enumerate}
        \item \((A, +)\) es un grupo abeliano.
        \item \((A, \cdot)\) es un monoide.
        \item Distributiva del producto respecto de la suma: \(a \cdot (b + c) = (a \cdot b) + (a \cdot c)\) para todo \(a, b, c \in A\).
    \end{enumerate}
    
    Si además \(\cdot\) es conmutativo en \(A\), decimos que \((A, +, \cdot)\) es un anillo conmutativo.
}

\rmk{
    \begin{itemize}
        \item El neutro de \(A\) con respecto a \(+\) se llama \textbf{cero} y se denota \(0\).
        \item El neutro de \(A\) con respecto a \(\cdot\) se llama \textbf{uno} y se denota \(1\).
        \item El simétrico de un elemento \(a\) con respecto a \(+\) se llama \textbf{opuesto} y se denota \(-a\).
        \item Si \(a\) es invertible con respecto a \(\cdot\), su simétrico se llama \textbf{inverso} y se denota \(a^{-1}\).
        \item En general para $+$ y $\cdot$ usamos la notación usual para sumas y productos
        \[
        a \cdot (b + c) = a(b+c)=ab+ac.
        \] 
    \end{itemize}
}

Como $(A, +)$ es un grupo, todo elemento de A es invertible respecto de la suma y por tanto cancelable. Diremos que un elemento de $A$ es regular en $A$ si es cancelable con respecto al producto. En caso contrario decimos que el elemento es singular en $A$ o divisor de cero. El termino divisor de cero se justifica por lo siguiente. Supongamos que $x \in A$ no es cancelable respecto al producto, en tal caso existen dos elementos distintos $a \neq b$ tales que $ax = bx$. Pero entonces es inmediato que
\[
(a-b)x = 0
\]
sin embargo, ni $(a-b)$ ni $x$ son cero, por lo que podemos interpretar que ambos son <<divisores del cero>>.

\subsection{Ejemplos de anillo}
\ex{
    Los conjuntos $\Z$, $\Q$, $\R$ y $\C$ son anillos conmutativos con la suma y el producto usuales. Notemos que todo elemento no nulo de $\Q$, $\R$ o $\C$ es invertible. Sin embargo, en $\Z$ solo hay dos elementos invertibles (1 y -1) aunque todos los elementos son regulares menos el 0.
}
\pf{
    Demostrar que se trata de anillos conmutativos es muy sencillo, basta comprobar que se verifican todas las propiedades pertinentes.
    
    Probaremos que en $\C$ todos los elementos salvo el 0 son invertibles, el resto de afirmaciones quedan como ejercicio. Sea $z = a + bi$ un número complejo cualquiera no nulo, en tal caso el número $w = \frac{a-bi}{a^2+b^2}$ verifica
    \[
    zw = \frac{(a+bi)(a-bi)}{a^2+b^2} = \frac{a^2 -abi +abi -b(-1)}{a^2+b^2} = \frac{a^2+b^2}{a^2+b^2} = 1
    \]
    luego $w = z^{-1}$ y por tanto $z$ tiene inverso.
}

\ex{
    Sean $A$ y $B$ dos anillos. Entonces el producto cartesiano $A \times B$ tiene una estructura de anillo con las operaciones definidas componente a componente:
    \[
    (a_1, b_1) + (a_2, b_2) = (a_1 + a_2, b_1 + b_2)
    \]
    \[
    (a_1, b_1) \cdot (a_2, b_2) = (a_1 \cdot a_2, b_1 \cdot b_2)
    \]
    Obsérvese que $A \times B$ es conmutativo si y solo si lo son $A$ y $B$, y que esta construcción se puede generalizar a productos cartesianos de cualquier familia (finita o no) de anillos.
}

\clearpage

\ex{
    Dados un anillo $A$ y un conjunto $X$, el conjunto $A^X$ de las aplicaciones de $X$ en $A$ es un anillo con las siguientes operaciones:
    \[
    (f + g)(x) = f(x) + g(x)
    \]
    \[
    (f \cdot g)(x) = f(x) \cdot g(x)
    \]
    Si definimos la familia de conjuntos $\{A_i = A : i \in X\}$ entonces es inmediato que $\cup_{i\in X} A_i = A$. Recordemos ahora que el producto $\prod_{i \in X} A_i$ es el conjunto de funciones $f: X \to \cup_{i\in X} A_i$, es decir, el conjunto de funciones $f : X \to A$, luego $A^X$ es un anillo correspondiente a un producto <<infinito>> del anillo $A$ consigo mismo. Para más información ver la Definición \ref{defn:producto-cartesiano}.
}

\ex{
    Dado un anillo \( A \), un polinomio en una indeterminada es una expresión:
    \[
    P = a_0 + a_1 X + a_2 X^2 + \cdots + a_n X^n,
    \]
    donde \( n \geq 0 \) y \( a_i \in A \) para todo \( i \). El conjunto de polinomios con coeficientes en \( A \) se denota \( A[X] \). La suma y producto en \( A[X] \) se definen de la forma usual.
}

\ex{
    Dado un anillo \( A \), denotamos por \( A[[X]] \) el conjunto de sucesiones \((a_0, a_1, a_2, \ldots)\) de elementos de \( A \). Con las operaciones:
    \[
    (a_0, a_1, \ldots) + (b_0, b_1, \ldots) = (a_0 + b_0, a_1 + b_1, \ldots),
    \]
    \[
    (a_0, a_1, \ldots)(b_0, b_1, \ldots) = (a_0 b_0, a_0 b_1 + a_1 b_0, \ldots),
    \]
    \( A[[X]] \) es un anillo llamado anillo de series de potencias con coeficientes en \( A \).
}

\clearpage
\subsection{Propiedades de los anillos}

\lemp{}{prop-anillos}{
    Sea $A$ un anillo y sean $a,b,c \in A$. Se verifican las siguientes propiedades
    \begin{enumerate}
        \item Todo elemento de $A$ es cancelable respecto de la suma.
        \item Todo elemento invertible de $A$ es regular en $A$.
        \item Si $b + a = a$ entonces $b=0$. Si $ba=a$ para todo $a$, entonces $b=1$. En particular, el cero y uno son únicos.
        \item El opuesto de $a$ es único y si $a$ es invertible, entonces $a$ tiene un único inverso.
        \item $0a=0=a0$.
        \item $a(-b)=(-a)b=-(ab)$.
        \item $a(b-c)=ab-ac$.
        \item $a$ y $b$ son invertibles si y solo si $ab$ y $ba$ son invertibles. En tal caso $(ab)^{-1}=b^{-1}a^{-1}$.
        \item Si $0=1$ entonces $A=\{0\}$.
    \end{enumerate}
}{
    \begin{enumerate}
        \item Como $A$ es un grupo respecto de la suma todo elemento tiene inverso, y por la Proposición \ref{prop:prop-operaciones} todo elemento invertible (respecto a la suma) es cancelable (respecto a la suma).
        \item De nuevo por la Proposición \ref{prop:prop-operaciones} todo elemento invertible (respecto al producto) es cancelable (respecto al producto).
        \item Si $b + a = a$ entonces como $a$ es cancelable por el apartado 1, tenemos $b=0$. Si $ba=a$ para todo $a$, entonces como el neutro es único $b=1$.
        \item De nuevo se sigue de la Proposición \ref{prop:prop-operaciones}.
        \item Basta aplicar un pequeño truco
        \[
        0a=(0+0)a = 0a + 0a \implies 0 = 0a
        \]
        para el caso $a0$ se procede igual.
        \item Basta notar que
        \[
        ab + a(-b) = a(b-b) = 0, ab + (-a)b = (a-a)b = 0 \implies -(ab)=a(-b) = (-a)b 
        \]
        ya que los opuestos son únicos.
        \item $a(b-c)=a(b+(-c))=ab+a(-c)=ab+(-ac)=ab-ac$.
        \item En primer lugar si $a,b$ son invertibles entonces existen $a^{-1},b^{-1}$ y es fácil ver que
        \[
        ab(b^{-1}a^{-1})=e=(b^{-1}a^{-1})ab, ba(a^{-1}b^{-1})=e=(a^{-1}b^{-1})ba
        \]
        luego $ab,ba$ son invertibles. Para el recíproco, si $ab,ba$ son invertibles entonces
        \[
        a(b(ab)^{-1})=ab(ab)^{-1}=e, ((ba)^{-1}b)a=(ba)^{-1}ba=e
        \]
        por tanto, por la Proposición \ref{prop:prop-operaciones} ambos simétricos $b(ab)^{-1},(ba)^{-1}b$ son iguales (ambos son $a^{-1}$) y $a$ es invertible. Para ver que $b$ es invertible se procede igual.
        \item Si $0=1$ entonces dado $x \in A$ tenemos
        \[
        x = x1 = x0 = 0 \implies A=\{0\}.
        \]
    \end{enumerate}
}

Dados un anillo $A$, un elemento $a \in A$ y un entero positivo $n$, la notación $na$ (respectivamente $a^n$) representa el resultado de sumar (respectivamente multiplicar) a consigo mismo $n$ veces, y si $n = 0$ convenimos que $0a = 0$ y $a^0 = 1$. Más rigurosamente, a partir de estas últimas igualdades se definen $na$ y $a^n$ de forma recurrente poniendo $(n + 1)a = a + na$ y $a^{n+1} = aa^n$ para $n \geq 0$. Por último, si $n \geq 1$ se define $(-n)a = -(na)$, y si además $a$ es invertible se define $a^{-n} = (a^{-1})^n$.

\lemp{}{multiplicacion-anillos}{
    Sea \( A \) un anillo, \( a, b \in A \), y \( m, n \in \mathbb{Z} \). Se verifican:
    \begin{enumerate}
        \item \( n (a + b) = n a + n b \).
        \item \( (n + m) a = n a + m a \).
        \item Si \( n, m \geq 0 \), entonces \( a^{n + m} = a^n a^m \). Si \( a \) es invertible, la igualdad vale para \( n, m \) arbitrarios.
        \item Si \( A \) es conmutativo y \( n \geq 0 \), entonces \( (a b)^n = a^n b^n \). Si \( a \) y \( b \) son invertibles, la igualdad vale para todo \( n \).
    \end{enumerate}
}{
    \begin{enumerate}
        \item Por inducción: el caso base $n=0$ es inmediato, si lo suponemos para $n$ entonces
        \[
        (n+1) (a + b) = (a+b)+ n a + n b = (n+1) a + (n+1) b.
        \]
        \item Basta aplicar recursivamente que $(n+1)a = a + na$.
        %%%% Faltan casos < 0 en los dos primeros apartados
        \item Basta aplicar recursivamente que $a^{n+1} = a a^n$. Si $a$ es invertible entonces podemos usar que $a^{-n}=(a^{-1})^n$ distinguiendo casos. Por ejemplo si $n>0,m<0, n>m$ entonces
        \[
        a^{n}a^{m}=a^n(a^{-1})^{-m}=a^{n+m}a^{-m}(a^{-1})^{-m}=a^{n+m}.
        \]
        \item Por inducción: el caso base $n=0$ es inmediato, si lo suponemos para $n$ entonces
        \[
        (ab)^{n+1}=ab(ab)^n=aba^nb^n=aa^nbb^n=a^{n+1}b^{n+1}.
        \]
        Cuando \( a \) y \( b \) son invertibles, si $n<0$
        \[
        (ab)^n=((ab)^{-1})^{-n}=(b^{-1}a^{-1})^{-n}=(b^{-1})^{-n}(a^{-1})^{-n}=b^n a^n.
        \]
    \end{enumerate}
}

\clearpage
\section{Subanillos}

\rmk{A partir de ahora supondremos que todos los anillos que aparecen son conmutativos.}

Sea $*$ una operación en un conjunto $A$ y sea $B$ un subconjunto de $A$. Decimos que $B$ es cerrado con respecto a $*$ si para todo $a, b \in B$ se verifica que $a * b \in B$. En tal caso podemos
considerar $*$ como una operación en $B$ que se dice inducida por la operación en $A$.

\defn{Subanillo}{subanillo}{
    Un subanillo de un anillo es un subconjunto suyo que con la misma suma y producto es un anillo con el mismo uno.
}

\prop{Caracterización de subanillos}{carac-subanillos}{
    Las siguientes condiciones son equivalentes para \( B \subseteq A \):
    \begin{enumerate}
    \item \( B \) es un subanillo de \( A \).
    \item \( B \) contiene al 1 y es un anillo, luego es cerrado para sumas, productos y opuestos.
    \item \( B \) contiene al 1 y es cerrado para restas y productos.
    \end{enumerate}
}{
    $(1) \implies (2)$: Si \( B \) es un subanillo de \(A\) entonces contiene al 1 y es cerrado para sumas y productos. Por otro lado, como $B$ es un anillo, tiene un cero, que de momento denotamos $0_B$ y cada elemento $b \in B$ tiene un opuesto en $B$. En realidad $0_B + 0_B = 0_B = 0 + 0_B$, con lo que aplicando la propiedad cancelativa de la suma deducimos que $0_B = 0$, o sea, el cero de $A$ está en $B$ y por tanto es el cero de B (el único que puede tener). Por la unicidad del opuesto, el opuesto de $b$ ha de ser el de $A$, con lo que $B$ es cerrado para opuestos.

    \noindent$(2) \implies (3)$: Inmediato.

    \noindent$(3) \implies (1)$: Si \( B \) contiene al 1 y es cerrado para restas, entonces \( 0 = 1 - 1 \in B \), y para \( b \in B \), \(-b = 0 - b \in B \). Además, \( a + b = a - (-b) \in B \), luego \( B \) es cerrado para sumas, por tanto, es un subanillo de $A$.
}

\ex{
    Todo anillo $A$ es un subanillo de si mismo, al que llamamos impropio por oposición al resto de subanillos, que se dicen propios.
}

\ex{
    En la cadena de contenciones $\Z \subset \Q \subset \R \subset \C$ cada uno es un subanillo de los posteriores.
}

\ex{
    Si $A$ es un anillo, el subconjunto $\{0\}$ es cerrado para sumas, productos y opuestos. Si $A = \{0\}$ entonces $\{0\}$ sería subanillo de $A$, pero este es el único caso en el que esto pasa pues en todos los demás casos $1 \neq 0$.

    En efecto si $1=0$ entonces para cualquier $a \in A, a = 1a = 0a = 0 \implies A = \{0\}$.
}

\clearpage
\section{Homomorfismos de anillos}

\defn{Homomorfismo de anillos}{hom-anillo}{
    Sean \(A\) y \(B\) dos anillos. Un homomorfismo de anillos entre \(A\) y \(B\) es una aplicación \(f: A \to B\) que satisface:
    \begin{enumerate}
        \item \(f(x + y) = f(x) + f(y)\)
        \item \(f(x \cdot y) = f(x) \cdot f(y)\)
        \item \(f(1) = 1\)
    \end{enumerate}
    
    Un isomorfismo es un homomorfismo biyectivo. Dos anillos \(A\) y \(B\) son isomorfos (\(A \cong B\)) si existe un isomorfismo entre ellos.
}

\rmk{
    En la definición anterior hemos usado el mismo símbolo para las operaciones en ambos anillos, pero es importante notar que:
    \begin{itemize}
        \item En \(f(x + y)\), la suma se realiza en \(A\)
        \item En \(f(x) + f(y)\), la suma se realiza en \(B\)
        \item Lo mismo aplica para el producto
    \end{itemize}
}

\defn{Tipos de homomorfismos}{tipos-hom}{
    \begin{itemize}
        \item Un \textbf{endomorfismo} es un homomorfismo de un anillo en sí mismo.
        \item Un \textbf{isomorfismo} es un homomorfismo biyectivo.
        \item Un \textbf{automorfismo} es un isomorfismo de un anillo en sí mismo.
    \end{itemize}
}

\ex{
    Si $B = \{0\}$ entonces la aplicación $f(a) = 0_B, \forall a \in A$ es un homomorfismo.
}

\prop{Propiedades básicas}{prop-hom-basicas}{
    Sea \(f: A \to B\) un homomorfismo de anillos. Entonces para todo \(a, b, a_1, \ldots, a_n \in A\) se verifica:
    \begin{enumerate}
        \item \(f(0_A) = 0_B\)
        \item \(f(-a) = -f(a)\)
        \item \(f(a - b) = f(a) - f(b)\)
        \item \(f(a_1 + \cdots + a_n) = f(a_1) + \cdots + f(a_n)\)
        \item \(f(na) = nf(a)\) para todo \(n \in \mathbb{Z}\)
        \item Si \(a\) es invertible en \(A\), entonces \(f(a)\) es invertible en \(B\) y \(f(a^{-1}) = f(a)^{-1}\)
        \item \(f(a_1 \cdots a_n) = f(a_1) \cdots f(a_n)\)
    \end{enumerate}
}{
    Demostremos algunas de estas propiedades:
    \begin{itemize}
        \item Para (1): \(f(0_A) = f(0_A + 0_A) = f(0_A) + f(0_A)\), luego por cancelación en \(B\), \(f(0_A) = 0_B\).
        
        \item Para (2): \(f(a) + f(-a) = f(a + (-a)) = f(0_A) = 0_B\), luego \(f(-a) = -f(a)\).
        
        \item Para (6): Si \(a\) es invertible, \(aa^{-1} = 1_A\), luego \(f(a)f(a^{-1}) = f(aa^{-1}) = f(1_A) = 1_B\), por tanto \(f(a^{-1}) = f(a)^{-1}\).
    \end{itemize}
}

\defn{Núcleo e imagen}{nucleo-imagen}{
    Sea \(f: A \to B\) un homomorfismo de anillos. Definimos:
    \begin{itemize}
        \item El {núcleo} de \(f\): \(\ker f = \{a \in A : f(a) = 0_B\}\)
        \item La {imagen} de \(f\): \(\operatorname{Im} f = \{f(a) \in B : a \in A\}\)
    \end{itemize}
}

\prop{Propiedades del núcleo e imagen}{prop-nucleo-imagen}{
    Sea \(f: A \to B\) un homomorfismo de anillos. Entonces:
    \begin{enumerate}
        \item \(\ker f\) es un ideal de \(A\)
        \item \(\operatorname{Im} f\) es un subanillo de \(B\)
        \item \(f\) es inyectivo si y solo si \(\ker f = \{0_A\}\)
        \item \(f\) es suprayectivo si y solo si \(\operatorname{Im} f = B\)
    \end{enumerate}
}{
    \begin{enumerate}
        \item Para ver que \(\ker f\) es un ideal:
        \begin{itemize}
            \item \(0_A \in \ker f\) pues \(f(0_A) = 0_B\)
            \item Si \(x, y \in \ker f\), entonces \(f(x + y) = f(x) + f(y) = 0_B + 0_B = 0_B\), luego \(x + y \in \ker f\)
            \item Si \(x \in \ker f\) y \(a \in A\), entonces \(f(ax) = f(a)f(x) = f(a) \cdot 0_B = 0_B\), luego \(ax \in \ker f\)
        \end{itemize}
        
        \item Para la inyectividad: si \(f\) es inyectivo y \(x \in \ker f\), entonces \(f(x) = 0_B = f(0_A)\), luego \(x = 0_A\). Recíprocamente, si \(\ker f = \{0_A\}\) y \(f(a) = f(b)\), entonces \(f(a - b) = 0_B\), luego \(a - b \in \ker f = \{0_A\}\), por tanto \(a = b\).
    \end{enumerate}
}

\subsection{Ejemplos importantes de homomorfismos}

\ex{Homomorfismo inclusión

    Si \(B\) es un subanillo de \(A\), la aplicación inclusión \(i: B \hookrightarrow A\) dada por \(i(b) = b\) es un homomorfismo inyectivo.
}

\ex{Homomorfismo proyección

    Si \(I\) es un ideal de \(A\), la proyección canónica \(\pi: A \to A/I\) dada por \(\pi(a) = a + I\) es un homomorfismo suprayectivo con \(\ker \pi = I\).
}

\ex{Homomorfismo de sustitución

    Sea \(A\) un anillo y \(b \in A\). La aplicación \(\varphi_b: A[X] \to A\) dada por:
    \[
    \varphi_b(a_0 + a_1X + \cdots + a_nX^n) = a_0 + a_1b + \cdots + a_nb^n
    \]
    es un homomorfismo suprayectivo llamado \textbf{homomorfismo de sustitución} en \(b\).
}

\ex{Homomorfismo único \(\mathbb{Z} \to A\)

    Para cualquier anillo \(A\), existe un único homomorfismo \(f: \mathbb{Z} \to A\) dado por \(f(n) = n \cdot 1_A\). Este homomorfismo está determinado por la característica del anillo \(A\).
}

\ex{Conjugación en \(\mathbb{C}\)

    La conjugación compleja \(f: \mathbb{C} \to \mathbb{C}\) dada por \(f(a + bi) = a - bi\) es un automorfismo de \(\mathbb{C}\).
}

\subsection{Propiedades de preservación}

\prop{Preservación de subestructuras}{preservacion-subestructuras}{
    Sea \(f: A \to B\) un homomorfismo de anillos.
    \begin{enumerate}
        \item Si \(A_1\) es un subanillo de \(A\), entonces \(f(A_1)\) es un subanillo de \(B\)
        \item Si \(B_1\) es un subanillo de \(B\), entonces \(f^{-1}(B_1)\) es un subanillo de \(A\)
        \item Si \(I\) es un ideal de \(B\), entonces \(f^{-1}(I)\) es un ideal de \(A\)
    \end{enumerate}
}

\rmk{
    \textbf{¡Cuidado!} La imagen de un ideal por un homomorfismo no necesariamente es un ideal, a menos que el homomorfismo sea suprayectivo.
}

\ex{Contraejemplo

    Sea \(i: \mathbb{Z} \hookrightarrow \mathbb{Q}\) la inclusión. El conjunto \(2\mathbb{Z}\) es un ideal de \(\mathbb{Z}\), pero \(i(2\mathbb{Z}) = 2\mathbb{Z}\) no es un ideal de \(\mathbb{Q}\), pues por ejemplo \(1 \in \mathbb{Q}\) y \(2 \in 2\mathbb{Z}\), pero \(1 \cdot 2 = 2 \notin 2\mathbb{Z}\) en \(\mathbb{Q}\).
}

\subsection{Composición y propiedades funtoriales}

\prop{Composición de homomorfismos}{comp-hom}{
    Si \(f: A \to B\) y \(g: B \to C\) son homomorfismos de anillos, entonces la composición \(g \circ f: A \to C\) es un homomorfismo de anillos.
}{
    Queda como ejercicio.
}

\prop{Propiedades de isomorfismos}{prop-iso}{
    \begin{enumerate}
        \item La composición de isomorfismos es un isomorfismo
        \item Si \(f: A \to B\) es un isomorfismo, entonces \(f^{-1}: B \to A\) es un isomorfismo
        \item La relación "ser isomorfo" es una relación de equivalencia en la clase de todos los anillos
    \end{enumerate}
}{
    Queda como ejercicio.
}

\subsection{Homomorfismos y productos}

\thm{Homomorfismos en productos}{hom-productos}{
    Sean \(A, B, C\) anillos. Existe una biyección natural:
    \[
    \operatorname{Hom}(A, B \times C) \cong \operatorname{Hom}(A, B) \times \operatorname{Hom}(A, C)
    \]
    dada por \(f \mapsto (\pi_B \circ f, \pi_C \circ f)\), donde \(\pi_B\) y \(\pi_C\) son las proyecciones canónicas.
}

\ex{Aplicación

    Para determinar todos los homomorfismos \(f: \mathbb{Z} \to \mathbb{Z}_2 \times \mathbb{Z}_3\), basta determinar los homomorfismos \(\mathbb{Z} \to \mathbb{Z}_2\) y \(\mathbb{Z} \to \mathbb{Z}_3\) por separado.
}

\clearpage
\section{Ideales y anillos cociente}

\defn{Ideal}{ideal}{
    Un subconjunto \(I\) de un anillo \(A\) es un \textbf{ideal} si:
    \begin{enumerate}
        \item \(I \neq \emptyset\)
        \item Para todo \(x, y \in I\), se verifica que \(x + y \in I\)
        \item Para todo \(x \in I\) y \(a \in A\), se verifica que \(ax \in I\)
    \end{enumerate}
}

\rmk{
    \begin{itemize}
        \item La condición \(I \neq \emptyset\) puede sustituirse por \(0 \in I\), ya que si \(a \in I\) entonces \(0 = a + (-1)a \in I\)
        \item Si \(I\) es un ideal de \(A\), entonces para todo \(a_1, \ldots, a_n \in A\) y \(x_1, \ldots, x_n \in I\) se tiene que \(\sum_{i=1}^n a_i x_i \in I\)
    \end{itemize}
}

\subsection{Ejemplos de ideales}

\ex{Ideales triviales:

    \begin{itemize}
        \item El \textbf{ideal cero}: \(\{0\}\)
        \item El \textbf{ideal impropio}: \(A\)
    \end{itemize}
}

\ex{Ideales principales:

    Sea \(A\) un anillo y \(b \in A\). El conjunto:
    \[
    (b) = bA = \{ba : a \in A\}
    \]
    es un ideal de \(A\) llamado \textbf{ideal principal generado por \(b\)}.
    
    Observaciones:
    \begin{itemize}
        \item \((1) = A\)
        \item \((0) = \{0\}\)
        \item \((b)\) es el menor ideal de \(A\) que contiene a \(b\)
    \end{itemize}
}

\clearpage

\ex{Ideal generado por un conjunto

    Sea \(T \subseteq A\). El \textbf{ideal generado por \(T\)} es:
    \[
    (T) = \left\{\sum_{i=1}^n a_i t_i : n \geq 0, a_i \in A, t_i \in T\right\}
    \]
    Este es el menor ideal de \(A\) que contiene a \(T\).
}

\ex{Ideales en anillos producto

    Si \(A\) y \(B\) son anillos, entonces \(A \times \{0\} = \{(a, 0) : a \in A\}\) es un ideal de \(A \times B\).
}

\ex{Ideales en anillos de polinomios

    Sea \(A[X]\) el anillo de polinomios.
    \begin{itemize}
        \item \(I = \{a_1X + \cdots + a_nX^n : a_i \in A\}\) (polinomios sin término constante) es un ideal
        \item Si \(I\) es ideal de \(A\), entonces \(J = \{a_0 + a_1X + \cdots + a_nX^n : a_0 \in I\}\) es un ideal de \(A[X]\)
        \item \(I[X] = \{a_0 + a_1X + \cdots + a_nX^n : a_i \in I\}\) es un ideal de \(A[X]\)
    \end{itemize}
}

\subsection{Propiedades básicas de los ideales}

\prop{Intersección de ideales}{interseccion-ideales}{
    La intersección de cualquier familia de ideales de \(A\) es un ideal de \(A\).
}{
    Si $I_\alpha$ es una familia de ideales y $J = \cap_{\alpha \in X} I_\alpha$ entonces $0 \in J \implies J \neq \emptyset$. Además, para todo indice $\alpha$
    \[
    x,y \in J \implies x+y \in I_\alpha \implies x+y \in J
    \]
    y para cualquier $a \in A$
    \[
    x \in J \implies ax \in I_\alpha \implies ax \in J.
    \]
}

\prop{Ideales de \(\mathbb{Z}\)}{ideales-Z}{
    Todos los ideales de \(\mathbb{Z}\) son principales. Es decir, para todo ideal \(I \subset \mathbb{Z}\), existe \(n \in \mathbb{Z}\) tal que \(I = (n)\).
}{
    Sea \(I\) un ideal de \(\mathbb{Z}\). Si \(I=0\) entonces \(I=(0)\) con lo que \(I\) es principal. Supongamos que \(I\neq 0\) y sea \(n\in I\setminus 0\). Entonces \(-n\in I\), con lo que \(I\) tiene un elemento positivo, o sea \(I\cap\mathbb{N}\neq\emptyset\). Como \(\mathbb{N}\) está bien ordenado, \(I\) tiene un mínimo que denotamos como \(a\). Como \(a\in I\) se tiene que \((a)\subseteq I\).
    
    Para ver que se da la igualdad tomamos \(b\in I\) y sean \(q\) y \(r\) el cociente y el resto de la división entera de \(b\) entre \(a\). Entonces \(b=qa+r\) y \(0\leq r<a\). Pero \(r=b-qa\in I\), por que \(I\) es un ideal de \(\mathbb{Z}\) que contiene a \(a\) y \(b\) y \(q\in\mathbb{Z}\). Como \(r\) es estrictamente menor que \(a\) y \(a\) es mínimo en \(I\cap\mathbb{N}\), necesariamente \(r\not\in\mathbb{N}\), es decir \(r\) no es positivo. Luego \(r=0\), con lo que \(b=qa\in(a)\).
 }

\defn{Congruencia módulo un ideal}{congruencia-ideal}{
    Sea \(I\) un ideal de un anillo \(A\). Decimos que \(a, b \in A\) son \textbf{congruentes módulo \(I\)}, y escribimos \(a \equiv b \pmod{I}\), si \(b - a \in I\).
}

\lem{Propiedades de la congruencia}{prop-congruencia}{
    Sea \(I\) ideal de \(A\). Para todo \(a, b, c, d \in A\):
    \begin{enumerate}
        \item \(a \equiv a \pmod{I}\) (reflexiva)
        \item Si \(a \equiv b \pmod{I}\), entonces \(b \equiv a \pmod{I}\) (simétrica)
        \item Si \(a \equiv b \pmod{I}\) y \(b \equiv c \pmod{I}\), entonces \(a \equiv c \pmod{I}\) (transitiva)
        \item \(a \equiv b \pmod{(0)}\) si y solo si \(a = b\)
    \end{enumerate}
}

\rmk{
    La congruencia módulo \(I\) es una relación de equivalencia en \(A\).
}

\subsection{Anillos cociente}

\defn{Anillo cociente}{anillo-cociente}{
    Sea \(I\) un ideal de \(A\). El conjunto de clases de equivalencia:
    \[
    A/I = \{a + I : a \in A\}
    \]
    con las operaciones:
    \begin{align*}
        (a + I) + (b + I) &= (a + b) + I \\
        (a + I) \cdot (b + I) &= (ab) + I
    \end{align*}
    es un anillo llamado \textbf{anillo cociente de \(A\) módulo \(I\)}.
}

\prop{Buena definición del cociente}{bien-def-cociente}{
    Las operaciones en \(A/I\) están bien definidas y dotan a \(A/I\) de estructura de anillo con:
    \begin{itemize}
        \item Elemento cero: \(0 + I\)
        \item Elemento uno: \(1 + I\)
    \end{itemize}
}{
    Sean \(a + I = a' + I\) y \(b + I = b' + I\). Entonces \(a - a', b - b' \in I\). Luego:
    \begin{itemize}
        \item \((a + b) - (a' + b') = (a - a') + (b - b') \in I\)
        \item \(ab - a'b' = ab - ab' + ab' - a'b' = a(b - b') + (a - a')b' \in I\)
    \end{itemize}
    Por tanto las operaciones están bien definidas.
}

\defn{Proyección canónica}{proyeccion-canonica}{
    La aplicación \(\pi: A \to A/I\) dada por \(\pi(a) = a + I\) es un homomorfismo suprayectivo llamado \textbf{proyección canónica}.
}

\subsection{Ejemplos de anillos cociente}

\ex{Anillos \(\mathbb{Z}_n\)

    Para \(n > 0\), \(\mathbb{Z}_n = \mathbb{Z}/(n)\) es el anillo cociente. Tiene exactamente \(n\) elementos: \(0 + (n), 1 + (n), \ldots, (n-1) + (n)\).
}

\clearpage

\ex{Cocientes triviales

    \begin{itemize}
        \item \(A/\{0\} \cong A\)
        \item \(A/A \cong \{0\}\)
    \end{itemize}
}

\ex{Cociente por ideales en polinomios

    Sea \(I = \{a_1X + \cdots + a_nX^n\} \subseteq A[X]\) (polinomios sin término constante). Entonces:
    \[
    A[X]/I \cong A
    \]
    mediante el isomorfismo que envía \(P(X) + I\) al término constante de \(P\).
}

\ex{Cociente en productos

    Sean \(A, B\) anillos, \(I = A \times \{0\}\). Entonces:
    \[
    (A \times B)/I \cong B
    \]
}

\subsection{Caracterización de ideales maximales}

\lemp{Caracterización de ideales impropios}{caract-ideales-impropios}{
    Sea \(A\) un anillo. Para un ideal \(I \subseteq A\), las siguientes condiciones son equivalentes:
    \begin{enumerate}
        \item \(I = A\) (ideal impropio)
        \item \(1 \in I\)
        \item \(I\) contiene una unidad de \(A\) (i.e., \(I \cap A^* \neq \emptyset\))
    \end{enumerate}
}{
    \begin{itemize}
        \item (1) \(\Rightarrow\) (2): si \(I = A\), entonces \(1 \in I\)
        \item (2) \(\Rightarrow\) (3): \(1\) es una unidad
        \item (3) \(\Rightarrow\) (1): si \(u \in I \cap A^*\), entonces \(1 = uu^{-1} \in I\), luego \(I = A\)
    \end{itemize}
}

\subsection{Núcleo y teorema de correspondencia}

\defn{Núcleo de un homomorfismo}{nucleo-homomorfismo}{
    Sea \(f: A \to B\) un homomorfismo de anillos. El \textbf{núcleo} de \(f\) es:
    \[
    \ker f = \{a \in A : f(a) = 0\}
    \]
}

\prop{Inyectividad y núcleo}{inyectividad-nucleo}{
    Un homomorfismo \(f: A \to B\) es inyectivo si y solo si \(\ker f = \{0\}\).
}{
    \begin{itemize}
        \item Si \(f\) es inyectivo y \(a \in \ker f\), entonces \(f(a) = 0 = f(0)\), luego \(a = 0\)
        \item Si \(\ker f = \{0\}\) y \(f(a) = f(b)\), entonces \(f(a - b) = 0\), luego \(a - b \in \ker f = \{0\}\), por tanto \(a = b\)
    \end{itemize}
}

\thmp{Teorema de correspondencia}{teorema-correspondencia}{
    Sea \(I\) un ideal de un anillo \(A\). Las asignaciones:
    \begin{align*}
        J &\mapsto J/I \\
        X &\mapsto \pi^{-1}(X)
    \end{align*}
    definen biyecciones (una inversa de la otra) que preservan la inclusión entre:
    \begin{itemize}
        \item El conjunto de ideales de \(A\) que contienen a \(I\)
        \item El conjunto de todos los ideales de \(A/I\)
    \end{itemize}
}{
    Basta verificar:
    \begin{itemize}
        \item Si \(J\) es ideal de \(A\) con \(I \subseteq J\), entonces \(J/I\) es ideal de \(A/I\) y \(\pi^{-1}(J/I) = J\)
        \item Si \(X\) es ideal de \(A/I\), entonces \(\pi^{-1}(X)\) es ideal de \(A\) que contiene a \(I\) y \(\pi^{-1}(X)/I = X\)
        \item Las asignaciones preservan inclusiones
    \end{itemize}
}

\ex{Aplicación del teorema de correspondencia

    En \(\mathbb{Z}_n = \mathbb{Z}/(n)\), los ideales son de la forma \(d\mathbb{Z}_n = (d)/(n)\) donde \(d \mid n\). Además, \(d\mathbb{Z}_n \subseteq d'\mathbb{Z}_n\) si y solo si \(d' \mid d\).
}

\clearpage

\section{Operaciones con ideales}

\defn{Suma de ideales}{suma-ideales}{
    Si \(I\) y \(J\) son ideales de \(A\), su \textbf{suma} es:
    \[
    I + J = \{x + y : x \in I, y \in J\}
    \]
}

\defn{Producto de ideales}{producto-ideales}{
    Si \(I\) y \(J\) son ideales de \(A\), su \textbf{producto} es:
    \[
    IJ = \left\{\sum_{i=1}^n x_i y_i : x_i \in I, y_i \in J, n \geq 0\right\}
    \]
}

\rmk{
    Más generalmente, para ideales \(I_1, \ldots, I_n\):
    \begin{itemize}
        \item \(I_1 + \cdots + I_n = \{x_1 + \cdots + x_n : x_i \in I_i\}\)
        \item \(I_1 \cdots I_n\) está generado por productos \(x_1 \cdots x_n\) con \(x_i \in I_i\)
    \end{itemize}
}

\prop{Propiedades de las operaciones}{prop-operaciones-ideales}{
    Para ideales \(I, J, K\) de \(A\):
    \begin{enumerate}
        \item \(IJ \subseteq I \cap J\)
        \item \(I(J \cap K) \subseteq IJ \cap IK\)
        \item \(I(JK) = (IJ)K\)
        \item \(I(J + K) = IJ + IK\)
        \item \(IA = I\)
    \end{enumerate}
}{
    Ejercicio.
}

\ex{Operaciones en \(\mathbb{Z}\)

    Sean \((n)\) y \((m)\) ideales de \(\mathbb{Z}\). Entonces:
    \begin{align*}
        (n)(m) &= (nm) \\
        (n) \cap (m) &= (\mathrm{mcm}(n, m)) \\
        (n) + (m) &= (\mathrm{mcd}(n, m))
    \end{align*}
}

\ex{Ideal no principal

    En \(\mathbb{Z}[X]\), el ideal \((2) + (X)\) (polinomios con término constante par) no es principal.
}

\pf{
    Supongamos que \((2) + (X) = (a)\) para algún \(a \in \mathbb{Z}[X]\). Entonces:
    \begin{itemize}
        \item \(2 = ab\) para algún \(b\), luego \(a \in \mathbb{Z}\)
        \item Como \(a \in (2, X)\), debe ser \(a\) par
        \item Pero entonces \(X \notin (a)\), contradicción
    \end{itemize}
}

\subsection{Ideales primos y maximales}

\defn{Ideal primo}{ideal-primo}{
    Un ideal \(P \subsetneq A\) es \textbf{primo} si para todo \(a, b \in A\):
    \[
    ab \in P \Rightarrow a \in P \text{ o } b \in P
    \]
}

\defn{Ideal maximal}{ideal-maximal}{
    Un ideal \(M \subsetneq A\) es \textbf{maximal} si no existe ningún ideal \(I\) tal que \(M \subsetneq I \subsetneq A\).
}

\lem{Caracterizaciones}{caract-primos-maximales}{
    \begin{enumerate}
        \item \(P\) es primo si y solo si \(A/P\) es un dominio de integridad
        \item \(M\) es maximal si y solo si \(A/M\) es un cuerpo
        \item Todo ideal maximal es primo
    \end{enumerate}
}

\ex{Ejemplos en \(\mathbb{Z}\)

    \begin{itemize}
        \item Los ideales primos de \(\mathbb{Z}\) son \((0)\) y \((p)\) con \(p\) primo
        \item Los ideales maximales de \(\mathbb{Z}\) son \((p)\) con \(p\) primo
    \end{itemize}
}

\clearpage
\section{Teoremas de isomorfía}

\thmp{Primer teorema de isomorfía}{iso1}{
    Sea \(f: A \to B\) un homomorfismo de anillos. Entonces existe un isomorfismo:
    \[
    A/\ker f \cong \operatorname{Im} f
    \]
}{}

\thmp{Segundo teorema de isomorfía}{iso2}{
    Sea \(A\) un anillo y \(I \subseteq J\) ideales de \(A\). Entonces:
    \[
    \frac{A/I}{J/I} \cong \frac{A}{J}
    \]
}{}

\thmp{Tercer teorema de isomorfía}{iso3}{
    Sea \(A\) un anillo, \(B\) un subanillo de \(A\) e \(I\) un ideal de \(A\). Entonces:
    \[
    \frac{B}{B \cap I} \cong \frac{B + I}{I}
    \]
}{}

\thmp{Teorema chino de los restos}{chino}{
    Sea \(A\) un anillo y \(I_1, \ldots, I_n\) ideales de \(A\) tales que \(I_i + I_j = A\) para todo \(i \neq j\). Entonces:
    \[
    \frac{A}{I_1 \cap \cdots \cap I_n} \cong \frac{A}{I_1} \times \cdots \times \frac{A}{I_n}
    \]
}{}
