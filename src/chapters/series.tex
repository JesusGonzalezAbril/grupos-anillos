\chapter{Series}

\section{Sucesiones en espacios métricos}

En esta primera sección trabajamos en un espacio métrico $(X,d)$ genérico. Escribimos las sucesiones como $(x_n)_{n \in \N}$, $(x_n)_n$ o $(x_n)$.

\defn{Sucesión convergente}{suc-conv}{
    Una sucesión $(x_n)_{n \in \N}$ se dice convergente si existe $p \in X$ tal que para todo $\varepsilon > 0$ existe $N_\varepsilon \in \N$ tal que
    \[
    n \geq N_\varepsilon \implies d(x_n, p) < \varepsilon.
    \]
    En tal caso escribiremos $\lim_n x_n = p$.
}

\defn{Sucesión de Cauchy}{suc-cauchy}{
    Una sucesión $(x_n)_{n \in \N}$ se dice de Cauchy si para todo $\varepsilon > 0$ existe $N_\varepsilon \in \N$ tal que
    \[
    n \geq m \geq N_\varepsilon \implies d(x_n, x_m) < \varepsilon.
    \]
}

\defn{Espacio completo}{completo}{
    Un espacio métrico se dice completo si toda sucesión de Cauchy es convergente.
}

\exer{
    Probar que toda subsucesión de un sucesión convergente es convergente.
}

\exer{
    Probar que toda sucesión convergente está acotada.
}

\pf{
    Sea $(x_n)$ una sucesión convergente a un punto $p$. Entonces fijado $\varepsilon = 1$ existe $N_0$ tal que si $n \geq N_0$ entonces $d(x_n,p) < 1$. Consideremos el conjunto de números reales
    \[
    S = \{ d(x_m,p) : m \leq N_0 \}
    \]
    que claramente tiene un número finito de elementos, y por tanto un máximo $M = \max S$. Finalmente basta tomar $r = \max \{M, 1\}$ lo cual garantiza que $\forall n \in \N$
    \[
    d(x_n,p) \geq r
    \]
    es decir, la sucesión está acotada.
}

\exer{
    Probar que toda sucesión convergente es de Cauchy.
}

\clearpage

\section{Límite superior y límite inferior de una sucesión real}

En lo que sigue trabajamos con sucesiones de números reales en el espacio métrico $(\R, d_u)$.

\defn{Límite superior}{lim-sup}{
    Sea $(x_n)_{n \in \N}$ una sucesión. Si existe un número real $U$ que satisface
    \begin{enumerate}
        \item Para cada $\varepsilon > 0$ existe $N \in \N$ tal que $n > N$ implica \[ x_n < U + \varepsilon. \]
        \item Dados $\varepsilon > 0$ y $m > 0$ existe $n > m$ tal que \[ x_n > U - \varepsilon. \]
    \end{enumerate}
    Entonces $U$ se llama el límite superior de $(x_n)$ y se escribe
    \[
    U = \lim_n \sup x_n.
    \]
}

% Mejor definición más larga del Apostol.

\defn{Límite inferior}{lim-inf}{
    Sea $(x_n)_{n \in \N}$ una sucesión. Si existe un número real $U$ que satisface
    \begin{enumerate}
        \item Para cada $\varepsilon > 0$ existe $N \in \N$ tal que $n > N$ implica \[ x_n > U - \varepsilon. \]
        \item Dados $\varepsilon > 0$ y $m > 0$ existe $n > m$ tal que \[ x_n < U + \varepsilon. \]
    \end{enumerate}
    Entonces $U$ se llama el límite inferior de $(x_n)$ y se escribe
    \[
    U = \lim_n \text{inf } x_n.
    \]
}

\exer{
    Probar que toda sucesión real tiene un límite superior y un límite inferior en el sistema de los números reales ampliado $\R \cup \{ -\infty, +\infty \}.
}

\pf{

}
